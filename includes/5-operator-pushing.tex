\chapter{全息张量网络中的算符推移}
\label{chap:operator-pushing}

\renewcommand{\arraystretch}{0.8}

在本章中,我们将沿着第 \ref{chap:strange-correlator} 章的思路,继续考察基于拓扑序的重整化张量网络。这类张量网络可以揭示出对应 CFT 的性质,并且在数值上还可以看到重整化张量与 AdS 体的相似性。因此,研究其中的算符推移及演生广义自由场的条件具有重要意义。

\section{广义自由场与算符推移}

AdS/CFT 对应\cite{maldacena1999large}中的一项重要特征是在 AdS 体 (bulk) 中演生出的\emph{自由场} (free field)。这使得我们可以为体态提供一套半经典的描述,同时也可以通过体来计算边界 CFT 中的物理量。这里所谓“自由场”,是指其关联函数可以通过 Wick 缩并来计算。一般来说,边界 CFT 中的\emph{单迹算符} (single trace operator) 近似对应了半经典的体态理论中的自由 Gauss 场。这就使得 CFT 中的关联函数可以在 AdS 一侧通过 Witten 图来计算\cite{witten1998anti,gubser1998gauge}。也有很多工作探讨了全息 CFT 中所演生出的广义自由场\cite{dutsch2003generalized,liu2019dimensional,collier2019quantum,nebabu2023bulk}。

另一方面,受 Ryu--Takayanagi 公式\cite{ryu2006holographic}的启发,也有人提出张量网络是构建 CFT 与 AdS 自由度之间线性映射的合适框架\cite{swingle2012entanglement}。如图~\ref{fig:mera-ads-cft} 所示,这类张量网络的图形表示也和 AdS 空间的离散版本非常相似。它描述了 CFT 边界自由度的重整化(粗粒近似)过程,而这些自由度同时也位于张量网络的渐进边界。体中的算符作用在张量网络的辅助指标上,而 CFT 中的算符则作用在张量网络的渐进边界上。因此,张量网络提供了这些算符之间的线性映射\cite{pastawski2015holographic,hayden2016holographic}。

\begin{figure}[htb]
  \centering
  \includegraphics[width=0.5\textwidth]{images/temp/mera-ads-cft.pdf}
  \caption[MERA 张量网络与 AdS/CFT]{MERA 张量网络可以被理解为 AdS/CFT 的一种离散实现。一维格点模型的基态波函数对应了 $\text{CFT}_{1+1}$ 的真空态,而由此“生长”出的二维 MERA 张量网络则给出了 $\text{AdS}_{2+1}$ 时间切片的一种离散表示。图片来源:\parencite{evenbly2011tensor}。}
  \label{fig:mera-ads-cft}
\end{figure}

由于张量网络实际上是局域化的,它可以分解为一系列张量的乘积,而这些张量只与邻近的张量相缩并(邻近张量的具体数目取决于体空间的维数),因而我们可以通过\emph{算符推移} (operator pushing) 的方法来找出相应的\emph{体—边映射} (bulk-boundary map)\cite{pastawski2015holographic,bhattacharyya2016exploring,bhattacharyya2018tensor}。

于是这就引出了一个自然的问题:对于一个弱耦合的体理论,其中的场几乎都是自由的,那么此时什么样的张量网络最能描述相应 AdS/CFT 中的体—边映射?文献 \parencite{bhattacharyya2018tensor} 对这个问题进行了一些讨论。如图~\ref{fig:operator-pushing} 所示,对于体中的算符 $\mathcal{O}$,它作用在体内部的自由度上;而当利用算符推移把 $\mathcal{O}$ 拉到边界时,这种作用应当等价于另一组算符 $\mathcal{O}'$ 在边界自由度上的作用。在利用边界算符对体算符进行重建时,我们有必要对入射、出射腿的方向做出选择。事实上,当张量网络对应了重整化(粗粒近似)操作时,这种选择是唯一的:作用在粗粒近似后的自由度上的算符,应当由作用在粗粒近似前的自由度上的算符来决定。此时可以证明,一个广义的自由场在利用算符推移穿过张量网络后,可以被分解为一系列\emph{简单算符} (simple operators) 之和\cite{bhattacharyya2018tensor}。

\begin{figure}[htb]
  \centering
  % TODO: operator-pushing
  \caption{算符推移}
  \label{fig:operator-pushing}
\end{figure}

考虑一个由重整化张量 $M^{i_1,\dots,i_K}_{j_1,\dots,j_H}\colon\mathbb{V}^K\to\mathbb{V}^H$ 构成的张量网络,其中 $K>H$,$i_l,j_l\in\mathbb{V}$,而 $\mathbb{V}$ 是一个 $d$ 维向量空间。那么算符推移对应于
\begin{equation}
  \mathcal{O} \bigl( \mathbb{V}^H \bigr) \cdot M = M \cdot \mathcal{O} \bigl( \mathbb{V}^K \bigr).
  \label{eq:operator-pushing}
\end{equation}
一个近似自由的体算符会作用在 $\mathbb{V}^H$ 其中一个指标上,即
\begin{equation}
    \mathcal{O} \bigl( \mathbb{V}^H \bigr)
  = \mathbb{I}_1 \otimes \cdots \otimes \mathbb{I}_{i-1} \otimes X_i \otimes \mathbb{I}_{i+1} \otimes \dots \otimes \mathbb{I}_{H}.
\end{equation}
当 $\mathcal{O}$ 满足式~\eqref{eq:operator-pushing} 时,则有
\begin{equation}
    \mathcal{O} \bigl( \mathbb{V}^K \bigr)
  = \sum_{i=1}^K \alpha_i \, \bigl(
      \mathbb{I}_1 \otimes \cdots \otimes \mathbb{I}_{i-1} \otimes X_i \otimes \mathbb{I}_{i+1} \otimes \dots \otimes \mathbb{I}_{K}
    \bigr),
  \label{eq:operator-pushing-coefficients}
\end{equation}
其中 $\{\alpha_i\}$ 为常数。此时,作用在体自由度 $l_B$ 上的体算符 $\mathcal{O}(l_B)$ 就可以用边界算符重建出来:
\begin{equation}
  \mathcal{O} \bigl( l_B \bigr) = \sum_b K^I \bigl( l_b, l_B \bigr) \, \mathcal{O}^I \bigl( l_b \bigr),
\end{equation}
其中 $\{\mathcal{O}^I\}$ 是作用在边界自由度 $l_b$ 上算符的一组完备基,而 $K^I(l_b, l_B)$ 是\emph{体—边核} (bulk-boundary kernel),它可用式~\eqref{eq:operator-pushing-coefficients} 中的系数 $\alpha_i$ 表示。可以发现,这一表达式与通过\emph{体—边传播子} (bulk-boundary propagator) 构造的 \emph{HKLL 核} (HKLL kernel)\cite{hamilton2006local,hamilton2006holographic}具有相同的形式。此外还可以证明,体算符关联函数的行为与广义自由场是类似的\cite{bhattacharyya2018tensor,hung2019padic}。由于 $M$ 是一个长方形的矩阵,所以这种重建不是唯一的。然而,只要式~\eqref{eq:operator-pushing-coefficients} 成立,那么广义自由场的解总是存在的。

\section{1+1 维中的算符推移}

我们首先讨论 1+1 维的情形,即由有限群 $G$ 所描述的 Dijkgraaf--Witten 理论。在 \ref{subsec:holographic-tensor-network-1+1d} 小节中,我们已经给出了重整化算符的构造。它对应了一个树状结构的张量网络,其中每个顶点都是三分支的。在无扭转的 Dijkgraaf--Witten 理论中,每个顶点对应了一个带有三个指标的张量 $M^{g_1g_2}_{g_3}$:
\begin{equation}
  \tikzinput{operator-pushing/tensor-1+1d},
\end{equation}
其中 $g_1$、$g_2$、$g_3$ 都是群 $G$ 中的元素。张量 $M$ 描述了 $G$ 中的群乘法(也相当于融合规则),即
\begin{equation}
  M^{ij}_k = \delta_{G(i,j), k}, \quad
  G(g_1, g_2) \coloneq g_1 \times g_2 = g_3.
\end{equation}

如图~\ref{fig:rg-1+1d} 所示,这一由重整化算富所构成的张量网络具有一维边界。当在体中插入一个算符 $B$ 时,它的作用应当等价于边界上插入的另外一些算符。找到能再现体算符作用的边界算符,也就是我们所说的算符推移或\emph{体算符重建} (bulk operator reconstruction) 问题。

由于张量网络是局域的,我们可以通过研究重整化张量网络中的一个张量来研究体算符重建。具体来说,对于一个张量单元而言,算符推移或重建相当于为给定的算符 $B$ 找到对应的算符 $A$,并使其满足
\begin{equation}
  \tikzinput{operator-pushing/constraint-1+1d}.
  \label{eq:constraint-1+1d}
\end{equation}
当算符 $B$ 是所谓广义自由场时,由 $A$ 给出的算符重建应当可以表示为“简单”形式(即 $\mathbb{I}\otimes\tilde{A}$ 或 $\tilde{A}\otimes\mathbb{I}$ 的线性组合)。为了确保 $B$ 的行为满足广义自由场的条件,我们还要求 $\tilde{A}$ 仍然属于算符集 $\{B\}$,也就是要求这一重建过程可以被不断重复至任意层的张量网络。式~\eqref{eq:constraint-1+1d} 可以表述为
\begin{equation}
  A_{(ij), (i'j')} M_{(i'j'), k} = M_{(ij), k'} B_{k', k}.
  \label{eq:constraint-equation-1+1d}
\end{equation}
为计算方便,我们把 $i$、$j$ 合并成了单个指标 $(ij)$,类似张量网络中的变形操作[式~\eqref{eq:tensor-reshape}]。接下来我们考虑对一般的算符 $B$ 求解式~\eqref{eq:constraint-equation-1+1d},并从中确定广义自由场的集合。还可以注意到,在张量网络中,如果算符 $A$ 能够表示为简单形式,则算符 $B$ 的集合将构成广义自由场的一组完备基。

在式~\eqref{eq:constraint-1+1d} 中张量的每条腿上,都有一个 $|G|$ 维的向量空间。作为 $\mathbb{Z}_2$ 情形中 Pauli 矩阵的推广,我们可以利用\emph{广义 Pauli 矩阵} (generalized Pauli matrices)\cite{patera1988pauli}作为基,来构建作用在每条腿上的算符。对于有限群 $G$ 且有 $|G|=n$,$G$ 中群元标记为 $0, 1, \dots, n-1$,则 $|G|$ 维的向量空间中的基可以表示为
\begin{equation}
  \ket{i} = \begin{pmatrix} 0 \\ \vdots \\ 1_{\text{$i$-th}} \\ \vdots \\ 0 \end{pmatrix}, \quad
  i \in \{ 0, 1, \dots, n-1 \}.
\end{equation}
在这组基下,广义 Pauli 矩阵可通过\emph{移位矩阵} (shift matrix) $X$ 和\emph{时钟矩阵} (clock matrix) $Z$ 生成:
\begin{equation}
  X = \begin{pmatrix}
    0      & 1      & 0      & \cdots & 0      \\
    0      & 0      & 1      & \cdots & 0      \\
    \vdots & \vdots & \vdots & \ddots & \vdots \\
    0      & 0      & 0      & \cdots & 1      \\
    1      & 0      & 0      & \cdots & 0
  \end{pmatrix}, \quad
  Z = \begin{pmatrix}
    1      & 0      & \cdots & 0            & 0      \\
    0      & \omega & \cdots & 0            & 0      \\
    \vdots & \vdots & \ddots & \vdots       & \vdots \\
    0      & 0      & \cdots & \omega^{n-2} & 0      \\
    0      & 0      & \cdots & 0            &\omega^{n-1}
  \end{pmatrix},
  \label{eq:generalized-pauli-matrices}
\end{equation}
这里 $\omega=\ee^{2\pi i/n}$ 是 $n$ 次单位根。于是广义 Pauli 矩阵为
\begin{equation}
  \sigma^\mu \coloneq \sigma^{ns+t} = X^t Z^s,
\end{equation}
其中 $s=\lfloor \mu/n\rfloor$、$t=\mu\bmod n$,而 $\mu\in\{0,\dots,n^2-1\}$。此时式~\eqref{eq:constraint-equation-1+1d} 可以写为
\begin{align}
     A_{(ij), (i'j')} \delta_{G(i',j'), k}
  &= \delta_{G(i,j), k'} B_{k', k} \notag \\
  &= \delta_{G(i,j), k'} \sigma^\mu_{k', k}
   = \sigma^\mu_{G(i,j), k}.
  \label{eq:1+1d-constraint-equation}
\end{align}
因此,一旦给定了群乘法 $G(i,j)$,我们就可以根据标准的线性代数方法,对于每一个 $\sigma_\mu$ 求解上式中的 $A$。完整的解系应包含通解和特解两个部分,其中通解部分对应于齐次方程
\begin{equation}
  A_{(ij), (i'j')} \delta_{G(i',j'), k} = 0
\end{equation}
的解,即相当于 $(\sigma^\mu_{G(i,j),k})^{\mathrm{T}}=\sigma^\mu_{k,G(i,j)}$ 的\emph{零空间} (null space)。而注意到
\begin{equation}
  G(i,0) = i, \quad G(0,j) = j,
\end{equation}
我们可以给出一组特解:
\begin{equation}
  A^{(\mu)}_{(ij), (i'j')} = \begin{cases}
    \sigma^\mu_{G(i,j), j'}, & i' = 0; \\
    0, & i' \neq 0.
  \end{cases}
  \label{eq:1+1d-specific-solution}
\end{equation}

我们还想找到能使算符 $A$ 为简单形式的体算符 $B$ 的子集,此时要求 $A$ 在张量一条腿上的作用是平凡的,即
\begin{equation}
  A_{(ij), (i'j')} = \tilde{A}_{ii'} \delta_{jj'} \quad \text{or} \quad
  A_{(ij), (i'j')} = \delta_{ii'} \tilde{A}_{jj'}.
\end{equation}
于是有
\begin{equation}
  \tilde{A}_{ii'} \delta_{G(i',j), k} = \sigma^\mu_{G(i,j), k} \quad \text{or} \quad
  \tilde{A}_{jj'} \delta_{G(i,j'), k} = \sigma^\mu_{G(i,j), k}, \quad
  \forall j \in \{ 0, 1, \dots, n-1 \}.
  \label{eq:1+1d-simple-form-constraint-equation}
\end{equation}
$\tilde{A}$ 仍然可以很容易地求解得到。根据线性代数中的 Rouch\'e--Capelli 定理,上式有解的充分必要条件是对于 $(\sigma^\mu_{G(i,j),k})^{\mathrm{T}}$ 的每一列,增广矩阵的秩都满足
\begin{equation}
    \rank\bigl[ (\delta_{G(i',j), k})^{\mathrm{T}} \big| (\sigma^\mu_{G(i,j),k})^{\mathrm{T}}_s \bigr]
  = \rank\bigl[ (\delta_{G(i',j), k})^{\mathrm{T}} \bigr].
\end{equation}
又因为
\begin{equation}
    \rank\bigl[ (\delta_{G(i',j), k})^{\mathrm{T}} \bigr]
  = \rank\bigl( \delta_{G(i',j), k} \bigr) = n,
\end{equation}
如果式~\eqref{eq:1+1d-simple-form-constraint-equation} 有解,也只存在一组解。

\subsection{Abelian 的例子:\texorpdfstring{$\mathbb{Z}_n$}{ℤₙ} 群}

接下来我们考察几个具体的例子。对于 $\mathbb{Z}_n$ 群,其融合规则由模算术给出:
\begin{equation}
  G(i,j) = (i+j)\bmod n,
  \label{eq:Z_n-fusion-rules}
\end{equation}
因此 $\delta_{G(i,j),k}=\delta_{(i+j)\bmod n,k}$ 是一个秩为 $n$ 的 $n^2\times n$ 矩阵,其转置矩阵的零空间可由 $n^2-n$ 个向量 $\mathbf{v}$ 张成:
\begin{equation}
  v^{(p)}_q = \delta_{(-p-\lfloor p/n\rfloor-2)\bmod n, \, q} - \delta_{n^2-p-1, \, q},
\end{equation}
其中
\begin{equation}
  p \in \left\{ 0, 1, \dots, n^2-n-1 \right\}, \quad
  q \in \left\{ 0, 1, \dots, n^2-1 \right\}.
\end{equation}
于是通解 $A^*$ 可由 $v^{(p)}$ 的线性组合给出:
\begin{equation}
  A^* = \begin{pmatrix}
    \beta_{0,0} \mathbf{v}^{(0)} + \dots + \beta_{0,n^2-n-1} \mathbf{v}^{(n^2-n-1)} \\
    \vdots \\
    \beta_{n^2-1,0} \mathbf{v}^{(0)} + \dots + \beta_{n^2-1,n^2-n-1} \mathbf{v}^{(n^2-n-1)}
  \end{pmatrix},
\end{equation}
其中 $\beta_{i,j}$ 是任意常数。而根据式~\eqref{eq:1+1d-specific-solution},特解部分为
\begin{equation}
  A^{(\mu)}_{(ij), (0j')} = \sigma^\mu_{(i+j)\bmod n, j'}.
\end{equation}

对于算符 $A$ 为简单形式的情形,我们注意到当且仅当 $\mu\in\{0,1,\dots,n-1\}$ 时式~\eqref{eq:1+1d-simple-form-constraint-equation} 有解,此时
\begin{equation}
  \tilde{A}^{(\mu)}_{ii'} = \sigma_\mu.
\end{equation}
这意味着当且仅当 $B=\sigma_k$ 而 $k\in\{1,2,\dots,n-1\}$ 时($\sigma_0=\mathbb{I}$ 对应了同样是平凡算符的 $A=\mathbb{I}\otimes\mathbb{I}$,因此将其忽略),对应的
\begin{equation}
  A = \sigma_k   \otimes \mathbb{I} \quad \text{or} \quad
  A = \mathbb{I} \otimes \sigma_k
\end{equation}
为简单算符。由于 $\tilde{A}=\sigma_k$ 可以作为新一层的体算符 $B$,这样的算符推移操作可以不断进行下去。考虑一个 $L$ 层的张量网络(见图~\ref{fig:rg-1+1d}),对于上述的 $B=\sigma_k$,边界算符仍然具有简单形式,并且可以写为:
\begin{equation}
  A_L = \mathbb{I}^{\otimes L-l} \otimes \sigma_k \otimes \mathbb{I}^{\otimes l-1}, \quad l = 0,\dots,L.
\end{equation}

下面我们以 $\mathbb{Z}_2$ 群(即 $n=2$)为例给出一个具体的解。$\delta_{k,(i+j)\bmod n}$ 的零空间可以由向量组
\begin{equation}
  \{ v^{(p)} \} = \Biggl\{ \,
    \begin{pmatrix} 1 \\ 0 \\ 0 \\ -1 \end{pmatrix}, \,
    \begin{pmatrix} 0 \\ 1 \\ -1 \\ 0 \end{pmatrix} \,
  \Biggr\}
\end{equation}
张成,这等价于通解
\begin{equation}
  A^* = \begin{pmatrix}
    \beta_{0,0} & \beta_{0,1} & -\beta_{0,1} & -\beta_{0,0} \\
    \beta_{1,0} & \beta_{1,1} & -\beta_{1,1} & -\beta_{1,0} \\
    \beta_{2,0} & \beta_{2,1} & -\beta_{2,1} & -\beta_{2,0} \\
    \beta_{3,0} & \beta_{3,1} & -\beta_{3,1} & -\beta_{3,0}
  \end{pmatrix},
\end{equation}
其中 $\beta_{i,j}$ 是任意常数。因而完整的解可以表示为
\begin{align}
  B = \sigma_0 = \mathbb{I}   &\implies A = A^* + \begin{pmatrix} 1 &  0 & 0 & 0 \\ 0 &  1 & 0 & 0 \\ 0 &  1 & 0 & 0 \\ 1 &  0 & 0 & 0 \end{pmatrix}, \notag \displaybreak[0] \\
  B = \sigma_1 = \sigma_x     &\implies A = A^* + \begin{pmatrix} 0 &  1 & 0 & 0 \\ 1 &  0 & 0 & 0 \\ 1 &  0 & 0 & 0 \\ 0 &  1 & 0 & 0 \end{pmatrix}, \notag \displaybreak[0] \\
  B = \sigma_2 = \sigma_z     &\implies A = A^* + \begin{pmatrix} 1 &  0 & 0 & 0 \\ 0 & -1 & 0 & 0 \\ 0 & -1 & 0 & 0 \\ 1 &  0 & 0 & 0 \end{pmatrix}, \notag \displaybreak[0] \\
  B = \sigma_3 = -\ii\sigma_y &\implies A = A^* + \begin{pmatrix} 0 & -1 & 0 & 0 \\ 1 &  0 & 0 & 0 \\ 1 &  0 & 0 & 0 \\ 0 & -1 & 0 & 0 \end{pmatrix}.
  \label{eq:Z_2-solution}
\end{align}
可以看到当且仅当体算符 $B=\sigma_x$ 时,边界算符 $A$ 可以取到简单形式
\begin{equation}
  A = \mathbb{I} \otimes \sigma_x = \begin{pmatrix}
    0 & 1 & 0 & 0 \\
    1 & 0 & 0 & 0 \\
    0 & 0 & 0 & 1 \\
    0 & 0 & 1 & 0
  \end{pmatrix}
  \quad \text{or} \quad
  A = \sigma_x \otimes \mathbb{I} = \begin{pmatrix}
    0 & 0 & 1 & 0 \\
    0 & 0 & 0 & 1 \\
    1 & 0 & 0 & 0 \\
    0 & 1 & 0 & 0
  \end{pmatrix}.
\end{equation}

\subsection{非 Abelian 的例子:\texorpdfstring{$S_3$}{𝑆₃} 群}

$S_3$ 群的乘法表为:
\begin{center}
  \begin{tabular}{c|cccccc}
    & $g_0$ & $g_1$ & $g_2$ & $g_3$ & $g_4$ & $g_5$ \\
    \hline
    $g_0$ & 0 & 1 & 2 & 3 & 4 & 5 \\
    $g_1$ & 1 & 0 & 3 & 2 & 5 & 4 \\
    $g_2$ & 2 & 4 & 0 & 5 & 1 & 3 \\
    $g_3$ & 3 & 5 & 1 & 4 & 0 & 2 \\
    $g_4$ & 4 & 2 & 5 & 0 & 3 & 1 \\
    $g_5$ & 5 & 3 & 4 & 1 & 2 & 0 \\
  \end{tabular}
\end{center}
我们取
\begin{equation}
  G(i,j) = g_i g_j,
\end{equation}
则可知 $G(1,2)=g_1 g_2=g_3\eqcolon3$,$G(2,1)=g_2 g_1=g_4\eqcolon4$,显然关于 $i$、$j$ 是不对称的。

\section{2+1 维中的算符推移}

\subsection{例子:\texorpdfstring{$\mathbb{Z}_n$}{ℤₙ} 模型}

\subsection{例子:Fibonacci 模型}

\section{本章小结}
