\ExplSyntaxOn
\__fdu_chapter:n { 引 \quad 言 }
\ExplSyntaxOff

拓扑序\cite{wen1990topological,wen2013topological,wen2017colloquium,zeng2019introduction}的提出源自于上世纪 80 年代分数量子 Hall 效应\cite{tsui1982two,laughlin1983anomalous}、高温超导、手征自旋液体\cite{kalmeyer1987equivalence,wen1989chiral}等新现象的发现。传统上,对相以及相变的描述通常基于 Landau 对称性破缺理论:不同的相可由不同的对称群来刻画,对称性的自发破缺(即对称群的改变)以及相应序参量的变化则提示了相变的出现\cite{landau1980statistical,pathria2011statistical}。然而,随着这些新奇物态的发现,人们逐渐意识到 Landau 理论并不足以全面描述所有的量子物态。例如,不同的分数量子 Hall 态具有相同的对称性,无法通过对称性破缺来区分,也不能用局域序参量来描述\cite{stone1992quantum}。以上结果都表明,存在一种全新类型的序,后来称为拓扑序\footnote{得名于手征自旋态的低能有效理论——拓扑量子场论(TQFT)\cite{witten1989quantum}。}。拓扑序可以通过新的量子数(即拓扑不变量)来描述,例如基态简并度\cite{wen1989vacuum,wen1990ground}、非 Abel 几何相位\cite{wen1990topological}等。这些拓扑不变量不易受到局域微扰的影响,因而有望成为容错量子计算的物理基础\cite{kitaev2003fault,freedman2003topological,nayak2008nonabelian}。

正如 Newton 力学需要依靠微积分来描述,广义相对论需要利用 Riemann 几何,Landau 对称性破缺理论建立在群论的基础上,拓扑序也迫切需要新的数学语言:范畴论。在物理中,范畴论最早被应用于共形场论(CFT)\cite{segal1988definition,moore1989classical}与拓扑量子场论(TQFT)\cite{atiyah1988topological,turaev1992state}的研究。而随后,人们很快意识到具有特定结构的范畴,即张量融合范畴,可以成为描述拓扑序的自然框架\cite{levin2005string,kitaev2006anyons}。2+1 维拓扑序中的准粒子激发称为任意子(anyon),它们具有分数自旋的统计性质,在数学上对应于张量范畴中的简单对象。任意子之间可以融合(fuse)、编织(braid),它们同样可用张量范畴的语言来表述\cite{kong2014anyon,kong2014braided,kong2015boundary,lou2021dummy}。

% TODO: anyon condensation

格点模型是研究拓扑序的一种重要手段。二维拓扑序中常见的严格可解格点模型主要有两类:一类是基于有限群的 quantum double 模型\cite{kitaev2003fault,kitaev2006anyons},它是 Dijkgraaf--Witten TQFT\cite{dijkgraaf1990topological} 的格点近似;另一类是基于张量融合范畴的弦网模型\cite{levin2005string},它是 Turaev--Viro TQFT\cite{turaev1992state,kirillov2011string} 的格点近似。通过群 $G$ 到其表示范畴 $\mathbf{Rep}_G$ 的映射,弦网模型也可以视为 quantum double 模型的推广\cite{buerschaper2009mapping,buerschaper2013electric}。
