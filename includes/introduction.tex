\ExplSyntaxOn
\__fdu_chapter:n { 引 \quad 言 }
\ExplSyntaxOff

拓扑序 (topological order)\cite{wen1990topological,wen2013topological,wen2017colloquium,zeng2019introduction}的提出源自于上世纪 80 年代分数量子 Hall 效应\cite{tsui1982two,laughlin1983anomalous}、高温超导、手征自旋液体\cite{kalmeyer1987equivalence,wen1989chiral}等新现象的发现。传统上,对相以及相变的描述通常基于 Landau 对称性破缺理论:不同的相可由不同的对称群来刻画,对称性的自发破缺(即对称群的改变)以及相应序参量的变化则提示了相变的出现\cite{landau1980statistical,pathria2011statistical}。然而,随着这些新奇物态的发现,人们逐渐意识到 Landau 理论并不足以全面描述所有的量子物态。例如,不同的分数量子 Hall 态具有相同的对称性,无法通过对称性破缺来区分,也不能用局域序参量来描述\cite{stone1992quantum}。以上结果都表明,存在一种全新类型的序,后来称为拓扑序\footnote{得名于手征自旋态的低能有效理论——拓扑量子场论 (topological quantum field theory, TQFT)\cite{witten1989quantum}。}。拓扑序可以通过新的量子数(即拓扑不变量)来描述,例如基态简并度 (ground state degeneracy)\cite{wen1989vacuum,wen1990ground}、非 Abel 几何相位 (non-Abelian geometric phase)\cite{wen1990topological}等。这些拓扑不变量不易受到局域微扰的影响,因而有望成为容错量子计算 (fault-tolerant quantum computation) 的物理基础\cite{kitaev2003fault,freedman2003topological,nayak2008nonabelian}。

正如 Newton 力学需要依靠微积分来描述,广义相对论需要利用 Riemann 几何,Landau 对称性破缺理论建立在群论的基础上,拓扑序也迫切需要新的数学语言:范畴论。在物理中,范畴论最早被应用于共形场论\cite{segal1988definition,moore1989classical}与拓扑量子场论\cite{atiyah1988topological,turaev1992state}的研究。而随后,人们很快意识到具有特定结构的范畴,即张量融合范畴 (fusion category),可以成为描述拓扑序的自然框架\cite{levin2005string,kitaev2006anyons}。2+1 维拓扑序中的准粒子激发称为任意子 (anyon),它们具有分数自旋的统计性质,在数学上对应于张量范畴中的简单对象。任意子之间可以融合 (fuse)、编织 (braid),它们同样可用张量范畴的语言来表述,并且可用融合系数、$F$ 符号以及 $S$、$T$ 矩阵等来定量刻画\cite{bakalov2001lectures,kitaev2006anyons,bruillard2016rank,aasen2020topological}。张量范畴还可以通过任意子凝聚 (anyon condensation) 的机制得到拓扑序的体—边对应关系 (bulk-boundary relation)\cite{kong2014anyon,kong2014braided,kong2015boundary,kong2017boundary,lou2021dummy}。

% TODO: anyon condensation

格点模型是研究拓扑序的一种重要手段。二维拓扑序中常见的严格可解格点模型主要有两类:一类是基于有限群的 quantum double 模型\cite{kitaev2003fault,kitaev2006anyons},它是 Dijkgraaf--Witten TQFT\cite{dijkgraaf1990topological}的格点近似;另一类是基于张量融合范畴的弦网模型 (string-net model)\cite{levin2005string},它是 Turaev--Viro TQFT\cite{turaev1992state,kirillov2011string}的格点近似。通过群 $G$ 到其表示范畴 $\mathbf{Rep}_G$ 的映射,弦网模型也可以视为 quantum double 模型的推广\cite{buerschaper2009mapping,buerschaper2013electric}。

理解量子多体系统始终是凝聚态物理的核心研究课题。除了理论和实验方法外,数值模拟为人们提供了另一种思考量子多体系统的途径。然而,传统的数值算法如精确对角化、平均场理论、Monte Carlo 方法等,均存在一些限制和不足。例如,精确对角化难以处理规模较大的系统,平均场理论无法精确描述强关联系统,而量子 Monte Carlo 方法则会遇到所谓符号问题。其本质在于,这些传统方案很难对关联(或纠缠)提供准确的刻画。而张量网络 (tensor network)\cite{orus2014practical,bridgeman2017hand,biamonte2017tensor,orus2019tensor,ran2020tensor,evenbly2022practical}算法从一开始就能有效地描述量子态的关联性质,因而成为了一种强有力并且发展迅速的数值模拟手段。

张量网络最早源自于上世纪 90 年代提出的密度矩阵重整化群 (density matrix renormalization group, DMRG)\cite{white1992density,white1993density,schollwock2005density}算法。DMRG 的思路是在重整化过程中保留相关自由度,从而获得一维系统的基态。随后人们很快意识到保留下的自由度实际上是波函数的纠缠自由度,DMRG 也被理解为是矩阵乘积态 (matrix product state, MPS) 上的变分优化算法\cite{mcculloch2007density,perez2007matrix,verstraete2008matrix,schollwock2011density}。

近年来,人们提出了多种张量网络结构和算法,包括用于一维系统的多尺度纠缠重整化方法 (multiscale entanglement renormalization ansatz, MERA)\cite{vidal2007entanglement,evenbly2009algorithms,evenbly2014algorithms},用于二维系统的投影纠缠对态 (projected entangled pair state, PEPS)\cite{verstraete2004renormalization},用于表示算符的矩阵乘积算符 (matrix product operator, MPO)\cite{pirvu2010matrix}和投影纠缠对算符 (projected entangled pair operator, PEPO)\cite{czarnik2015variational},以及用于波函数演化模拟的时间演化块消减 (time-evolving block decimation, TEBD)\cite{vidal2003efficient,vidal2004efficient,vidal2007classical,orus2008infinite}算法。此外还有一些用于近似表示二维张量网络的算法,如角转移矩阵 (corner transfer matrix, CTM)\cite{nishino1996corner,orus2012exploring}、张量重整化群 (tensor renormalization group, TRG)\cite{levin2007tensor}、高阶张量重整化群 (higher order TRG, HOTRG)\cite{xie2012coarse}、张量纠缠过滤重整化 (tensor entanglement-filtering renormalization, TEFR)\cite{gu2009tensor1}、张量网络重整化 (tensor network renormalization, TNR)\cite{evenbly2015tensor1,evenbly2017algorithms} 等。其中 TRG 和 HOTRG 等还可以比较容易地推广到更高维度。

不同的张量网络反映了不同的几何特征。例如,MPS 和 PEPS 分别对应于维度 $D=1$ 和 $D>1$ 的物理几何,而 $D$ 维系统中的 MERA 则对应于 $D+1$ 维的全息几何\cite{evenbly2011tensor}。进一步,人们意识到 MERA 中计算得到的纠缠熵和 AdS/CFT\cite{maldacena1999large}中的全息纠缠熵 (holographic entanglement entropy)\cite{ryu2006holographic}是一致的\cite{swingle2012entanglement,swingle2012constructing},这也就暗示着张量网络是实现量子引力的一种可能方案。事实上,利用 $p$ 进数 ($p$-adic number) 中的张量网络,人们已经能够复现 $p$ 进数 AdS/CFT 中的相关结果\cite{bhattacharyya2018tensor,hung2019padic},并且能够给出引力动力学以及 Einstein 方程的构造\cite{chen2021emergent}。
