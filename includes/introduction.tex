\ExplSyntaxOn
\__fdu_chapter:n { 引 \quad 言 }
\ExplSyntaxOff

\begingroup

\renewcommand{\thesection}{\arabic{section}}

\section{拓扑序}

拓扑序 (topological order)\cite{wen1990topological,wen2013topological,wen2017colloquium,zeng2019introduction}的提出源自于上世纪 80 年代分数量子 Hall 效应\cite{tsui1982two,laughlin1983anomalous}、高温超导、手征自旋液体\cite{kalmeyer1987equivalence,wen1989chiral}等新现象的发现。传统上,对相以及相变的描述通常基于 Landau 对称性破缺理论:不同的相可由不同的对称群来刻画,对称性的自发破缺(即对称群的改变)以及相应序参量的变化则提示了相变的出现\cite{landau1980statistical,pathria2011statistical}。然而,随着这些新奇物态的发现,人们逐渐意识到 Landau 理论并不足以全面描述所有的量子物态。例如,不同的分数量子 Hall 态具有相同的对称性,无法通过对称性破缺来区分,也不能用局域序参量来描述\cite{stone1992quantum}。以上结果都表明,存在一种全新类型的序,后来称为拓扑序。拓扑序可以通过新的量子数(即拓扑不变量)来描述,例如基态简并度 (ground state degeneracy)\cite{wen1989vacuum,wen1990ground}、非 Abel 几何相位 (non-Abelian geometric phase)\cite{wen1990topological}等。这些拓扑不变量不易受到局域微扰的影响,因而有望使得拓扑序成为容错量子计算 (fault-tolerant quantum computation) 的物理基础\cite{kitaev2003fault,freedman2003topological,nayak2008nonabelian}。

正如 Newton 力学需要依靠微积分来描述,广义相对论需要利用 Riemann 几何,Landau 对称性破缺理论建立在群论的基础上,拓扑序也迫切需要新的数学语言:范畴论。在物理中,范畴论最早被应用于共形场论\cite{segal1988definition,moore1989classical}与拓扑量子场论\cite{atiyah1988topological,turaev1992state}的研究。随后,人们很快发现具有特定结构的范畴,即张量范畴 (tensor category),可以成为描述拓扑序的自然框架\cite{levin2005string,kitaev2006anyons}。2+1 维拓扑序中的准粒子激发称为任意子 (anyon),它们具有分数自旋的统计性质,在数学上对应于张量范畴中的简单对象。任意子之间可以融合 (fuse)、编织 (braid),这些性质同样可以用张量范畴的语言来表述,并且可用融合系数、$F$ 符号以及 $S$、$T$ 矩阵等来定量刻画\cite{bakalov2001lectures,kitaev2006anyons,bruillard2016rank,aasen2020topological}。张量范畴还可以通过任意子凝聚 (anyon condensation) 的机制得到拓扑序的体—边对应关系 (bulk-boundary relation)\cite{kong2014anyon,kong2014braided,kong2015boundary,kong2017boundary,lou2021dummy}。

格点模型是研究拓扑序的一种重要手段。二维拓扑序中常见的严格可解格点模型主要有两类:一类是基于有限群的 quantum double 模型\cite{kitaev2003fault,kitaev2006anyons},它是 Dijkgraaf--Witten TQFT\cite{dijkgraaf1990topological}的格点近似;另一类是基于张量融合范畴的弦网模型 (string-net model)\cite{levin2005string},它是 Turaev--Viro TQFT\cite{turaev1992state,kirillov2011string}的格点近似。通过群 $G$ 到其表示范畴 $\mathbf{Rep}_G$ 的映射,弦网模型也可以视为 quantum double 模型的推广\cite{buerschaper2009mapping,buerschaper2013electric}。更进一步,弦网模型还被认为可以解释光子与电子的起源\cite{levin2005colloquium}。

\section{张量网络}

理解量子多体系统始终是凝聚态物理的核心研究课题。除了理论和实验方法外,数值模拟为人们提供了另一种思考量子多体系统的途径。然而,传统的数值算法如精确对角化、平均场理论、Monte Carlo 方法等,均存在一些限制和不足。例如,精确对角化难以处理规模较大的系统,平均场理论无法精确描述强关联系统,而量子 Monte Carlo 方法则会遇到所谓符号问题。其本质在于,这些传统方案很难对关联(或纠缠)提供准确的刻画。而张量网络 (tensor network)\cite{orus2014practical,bridgeman2017hand,biamonte2017tensor,orus2019tensor,ran2020tensor,evenbly2022practical}算法从一开始就能有效地描述量子态的关联性质,因而成为了一种强有力并且发展迅速的数值模拟手段。

张量网络最早源自于上世纪 90 年代提出的密度矩阵重整化群 (density matrix renormalization group, DMRG)\cite{white1992density,white1993density,schollwock2005density}算法。DMRG 的思路是在重整化过程中保留相关自由度,从而获得一维系统的基态。随后人们很快意识到保留下的自由度实际上是波函数的纠缠自由度,DMRG 也被理解为是矩阵乘积态 (matrix product state, MPS) 上的变分优化算法\cite{mcculloch2007density,perez2007matrix,verstraete2008matrix,schollwock2011density}。

近年来,人们提出了多种张量网络结构和算法,包括用于一维系统的多尺度纠缠重整化方法 (multiscale entanglement renormalization ansatz, MERA)\cite{vidal2007entanglement,evenbly2009algorithms,evenbly2014algorithms},用于二维系统的投影纠缠对态 (projected entangled pair state, PEPS)\cite{verstraete2004renormalization},用于表示算符的矩阵乘积算符 (matrix product operator, MPO)\cite{pirvu2010matrix}和投影纠缠对算符 (projected entangled pair operator, PEPO)\cite{czarnik2015variational},以及用于波函数演化模拟的时间演化块消减 (time-evolving block decimation, TEBD)\cite{vidal2003efficient,vidal2004efficient,vidal2007classical,orus2008infinite}算法。此外还有一些用于近似表示二维张量网络的算法,如角转移矩阵 (corner transfer matrix, CTM)\cite{nishino1996corner,orus2012exploring}、张量重整化群 (tensor renormalization group, TRG)\cite{levin2007tensor}、高阶张量重整化群 (higher order TRG, HOTRG)\cite{xie2012coarse}、张量纠缠过滤重整化 (tensor entanglement-filtering renormalization, TEFR)\cite{gu2009tensor1}、张量网络重整化 (tensor network renormalization, TNR)\cite{evenbly2015tensor1,evenbly2017algorithms}\allowbreak 等。其中 TRG 和 HOTRG 等还可以比较容易地推广到更高维度。

不同的张量网络反映了不同的几何特征。例如,MPS 和 PEPS 分别对应于维度 $D=1$ 和 $D>1$ 的物理几何,而 $D$ 维系统中的 MERA 则对应于 $D+1$ 维的全息几何\cite{evenbly2011tensor}。进一步,人们意识到 MERA 中计算得到的纠缠熵和 AdS/CFT\cite{maldacena1999large}中的全息纠缠熵 (holographic entanglement entropy)\cite{ryu2006holographic}是一致的\cite{swingle2012entanglement,swingle2012constructing},这也就暗示着张量网络是实现量子引力的一种可能方案。事实上,利用 $p$-adic 张量网络,人们已经能够复现 $p$-adic AdS/CFT 中的相关结果\cite{bhattacharyya2018tensor,hung2019padic},并能给出引力动力学以及 Einstein 方程的构造\cite{chen2021emergent}。

利用张量网络方法来描述拓扑序也很快成为了一种自然的想法,例如 toric code 模型(以 $\mathbb{Z}_2$ 群作为输入数据的 quantum double 模型)和弦网模型的基态波函数都可以表示为 PEPS 张量网络\cite{verstraete2006criticality,gu2009tensor2,buerschaper2009explicit,luo2017structure}。人们进一步发现,这种表示的基础在于 PEPS 面内的 MPO 满足推拉条件 (pulling-through condition),而这一条件与拓扑序本身的对称性是等价的\cite{bultinck2017anyons,sahinoglu2021characterizing}。此外,不同的拓扑分区 (topological sector) 也可以通过 MPO 代数的中心幂等元 (central idempotent) 来确定\cite{bultinck2017anyons,vanhove2018mapping,aasen2020topological}。

另一方面,人们发现将弦网模型的基态波函数与某些特定的直积态做内积,可以直接得到相应临界格点模型(也对应于一个 CFT)的配分函数,这一构造称为“奇异关联子” (strange correlator)\cite{you2014wave,vanhove2018mapping,lootens2019cardy,aasen2020topological,vanhove2022topological}。弦网模型的基态给出了重整化的不动点波函数\cite{konig2009exact,konig2010quantum},因此基于配分函数的 TRG、TNR 等算法就可以改写为对波函数的操作,而具体的重整化算符可以通过 $F$ 符号来构建。这种重整化方案实际上把体 (bulk) 和边界 (boundary) 联系在了一起,因而有望用来构建一套全息张量网络 (holographic tensor network)。

\section{本文结构}

在上面这些研究的基础上,本文将进一步探讨基于拓扑序的张量网络中的一些问题。首先,我们将奇异关联子的概念推广到任意维度,并据此给出全息张量网络的构造。在这一框架下,我们常需要对不同层中的算符进行操作(例如计算体—边传播子),而一类重要的算符就是 CFT 中的 Virasoro 与 Kac--Moody 生成元,因此给出它们在张量网络中的构建方案也是有必要的。最后,我们还希望利用体算符来重建边界算符,即实现所谓“算符推移”。

本文的具体结构如下:

第 \ref{chap:topological-order} 章将简要回顾范畴论中的主要概念,尤其是介绍研究拓扑序所需要的张量范畴与融合范畴。我们将给出一些具体例子,而论文后续也将继续使用其中的数据和结果。我们还将介绍二维拓扑序的两种格点模型,即 toric code 模型和弦网模型,并给出它们的 Hamilton 量以及相应的拓扑性质。

第 \ref{chap:tensor-network} 章主要介绍张量网络方法,并利用图形方式描述一些常用的张量运算。接着我们将分别介绍一维和二维系统中的两类算法,前者以矩阵乘积态 (MPS) 为代表,后者则以张量重整化群 (TRG) 为代表。

第 \ref{chap:strange-correlator} 章将利用奇异关联子建立起从拓扑序到全息张量网络的桥梁。我们首先将给出弦网模型基态波函数的张量网络表示,然后用奇异关联子获得相应临界格点模型的配分函数,还将介绍其中包含的 MPO 对称性。基于这一方案,我们将尝试构建不同维度下的全息张量网络,并在具体的模型上给出体—边传播子的计算结果。本章的 \ref{sec:holographic-tensor-network} 和 \ref{sec:ads-cft-tensor-network} 节基于已发表的文章 (arXiv preprint, 2022)。

第 \ref{chap:virasoro} 章将介绍在二维格点模型配分函数对应的张量网络表示中构建 Virasoro 与 Kac--Moody 代数的方案。我们首先分别以 Ising 模型和 dimer 模型为例,展示了构造与验证 Virasoro 和 Kac--Moody 算符的主要步骤。然后,我们将其推广到更一般的情况,例如 Fibonacci 弦网模型。为了减少有限尺寸效应的影响,我们还需要引入拓扑投影算符。除 \ref{sec:cft-review} 节外,本章内容基于已发表的文章 (Physical Review B, 2022) 和 (Physical Review B, 2023)。

第 \ref{chap:operator-pushing} 章主要探讨在全息张量网络中实现算符推移的方案。我们将利用体算符来重建边界算符,并且得到边界算符为广义自由场的条件。对 1+1 维和 2+1 维中的几类不同模型,我们将给出具体的计算结果。

第 \ref{chap:conclusion} 章是全文的总结,并对进一步的研究可能做出了展望。

\endgroup
