\chapter{总结与展望}
\label{chap:conclusion}

在本文中,我们首先回顾了拓扑序以及描述它的数学框架——张量融合范畴,并介绍了张量网络这一能够有效处理量子多体问题的数值方法。接下来,我们给出了弦网模型基态波函数的张量网络表示,并通过奇异关联子将其与对应临界格点模型的配分函数联系起来。在这种基于拓扑序的张量网络中,本文进一步探讨了一些值得注意的问题,并得到了下面这些结论:

\begin{enumerate}
  \item 在奇异关联子的构造中,边界条件 $\bra{\Omega}$ 可以通过求解重整化算符的本征值问题来获得。这些重整化算符同时实现了一种全息张量网络:边界对应于共形场论,而体则对应了拓扑序(或拓扑量子场论)。我们利用这套方案数值计算了 2+1 维 $\mathcal{A}_{k+1}$ 模型中的耦合常数和 3+1 维 Ising 模型中的临界点,并发现它们与理论结果是基本一致的。
  \item 利用这种全息张量网络,我们还在二维 Ising 模型中计算了体—边传播子,其结果与 AdS/CFT 字典所给出的结论基本吻合。
  \item 在一般性的张量网络中,Virasoro 与 Kac--Moody 算符可以通过一种不依赖于 Hamilton 量的通用方式来构造:首先将转移矩阵表示为一个圆柱形的张量网络,对其进行精确对角化可以得到能动量张量(或流算符),将其变形后插回圆柱并添加对应的系数,即可给出相应的 Virasoro 与 Kac--Moody 生成元。Ising 和 dimer 模型中的数值结果表明,即使只使用很小的圆柱,这套方案也能达到较好的精度。
  \item 对于有限尺寸效应较大的系统(例如 Fibonacci 等拓扑序模型),拓扑投影算符(中心幂等元)可以帮助我们有效地计算并确定能动量张量,进而得到相应的 Virasoro 算符。
  \item 全息张量网络中的算符推移描述了穿越重整化算符时广义自由场的演化,这也相当于给出了利用体算符来重建边界算符的方案。算符推移的具体形式可以通过一组线性方程来描述,求解这组方程可以得到边界算符为广义自由场的条件。对于 1+1 维中的 $Z_n$ 群和 $S_3$ 群,我们还证明了其中广义自由场的数目等于群的阶数。
\end{enumerate}

本文关于全息张量网络的研究只是一个初步的框架,仍有许多方面可以改进和探索。理论方面,我们注意到奇异关联子、MPO 对称性等与范畴对称性\cite{ji2020categorical,kong2020algebraic,wu2021categorical,lootens2023dualities}\allowbreak 有关,而后者描述了拓扑缺陷的代数结构,这种联系值得进一步探讨。数值方面,本文中全息张量网络的构造比较粗糙,例如连接维数只取到了 1。通过优化重整化算符的结构以及相应的缩并算法,我们有望进一步提高计算规模和精度。此外,本文对于全息对偶的探索还非常基础,仅研究了体—边传播子和算符推移。我们希望能够在更多的模型中计算 AdS/CFT 字典的更多内涵,例如关联函数、纠缠楔重构等。对于算符推移,本文主要讨论了几个具体的模型,但对一般的群和张量范畴并没有给出一般性的结论,这也有待后续的研究。

关于 Virasoro 与 Kac--Moody 算符在张量网络中的构造,我们注意到,即使是通过较小的圆柱构造出的 Virasoro 生成元,在更大的圆柱中实际上也是有效的(即仍然可以近似作为升降算符),这也意味着其中可能存在有新的代数结构。另一方面,这套方案也可以被推广并用于研究其他的扩展对称性,例如 Potts 模型中的 $\mathcal{W}$ 代数\cite{fateev1987conformal}。对于更大的系统,我们也可以尝试利用 TNR 等重整化算法进行粗粒近似,以此进一步减少有限尺寸效应的影响。
