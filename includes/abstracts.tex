\begin{abstract}

拓扑序是一种超越 Landau 理论的新奇量子物态,而张量范畴则是描述拓扑序理论的自然框架。以张量范畴中的融合规则、量子维数以及 $F$ 符号等为基础,可以构造出拓扑序的张量网络表示。在这种表示中,矩阵乘积算符 (MPO) 可以揭示拓扑序的对称性,而奇异关联子(基态波函数与特定直积态的内积)则可以给出相应临界格点模型的配分函数。

本文将奇异关联子的概念推广到任意维度,并利用重整化算符给出全息张量网络的构造。我们阐明了配分函数的构建(即直积态的选取)可以转化为重整化算符的本征值问题。通过在不同模型中对临界点的位置进行计算,我们验证了这套重整化方案的可行性。进一步,我们还在二维 Ising 模型上获得了体—边传播子的初步计算结果,这表明本方案也有望给出 AdS/CFT 的张量网络实现。

在全息张量网络中,我们常需要对不同层中的算符进行操作。因而本文关注的另一个问题即为张量网络中 Virasoro 和 Kac--Moody 算符的构造。我们以 Ising 模型和 dimer 模型为例,给出了构造与验证 Virasoro 和 Kac--Moody 算符的主要步骤,并说明即使是在尺寸较小的系统中,也可以得到比较好的精度。而对于有限尺寸效应较大的系统(如 Fibonacci 模型),我们还引入了拓扑投影算符,从而准确地识别出能动量张量,并构造对应的 Virasoro 算符。

最后,本文还研究了在全息张量网络中实现算符推移的方案。我们利用体算符来重建边界算符,并且得到了边界算符为广义自由场的条件。在 Abelian Dijkgraaf--Witten 模型(如 1+1 维的 $Z_n$ 和 $S_3$ 群以及 2+1 维的 $Z_n$ 群)中,我们发现广义自由场的数目与群的阶数一致;而在非 Abelian 的融合范畴(如 Fibonacci 模型)中,计算结果则表明其中并不存在非平凡的广义自由场。

\end{abstract}

\begin{abstract*}

Topological orders are novel quantum states of matter that go beyond the traditional Landau's theory, which can be naturally described by tensor categories. By utilizing the input data from tensor categories, including fusion rules, quantum dimensions, and $F$ symbols, one can explicitly construct the tensor network representation of topological orders. In such representation, the underlying symmetries of the topological order can be characterized by matrix product operators (MPOs). Additionally, the partition function of the corresponding critical lattice model can be obtained by the strange correlator, which involves the inner product of the ground state wave-function and some specific product states.

In this thesis, we generalize the concept of strange correlators to arbitrary dimensions and build holographic tensor networks using the renormalization group (RG) operators. We demonstrate that the eigenstates of the RG operator will give rise to the construction of the partition function (i.e.\ the choice of product states). This RG scheme is verified via the calculation of critical points in explicit examples at different dimensions. Furthermore, we compute the bulk-boundary propagator in 2d Ising model, establishing a preliminary connection to the AdS/CFT correspondence.

One important aspect of holographic tensor networks is the manipulation of operators in different layers. Hence, another focus of this thesis is the implementation of Virasoro and Kac--Moody operators in generic tensor networks. Taking the Ising model and dimer model as examples, we provide the main steps for constructing and verifying Virasoro and Kac-Moody operators, which work to high accuracy even with a relatively small system size. For systems suffering from far more finite size effects (e.g.\ Fibonacci model), we introduce the topological projectors to identify the energy-momentum tensors and build the corresponding Virasoro generators.

Finally, we investigate the operator pushing in holographic tensor networks. We reconstruct boundary operators using bulk operators and obtain the conditions for boundary operators to be generalized free fields. In Abelian Dijkgraaf--Witten models (e.g.\ $Z_n$ and $S_3$ group in 1+1d as well as $Z_n$ group in 2+1d), we find that the number of generalized free fields scale with the rank of the group; while in non-Abelian fusion categories (e.g.\ Fibonacci model), we show that the RG tensor network admits no generalized free field.

\end{abstract*}
