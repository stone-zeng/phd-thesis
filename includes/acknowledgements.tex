\begin{acknowledgements}

博士的几年注定是一段孤独而漫长的旅程。纵使有老师、同学、朋友、亲人的陪伴,学术高峰和人生低谷也终归需要独自挑战和面对。然而,我仍然要感谢所有伴我走过这段旅程的人,因为“无穷的远方,无数的人们,都和我有关”。

感谢我的导师孔令欣教授。尽管由于疫情等种种原因,我们之间的讨论大都是远程进行,但她对物理的热情和对学术的严谨,却能穿透屏幕深深地感染着我。孔老师是良师,也是益友。在学术上,她凭借独特的眼光带领着我探索各种方向;而在生活中,她以身作则教会了我为人处事的正确态度。虽然我以后可能不再从事理论物理的研究,但孔老师的教诲将始终成为激励我前行的动力。

感谢万义顿、李晓鹏、戚扬、周洋、凌意、Bartek Czech 等各位评阅老师和答辩委员,本文的最终完成离不开他们的指导。

感谢闫晓辉、周雯婷、高太梅、唐诗蕊、肖文、张奕林、李炜等各位老师和辅导员在我博士期间的关怀与帮助,也感谢校工朋友们的工作与付出。

感谢王若水和沈策与我的合作,他们在科研上的宝贵经验让我受益匪浅。感谢杨智、陈超逸、娄家奇、程龙、Gabriel Wong、陈霖、季恺昕、刘希融、劳炳新、涂静欣、吴奇峰、杨宸等课题组的其他成员,和他们的交流也让我获得了很多支持与启发。

感谢王宏宇、李英诚、王思源、陈艳艳、于鑫阳、郝云超、林键、罗雨晨、陈东、夏威、朱家纬、张雯芊、邹杰、叶梦、常远和闵玥洋等办公室里的各位同学,以及邱型泽、南珏、王海、吴亚东、李子亮等各位博后。尽管方向不尽相同,但跨领域的讨论往往能激起更多灵感的火花。科研之外的日常生活更是离不开大家,朝夕相处、互相扶持,愿诸位都能有更美好的明天。

感谢室友李子晗对我阴间作息的包容,也感谢他与我一起熬过封校期间的艰苦时光。感谢郭家祺和费昳帆,他们对论文内容与格式提出了诸多修改建议,也为模板的开发做出了贡献。感谢 $\ket*{\text{Physics}}\otimes\ket*{\text{见证}}$ 群里的同学们,能与他们探讨物理、吐槽生活、针砭时弊,是我的幸运。

我还要特别感谢 3type(三言)、atelierAnchor(锚坞),以及许许多多热爱文字排印与字体设计的小伙伴们,他们为我推开了一扇通往全新世界的大门,让我的内心燃起了新的火苗。

最后,我要感谢父母和家人,是你们给了我生命,并用爱陪伴我的成长。

\end{acknowledgements}
