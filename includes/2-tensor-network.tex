\chapter{张量网络方法介绍}

\section{基本概念}

\emph{张量网络} (tensor network)\cite{orus2014practical,bridgeman2017hand} 为凝聚态物理、量子信息、机器学习等领域提供了一套统一的描述框架。顾名思义,张量网络就是根据一定规则连接起来张量单元。这里的\emph{张量} (tensor) 可以简单理解为多维数组,即标量(0 维张量)、向量(1 维张量)、矩阵(2 维张量)的推广。

如图~\ref{fig:tensors} 所示,张量网络常利用图形方式来描述,其中圆圈(也可用其他图形)表示张量,延伸出来的腿表示张量的指标。

\begin{figure}[htb]
  \centering
  \includegraphics[width=0.6\textwidth]{images/temp/tensors.pdf}
  \caption[张量单元]{三种张量单元,分别为向量、矩阵和三阶张量。}
  \label{fig:tensors}
\end{figure}

\subsection{基本张量运算}

常用的张量运算包括张量积、缩并、变形等。

\emph{张量积} (tensor product) 其实就是把若干个张量并排放置,并保持指标不变:
\begin{equation}
  (A \otimes B)_{i_1,\dots,i_r,j_1,\dots,j_s} \coloneq A_{i_1,\dots,i_r} B_{j_1,\dots,j_s}.
\end{equation}
图形描述为
\begin{equation}
  [[TODO:]]
\end{equation}

张量的\emph{缩并} (contraction) 是向量内积、矩阵乘法的推广,即对两个张量的某些指标进行求和,得到一个新的张量:
\begin{equation}
  C_{abc} = \sum_k A_{abk} B_{kc} \eqcolon A_{abk} B_{kc}
\end{equation}
这里我们根据 Einstein 求和约定省略了求和号。张量缩并的图形描述为
\begin{equation}
  \mbox{\includegraphics[width=0.75\textwidth]{images/temp/contraction.pdf}}
\end{equation}
即把需要缩并的腿(指标)连接起来。

同一个张量的指标也可以缩并,这样就得到了\emph{迹} (trace) 或\emph{偏迹} (partial trace):
\begin{equation}
    (\tr_{x,y} A)_{i_1,\dots,i_{x-1},i_{x+1},\dots,i_{y-1},u_{y+1},\dots,i_r}
  = A_{i_1,\dots,i_{x-1},k,i_{x+1},\dots,i_{y-1},k,i_{y+1},\dots,i_r}.
\end{equation}
其图形描述与缩并类似:
\begin{equation}
  [[TODO:]]
\end{equation}
利用图形语言很容易验证 $\tr(AB)=\tr(BA)$:
\begin{equation}
  [[TODO:]]
\end{equation}

张量的\emph{变形} (reshape) 相当于指标的重新组合,例如
\begin{equation}
  \mbox{\includegraphics[width=0.5\textwidth]{images/temp/reshape.pdf}}
\end{equation}
在数值计算中,把一般形状的张量变形为矩阵,往往可以利用特化的算法来加速计算。

\subsection{张量分解}

张量的\emph{分解} (decomposition) 可以理解为缩并的逆运算,最常用的是\emph{奇异值分解} (singular value decomposition, SVD),即
\begin{equation}
  M_{ij} = \sum_{k=1}^{\min(m,n)} U_{ik} S_{kk} V_{kj} = \sum_k U_{ik} S_{kk} \bigl( V^\dagger \bigr)_{jk},
\end{equation}
其中 $M$ 为 $m\times n$ 矩阵,$S$ 为对角矩阵(如果行数、列数不等,则相应补零),$U$ 和 $V$ 为幺正矩阵,$V^\dagger$ 表示共轭转置。利用张量的变形,可以很容易地将 SVD 推广为一般形状的张量,用图形描述为
\begin{equation}
  [[TODO:]]
\end{equation}

奇异值分解可以用来得到张量的近似表示。将奇异值($S$ 矩阵的对角元)按大小排列后,只保留前 $r$ 个值,就得到了秩为 $r$ 的近似张量:
\begin{equation}
  M_{ij} \approx \sum_{k=1}^r U_{ik} S_{kk} V_{kj}, \quad r < \min(m,n).
\end{equation}
并且可以证明,这种近似表示是所有秩为 $r$ 的张量中最优的。

对于一个一般的张量,原则上我们总可以把它改写为若干个小张量的缩并形式,并保证自由度数目不变。这种改写并不能降低计算复杂度,但我们可以利用奇异值分解把这些小张量替换为对应的近似表示,这样就可以大幅减少总自由度的数目。

\section{矩阵乘积态(MPS)算法}

\section{重整化算法}

\section{具体实现}

\citet{harris2020array,virtanen2020scipy}
