\chapter{Virasoro 代数的张量网络实现}

\section{二维共形场论回顾}

\emph{共形场论} (conformal field theory, CFT)\cite{belavin1984infinite,ginsparg1988applied,francesco2012conformal} 的诞生源于对相变与临界现象的研究。在(二阶)相变点附近,系统应当具有\emph{标度不变性} (scaling invariance);而在二维情况下,标度不变性与共形不变性是等价的。这就意味着可以用满足\emph{共形对称性} (conformal symmetry) 的量子场论——即共形场论——来处理临界系统。

\subsection{共形对称性与能动量张量}

二维共形场论可以用复平面上的坐标 $z$ 和 $\bar{z}$ 来描述。根据 Cauchy--Riemann 条件,共形变换 $z\to w(z)$ 和 $\bar{z}\to\bar{w}(\bar{z})$ 分别是\emph{全纯} (holomorphic) 和\emph{反全纯} (anti-holomorphic) 的,即它们只是 $z$ 和 $\bar{z}$ 的函数。在这一共形变换下,满足
\begin{equation}
  \phi(z,\bar{z}) \to \phi'(w,\bar{w}) =
  \left( \dv{w}{z} \right)^{-h} \left( \dv{\bar{w}}{\bar{z}} \right)^{-\bar{h}} \phi(z,\bar{z})
  \label{eq:quasi-primary-field}
\end{equation}
变换关系的场 $\phi$ 称为\emph{准初级场} (quasi-primary field),其中
\begin{equation}
  h = \frac12 \bigl( \Delta+s \bigr), \quad \bar{h} = \frac12 \bigl( \Delta-s \bigr)
\end{equation}
称为\emph{共形维数} (conformal dimension),而 $\Delta$ 和 $s$ 分别称为\emph{标度维数} (scaling dimension) 和\emph{自旋} (spin),它们反映了 $\phi$ 在标度和旋转变换下的性质。如果对任意的局部共形变换,式~\eqref{eq:quasi-primary-field} 都成立,则称 $\phi$ 为\emph{初级场} (primary field)。

根据 Noether 定理,(连续)对称性会与某种守恒流相对应。因而我们可以为一个一般的局部坐标变换定义\emph{能动量张量} (energy-momentum tensor),也称\emph{应力张量} (stress tensor)。在共形对称性的条件下,能动量张量 $T^{\mu\nu}$ 可以取为对称且无迹的,即只保留 $T(z)\coloneq T_{zz}(z)$ 和 $\bar{T}(\bar{z})\coloneq T_{\bar{z}\bar{z}}(\bar{z})$,同时它们也是(反)全纯函数\cite{ginsparg1988applied,cardy2010conformal,francesco2012conformal}。$T$ 和 $\bar{T}$ 的共形维数分别为 $(h_T,\bar{h}_T)=(2,0)$ 和 $(h_{\bar{T}},\bar{h}_{\bar{T}})=(0,2)$,即
\begin{equation}
  \Delta_T = \Delta_{\bar{T}} = 2, \quad s_T = 2, \quad s_{\bar{T}} = -2.
\end{equation}

对于共形维数为 $(h,\bar{h})$ 初级场 $\phi$,能动量张量与它的\emph{算子积展开} (operator product expansion, OPE) 具有如下形式:
\begin{equation}
  \begin{aligned}
    T(z) \phi(w,\bar{z}) &\sim
      \frac{h}{(z-w)^2} \phi(w,\bar{z}) + \frac{1}{z-w} \partial_w\phi(w,\bar{z}), \\
    \bar{T}(\bar{z}) \phi(w,\bar{z}) &\sim
      \frac{\bar{h}}{(\bar{z}-\bar{w})^2} \phi(w,\bar{z}) + \frac{1}{\bar{z}-\bar{w}} \partial_{\bar{w}}\phi(w,\bar{z}).
  \end{aligned}
  \label{eq:t-phi-ope}
\end{equation}
而能动量张量与自身的 OPE 则可写为
\begin{equation}
  \begin{aligned}
    T(z) T(w) &\sim
      \frac{c/2}{(z-w)^4} + \frac{2}{(z-w)^2} T(w) + \frac{1}{z-w} \partial_w T(w), \\
    \bar{T}(\bar{z}) \bar{T}(\bar{w}) &\sim
        \frac{\bar{c}/2}{(\bar{z}-\bar{w})^4}
      + \frac{2}{(\bar{z}-\bar{w})^2} \bar{T}(\bar{w})
      + \frac{1}{\bar{z}-\bar{w}} \partial_{\bar{w}}\bar{T}(\bar{w}).
  \end{aligned}
\end{equation}
其中 $(c,\bar{c})$ 称为\emph{中心荷} (central charge)。

\subsection{Virasoro 代数}

二维共形场论可以进行\emph{径向量子化} (radial quantization),即通过
\begin{equation}
  z = \exp\left( \frac{2\pi\xi}{l} \right), \quad \xi = t+\ii x
  \label{eq:radial-quantization}
\end{equation}
将圆柱面映射到平面上,这样时间 $t$、空间 $x$ 的平移变换就相当于复平面上的缩放与旋转变换。由于等时面 $t\to-\infty$ 被映射到了复平面的坐标原点 $z=\bar{z}=0$,因而可有\emph{态—算符对应} (state-operator correspondence):
\begin{equation}
  \ket{\phi} = \lim_{t\to-\infty}\phi(x,t) \ket{0} = \lim_{z,\bar{z}\to 0} \phi(z,\bar{z}) \ket{0},
\end{equation}
这意味着每一个场算符都可以生成一个对应的量子态(波函数)。

把能动量张量进行模展开,可以得到
\begin{equation}
  \begin{aligned}
    T(z)             &= \sum_{n\in\mathbb{Z}} z^{-n-2} L_n, &\quad
    L_n              &= \frac{1}{2\pi\ii} \oint z^{n+1} T(z) \, \dd z; \\
    \bar{T}(\bar{z}) &= \sum_{n\in\mathbb{Z}} \bar{z}^{-n-2} \bar{L}_n, &\quad
    \bar{L}_n        &= \frac{1}{2\pi\ii} \oint \bar{z}^{n+1} \bar{T}(\bar{z}) \, \dd\bar{z}.
  \end{aligned}
  \label{eq:virasoro-operators}
\end{equation}
式中 $L_n$ 和 $\bar{L}_n$ 称为 \emph{Virasoro 算符} (Virasoro operators),它们构成了 \emph{Virasoro 代数} (Virasoro algebra):
\begin{equation}
  \begin{aligned}
    \bigl[ L_n, L_m \bigr]
      &= (n-m) L_{n+m} + \frac{c}{12} n \bigl( n^2-1 \bigr) \delta_{n+m,0}, \\
    \bigl[ \bar{L}_n, \bar{L}_m \bigr]
      &= (n-m) \bar{L}_{n+m} + \frac{\bar{c}}{12} n \bigl( n^2-1 \bigr) \delta_{n+m,0}, \\
    \bigl[ L_n, \bar{L}_m \bigr] &= 0.
  \end{aligned}
  \label{eq:virasoro-algebra}
\end{equation}
在径向量子化后,Hamilton 算符 $H$ 和动量算符 $P$ 可以用 Virasoro 算符表示为
\begin{equation}
  % TODO: why
  H = L_0+\bar{L}_0, \quad P = \ii \bigl( L_0-\bar{L}_0 \bigr).
\end{equation}
另一方面,真空态 $\ket{0}$ 需要在全局共形变换下保持不变,这要求
\begin{equation}
  L_n \ket{0} = \bar{L}_n \ket{0} = 0, \quad n \geqslant -1.
\end{equation}
设初级场 $\phi$ 对应的态为 $\ket*{h,\bar{h}}\coloneq\phi(0,0)\ket{0}$。根据式~\eqref{eq:t-phi-ope},可知
\begin{equation}
  L_0       \ket*{h,\bar{h}} = h       \ket*{h,\bar{h}}, \quad
  \bar{L}_0 \ket*{h,\bar{h}} = \bar{h} \ket*{h,\bar{h}}, \quad
  L_n \ket*{h,\bar{h}} = \bar{L}_n \ket*{h,\bar{h}} = 0 \enspace (n > 0).
\end{equation}
代入式~\eqref{eq:virasoro-algebra} 中的对易关系,有
\begin{equation}
  \bigl[ L_0, L_{-n} \bigr] = n L_{-n}, \quad
  \bigl[ \bar{L}_0, \bar{L}_{-n} \bigr] = n \bar{L}_{-n}.
\end{equation}
可以看出 $L_{-n}\ket{0}$ 和 $\bar{L}_{-n}\ket{0}$ 分别是 $L_0$ 和 $\bar{L}_0$ 本征值为 $n$ 的本征态,因而 $L_{-n}$、$\bar{L}_{-n}$ 即可作为升算符,使得共形维数 $h$、$\bar{h}$ 增加 $n$。产生的这些态称为 $\ket*{h,\bar{h}}$ 的\emph{后代} (descendant),它们也可以通过对 $\phi$ 求导得到。

\subsection{环面配分函数}

将圆柱面的 $t\to\pm\infty$ 等时面“粘”在一起便可得到环面。环面的几何由参数 $\tau=\tau_1+\ii\tau_2$ 表示,它需要在变换
\begin{equation}
  \tau \to \frac{a\tau+b}{c\tau+d}, \quad \begin{pmatrix} a & b \\ c & d \end{pmatrix} \in PSL(2,\mathbb{Z})
\end{equation}
下保持不变,其中 $PSL(2,\mathbb{Z})=SL(2,\mathbb{Z})/\mathbb{Z}_2$ 称为\emph{模群} (modular group)。此时配分函数可以写为
\begin{align}
  Z &= \tr \Bigl[ \exp \bigl( -2\pi\tau_2 H \bigr) \exp \bigl( 2\pi\ii\tau_1 P \bigr) \Bigr] \notag \\
    &= \tr \Bigl[
         \exp \Bigl( -2\pi   \tau_2 \Bigl( L_0 + \bar{L}_0 - \frac{c}{12} \Bigr) \Bigr)
         \exp \Bigl(  2\pi\ii\tau_1 \bigl( L_0 - \bar{L}_0 \bigr) \Bigr)
       \Bigr].
\end{align}
设场 $\phi_\alpha$ 的共形维数为 $(h_\alpha,\bar{h}_\alpha)$。由 Virasoro 代数可知
\begin{align}
  Z &= \sum_\alpha \exp \Bigl[
         - 2\pi   \tau_2 \Bigl( h_0 + \bar{h}_0 - \frac{c}{12} \Bigr)
         + 2\pi\ii\tau_1 \bigl( h_0 - \bar{h}_0 \bigr)
       \Bigr] \notag \\
    &= \sum_\alpha \exp \Bigl[
         - 2\pi   \tau_2 \Bigl(\Delta_\alpha - \frac{c}{12} \Bigr)
         + 2\pi\ii\tau_1 s_\alpha
       \Bigr],
  \label{eq:torus-partition-function}
\end{align}
其中 $\Delta_\alpha$ 和 $s_\alpha$ 分别是场 $\phi_\alpha$ 的标度维数和自旋。

\section{格点近似}

\section{Virasoro 与 Kac--Moody 算符的构造}

\section{能动量张量的确定}
