\documentclass[type=doctor,oneside]{fduthesis}

\usepackage{booktabs}
\usepackage{diagbox}
\usepackage{enumitem}
\usepackage{rotating}
\usepackage{threeparttable}
\usepackage{tikz}
\usepackage{xcolor-material}

% fduthesis
\fdusetup{
  style = {
    font = libertinus,
    footnote-style = libertinus,
    fullwidth-stop = mapping,
    bib-backend = bibtex,
    bib-resource = {main.bib},
  },
  info = {
    title = {基于拓扑序的张量网络中若干问题的研究},
    title* = {Aspects on Tensor Networks for Topological Orders},
    author = {曾祥东},
    supervisor = {孔令欣\quad 教授},
    major = {理论物理},
    degree = academic,
    department = {物理学系},
    student-id = {18110190010},
    % date = {2023 年 1 月 1 日},
    keywords = {拓扑序, 张量网络, 奇异关联子, Virasoro 算符, 算符推移},
    keywords* = {Topological orders, tensor networks, strange correlators, Virasoro generators, operator pushing},
    clc = {O41},
  }
}

% enumitem
\setlist{itemsep=0pt}

% tikz
\usetikzlibrary{
  arrows.meta,
  calc,
  decorations.markings,
}
\tikzset{
  vector box/.style={thick, MaterialTeal, fill=MaterialTeal100},
  tensor box/.style={thick, MaterialBlue, fill=MaterialBlue100},
  tensor leg/.style={semithick, MaterialGrey},
  covered tensor leg/.style={semithick, dashed, MaterialGrey300},
  ->-/.style = {
    decoration = {
      markings,
      mark = at position #1 with {\arrow{Stealth}},
    },
    postaction = {decorate},
  },
}
\newcommand{\Vertex}[3]{
  \begin{tikzpicture}[baseline=0, thick]
    \draw (0,0) -- (90:1)  node [above] {$#1$}
          (0,0) -- (210:1) node [left]  {$#2$}
          (0,0) -- (330:1) node [right] {$#3$};
  \end{tikzpicture}
}
\newcommand{\Tetrahedron}[6]{
  \begin{tikzpicture}[baseline=1ex]
    \draw [covered tensor leg, MaterialGrey] (0,0) -- (3.2,0);
    \draw [tensor leg]
          (0,0) -- (2,2.5) -- (3.2,0) -- (2,-0.8) -- cycle
          (2,2.5) -- (2,-0.8);
    \draw (0.6, 1.7) node {$#1$}
          (3.2, 1.7) node {$#2$}
          (0.6,-0.9) node {$#3$}
          (3  ,-0.9) node {$#4$}
          (1.1, 0.5) node {$#5$}
          (2.4, 0.8) node {$#6$};
  \end{tikzpicture}
}
\newcommand{\Triangle}[6]{
  \begin{tikzpicture}[baseline=-0.5ex]
    \draw [tensor box]
          ( 90:1.5) -- (210:1.5) -- (330:1.5) -- cycle;
    \foreach \x in {90, 210, 330}
      \draw [tensor leg]
          (\x-40:1.7) .. controls (\x-20:0.8) and (\x+20:0.8) .. (\x+40:1.7)
          (\x-60:0.4) -- (\x-60:1.6);
    \draw ( 30:2) node {$#1$}
          (150:2) node {$#2$}
          (270:2) node {$#3$}
          ( 90:2) node {$#4$}
          (210:2) node {$#5$}
          (330:2) node {$#6$};
  \end{tikzpicture}
}
\newcommand{\VirutalHexagon}[1]{
  \begin{tikzpicture}[baseline=-0.2em]
    \draw [thick, dotted]
      (30:1.2) -- (90:1.2) -- (150:1.2) -- (210:1.2) -- (270:1.2) -- (330:1.2) -- cycle;
    #1;
  \end{tikzpicture}
}

% Hacks
\ExplSyntaxOn
\cs_set_protected:Npn \__fdu_load_cjk_font_fandol:
  {
    \__fdu_setCJKmainfont:nn  { HYShuSongErS } { BoldFont = HYZhongSongS }
    \__fdu_setCJKsansfont:n   { HYZhongHeiS  }
    \__fdu_set_cjk_font_kai:n { HYKaiTiS     }
  }
\cs_set:Npn \fdu_allow_url_break:
  {
    \cs_set:Npn \UrlBreaks
      {
        \do \. \do \@ \do \\ \do \/ \do \! \do \_ \do \| \do \; \do \>
        \do \] \do \) \do \, \do \? \do \& \do \' \do  + \do \= \do \#
      }
  }
\ExplSyntaxOff

\AtBeginDocument{
  \setlength{\bibsep}{4pt plus 2pt minus 1pt}
}

% Math commands
\newcommand{\dd}{\mathrm{d}}
\newcommand{\ee}{\mathrm{e}}
\newcommand{\ii}{\mathrm{i}}
\newcommand{\id}{\mathrm{id}}
\newcommand{\1}{\mathbb{1}}
\newcommand{\trans}{{\mathrm{T}}}
\newcommand{\bm}[1]{\symbf{#1}}
\newcommand{\ldual}[1]{{}^*\mspace{-2mu}#1}
\newcommand{\dv}[2]{\frac{\mathrm{d}#1}{\mathrm{d}#2}}
\newcommand{\pdv}[2]{\frac{\partial#1}{\partial#2}}
\NewDocumentCommand{\bra}{som}{%
  \IfBooleanTF{#1}{\langle#3|}{%
    \IfValueTF{#2}{#2\langle#3#2|}{\left\langle#3\right|}}}
\NewDocumentCommand{\ket}{som}{%
  \IfBooleanTF{#1}{|#3\rangle}{%
    \IfValueTF{#2}{#2|#3#2\rangle}{\left|#3\right\rangle}}}
\NewDocumentCommand{\ev}{som}{%
  \IfBooleanTF{#1}{\langle#3\rangle}{%
    \IfValueTF{#2}{#2\langle#3#2\rangle}{\left\langle#3\right\rangle}}}
\DeclareMathOperator{\Hom}{Hom}
\DeclareMathOperator{\End}{End}
\DeclareMathOperator{\rank}{rank}
\DeclareMathOperator{\tr}{tr}
\DeclareMathOperator*{\argmin}{arg\,min}

\newcommand{\tikzinput}[1]{\input{includes/tikz/#1.tex}}

%TC:envir fusionrules [ignore] text
\newenvironment{fusionrules}[1]{%
  \begingroup
    \renewcommand{\arraystretch}{0.8}%
    \begin{tabular}{#1}
      \hline
}{%
      \hline
    \end{tabular}
  \endgroup
}

\begin{document}

\frontmatter

\tableofcontents
\listoffigures
\listoftables

\begin{abstract}
  中文摘要
\end{abstract}

\begin{abstract*}
  English abstract
\end{abstract*}


\mainmatter

\ExplSyntaxOn
\__fdu_chapter:n { 引 \quad 言 }
\ExplSyntaxOff

\begingroup

\renewcommand{\thesection}{\arabic{section}}

\section{拓扑序}

拓扑序 (topological order)\cite{wen1990topological,wen2013topological,wen2017colloquium,wen2019choreographed,zeng2019introduction}的提出源自于上世纪 80 年代分数量子 Hall 效应\cite{tsui1982two,laughlin1983anomalous}、高温超导、手征自旋液体\cite{kalmeyer1987equivalence,wen1989chiral}等新现象的发现。传统上,对相以及相变的描述通常基于 Landau 对称性破缺理论:不同的相可由不同的对称群来刻画,对称性的自发破缺(即对称群的改变)以及相应序参量的变化则提示了相变的出现\cite{landau1980statistical,pathria2011statistical}。然而,随着这些新奇物态的发现,人们逐渐意识到 Landau 理论并不足以全面描述所有的量子物态。例如,不同的分数量子 Hall 态具有相同的对称性,无法通过对称性破缺来区分,也不能用局域序参量来描述\cite{stone1992quantum}。以上结果都表明,存在一种全新类型的序,后来称为拓扑序。拓扑序可以通过新的量子数(即拓扑不变量)来描述,例如基态简并度 (ground state degeneracy)\cite{wen1989vacuum,wen1990ground}、非 Abel 几何相位 (non-Abelian geometric phase)\cite{wen1990topological}等。这些拓扑不变量不易受到局域微扰的影响,因而有望使得拓扑序成为容错量子计算 (fault-tolerant quantum computation) 的物理基础\cite{kitaev2003fault,freedman2003topological,nayak2008nonabelian,kong2023categorical}。

正如 Newton 力学需要依靠微积分来描述,广义相对论需要利用 Riemann 几何,Landau 对称性破缺理论建立在群论的基础上,拓扑序也迫切需要新的数学语言:范畴论。在物理中,范畴论最早被应用于共形场论\cite{segal1988definition,moore1989classical}与拓扑量子场论\cite{atiyah1988topological,turaev1992state}的研究。随后,人们很快发现具有特定结构的范畴,即张量范畴 (tensor category),可以成为描述拓扑序的自然框架\cite{levin2005string,kitaev2006anyons}。2+1 维拓扑序中的准粒子激发称为任意子 (anyon),它们具有分数自旋的统计性质,在数学上对应于张量范畴中的简单对象。任意子之间可以融合 (fuse)、编织 (braid),这些性质同样可以用张量范畴的语言来表述,并且可用融合系数、$F$ 符号以及 $S$、$T$ 矩阵等来定量刻画\cite{bakalov2001lectures,kitaev2006anyons,bruillard2016rank,aasen2020topological}。张量范畴还可以通过任意子凝聚 (anyon condensation) 的机制得到拓扑序的体—边对应关系 (bulk-boundary relation)\cite{kong2014anyon,kong2014braided,kong2015boundary,kong2017boundary,lou2021dummy}。

格点模型是研究拓扑序的一种重要手段。二维拓扑序中常见的严格可解格点模型主要有两类:一类是基于有限群的 quantum double 模型\cite{kitaev2003fault,kitaev2006anyons},它是 Dijkgraaf--Witten TQFT\cite{dijkgraaf1990topological}的格点近似;另一类是基于张量融合范畴的弦网模型 (string-net model)\cite{levin2005string},它是 Turaev--Viro TQFT\cite{turaev1992state,kirillov2011string}的格点近似。通过群 $G$ 到其表示范畴 $\mathbf{Rep}_G$ 的映射,弦网模型也可以视为 quantum double 模型的推广\cite{buerschaper2009mapping,buerschaper2013electric}。更进一步,弦网模型还被认为可以解释光子与电子的起源\cite{levin2005colloquium}。

\section{张量网络}

理解量子多体系统始终是凝聚态物理的核心研究课题。除了理论和实验方法外,数值模拟为人们提供了另一种思考量子多体系统的途径。然而,传统的数值算法如精确对角化、平均场理论、Monte Carlo 方法等,均存在一些限制和不足。例如,精确对角化难以处理规模较大的系统,平均场理论无法精确描述强关联系统,而量子 Monte Carlo 方法则会遇到所谓符号问题。其本质在于,这些传统方案很难对关联(或纠缠)提供准确的刻画。而张量网络 (tensor network)\cite{orus2014practical,bridgeman2017hand,biamonte2017tensor,orus2019tensor,ran2020tensor,evenbly2022practical}算法从一开始就能有效地描述量子态的关联性质,因而成为了一种强有力并且发展迅速的数值模拟手段。

张量网络最早源自于上世纪 90 年代提出的密度矩阵重整化群 (density matrix renormalization group, DMRG)\cite{white1992density,white1993density,schollwock2005density}算法。DMRG 的思路是在重整化过程中保留相关自由度,从而获得一维系统的基态。随后人们很快意识到保留下的自由度实际上是波函数的纠缠自由度,DMRG 也被理解为是矩阵乘积态 (matrix product state, MPS) 上的变分优化算法\cite{mcculloch2007density,perez2007matrix,verstraete2008matrix,schollwock2011density}。

近年来,人们提出了多种张量网络结构和算法,包括用于一维系统的多尺度纠缠重整化方法 (multiscale entanglement renormalization ansatz, MERA)\cite{vidal2007entanglement,evenbly2009algorithms,evenbly2014algorithms},用于二维系统的投影纠缠对态 (projected entangled pair state, PEPS)\cite{verstraete2004renormalization},用于表示算符的矩阵乘积算符 (matrix product operator, MPO)\cite{pirvu2010matrix}和投影纠缠对算符 (projected entangled pair operator, PEPO)\cite{czarnik2015variational},以及用于波函数演化模拟的时间演化块消减 (time-evolving block decimation, TEBD)\cite{vidal2003efficient,vidal2004efficient,vidal2007classical,orus2008infinite}算法。此外还有一些用于近似表示二维张量网络的算法,如角转移矩阵 (corner transfer matrix, CTM)\cite{nishino1996corner,orus2012exploring}、张量重整化群 (tensor renormalization group, TRG)\cite{levin2007tensor}、高阶张量重整化群 (higher order TRG, HOTRG)\cite{xie2012coarse}、张量纠缠过滤重整化 (tensor entanglement-filtering renormalization, TEFR)\cite{gu2009tensor1}、张量网络重整化 (tensor network renormalization, TNR)\cite{evenbly2015tensor1,evenbly2017algorithms}\allowbreak 等。其中 TRG 和 HOTRG 等还可以比较容易地推广到更高维度。

不同的张量网络反映了不同的几何特征。例如,MPS 和 PEPS 分别对应于维度 $D=1$ 和 $D>1$ 的物理几何,而 $D$ 维系统中的 MERA 则对应于 $D+1$ 维的全息几何\cite{evenbly2011tensor}。进一步,人们意识到 MERA 中计算得到的纠缠熵和 AdS/CFT\cite{maldacena1999large}中的全息纠缠熵 (holographic entanglement entropy)\cite{ryu2006holographic}是一致的\cite{swingle2012entanglement,swingle2012constructing},这也就暗示着张量网络是实现量子引力的一种可能方案。事实上,利用 $p$-adic 张量网络,人们已经能够复现 $p$-adic AdS/CFT 中的相关结果\cite{bhattacharyya2018tensor,hung2019padic},并能给出引力动力学以及 Einstein 方程的构造\cite{chen2021emergent}。

利用张量网络方法来描述拓扑序也很快成为了一种自然的想法,例如 toric code 模型(以 $\Z_2$ 群作为输入数据的 quantum double 模型)和弦网模型的基态波函数都可以表示为 PEPS 张量网络\cite{verstraete2006criticality,gu2009tensor2,buerschaper2009explicit,luo2017structure}。人们进一步发现,这种表示的基础在于 PEPS 面内的 MPO 满足推拉条件 (pulling-through condition),而这一条件与拓扑序本身的性质具有密切联系\cite{bultinck2017anyons,sahinoglu2021characterizing}。此外,不同的拓扑分区 (topological sector) 也可以通过 MPO 代数的中心幂等元 (central idempotent) 来确定\cite{bultinck2017anyons,vanhove2018mapping,aasen2020topological}。

另一方面,人们发现将弦网模型的基态波函数与某些特定的直积态做内积,可以直接得到相应临界格点模型(也对应于一个 CFT)的配分函数,这一构造称为“奇异关联子” (strange correlator)\cite{you2014wave,vanhove2018mapping,lootens2019cardy,aasen2020topological,vanhove2022topological}。弦网模型的基态给出了重整化的不动点波函数\cite{konig2009exact,konig2010quantum},因此基于配分函数的 TRG、TNR 等算法就可以改写为对波函数的操作,而具体的重整化算符可以通过 $F$ 符号来构建。这种重整化方案实际上把体 (bulk) 和边界 (boundary) 联系在了一起,因而有望用来构建一套全息张量网络 (holographic tensor network)。

\section{本文结构}

在上面这些研究的基础上,本文将进一步探讨基于拓扑序的张量网络中的一些问题。首先,我们将奇异关联子的概念推广到任意维度,并据此给出全息张量网络的构造。在这一框架下,我们常需要对不同层中的算符进行操作(例如计算体—边传播子),而一类重要的算符就是 CFT 中的 Virasoro 与 Kac--Moody 生成元,因此给出它们在张量网络中的构建方案也是有必要的。最后,我们还希望利用体算符来重建边界算符,即实现所谓“算符推移”。

本文的具体结构如下:

第 \ref{chap:topological-order} 章将简要回顾范畴论中的主要概念,尤其是介绍研究拓扑序所需要的张量范畴与融合范畴。我们将给出一些具体例子,而论文后续也将继续使用其中的数据和结果。我们还将介绍二维拓扑序的两种格点模型,即 toric code 模型和弦网模型,并给出它们的 Hamilton 量以及相应的拓扑性质。

第 \ref{chap:tensor-network} 章主要介绍张量网络方法,并利用图形方式描述一些常用的张量运算。接着我们将分别介绍一维和二维系统中的两类算法,前者以矩阵乘积态 (MPS) 为代表,后者则以张量重整化群 (TRG) 为代表。

第 \ref{chap:strange-correlator} 章将利用奇异关联子建立起从拓扑序到全息张量网络的桥梁。我们首先将给出弦网模型基态波函数的张量网络表示,然后用奇异关联子获得相应临界格点模型的配分函数,还将介绍其中包含的 MPO 对称性。基于这一方案,我们将尝试构建不同维度下的全息张量网络,并在具体的模型上给出体—边传播子的计算结果。本章的 \ref{sec:holographic-tensor-network} 和 \ref{sec:ads-cft-tensor-network} 节基于已发表的论文 (arXiv preprint, 2022)。

第 \ref{chap:virasoro} 章将介绍在二维格点模型配分函数对应的张量网络表示中构建 Virasoro 与 Kac--Moody 代数的方案。我们首先分别以 Ising 模型和 dimer 模型为例,展示了构造与验证 Virasoro 和 Kac--Moody 算符的主要步骤。然后,我们将其推广到更一般的情况,例如 Fibonacci 弦网模型。为了减少有限尺寸效应的影响,我们还需要引入拓扑投影算符。除 \ref{sec:cft-review} 节外,本章内容基于已发表的论文 (Physical Review B, 2022) 和 (Physical Review B, 2023)。

第 \ref{chap:operator-pushing} 章主要探讨在全息张量网络中实现算符推移的方案。我们将利用体算符来重建边界算符,并且得到边界算符为广义自由场的条件。对 1+1 维和 2+1 维中的几类不同模型,我们将给出具体的计算结果。本章内容基于论文(准备中,2023)。

第 \ref{chap:conclusion} 章是全文的总结,并对进一步的研究可能做出了展望。

\endgroup


\chapter{拓扑序与张量范畴}
\label{chap:topological-order}

\emph{拓扑序} (topological order) 是本文探讨的中心对象,它是一类超越了 Landau 理论的新奇物态,并且具有基态简并、非 Abel 几何相位、分数激发等一系列全新的物理图像。描述 2+1 维拓扑序的数学语言是\emph{张量范畴} (tensor category),从它出发可以构建\emph{弦网模型} (string-net model),并得到相应的 Hamilton 量以及各种物理性质。

\section{拓扑序简介}

在 Landau 相变理论中,不同的相可根据对称性来确定,并可利用\emph{序参量} (order parameter) 来刻画,相变则通过对称性破缺以及序参量的变化来描述\cite{landau1980statistical,pathria2011statistical}。但在 20 世纪 80 年代以来,随着分数量子 Hall 效应、高温超导、手征自旋液体等现象的发现,人们意识到 Landau 的对称性破缺理论并不足以描述所有的物态。拓扑序的提出即为了描述这些物态的性质,并对其进行分类。其名称来源于手征自旋态的低能有效理论——拓扑量子场论 (topological quantum field theory, TQFT)\cite{witten1989quantum}。在 2+1 维中,拓扑序可由下面两条性质来确定\cite{wen2013topological,zeng2019introduction}:

\begin{itemize}
  \item \emph{基态的拓扑简并} (topological ground state degeneracy);
  \item \emph{非 Abel 几何相位} (non-Abelian geometric phase)。
\end{itemize}

我们以横场 Ising 模型
\begin{equation}
  H = -\sum_{\langle ij \rangle} \sigma^z_i\sigma^z_j - J \sum_i \sigma^x_i
\end{equation}
为例。当外场 $J=0$ 时,存在 $\mathbb{Z}_2$ 对称性 $\ket{\uparrow}\leftrightarrow\ket{\downarrow}$,因此其基态也有一个二重简并。但一旦增加一个小的微扰 (例如令 $J\neq0$),这种简并就会被破坏。拓扑序中的基态简并则与之不同,它在局域微扰下仍保持不变。注意到微扰往往会破坏系统的对称性,因此这种拓扑简并实际上是与对称性无关的,因而也就超越了传统 Landau 理论的范围。此外,拓扑序中的基态简并度通常还会和系统的拓扑有关,这在传统理论中也是无法解释的。

仅通过基态简并度并不足以完整地描述拓扑序,我们还需要引入一些新的拓扑不变量,即非 Abel 几何相位。设系统基态的简并度为 $g$,则非 Abel 几何相位定义为一个 $g\times g$ 的幺正矩阵 $U$,它可由系统底流形上的自同构给出。$U$ 会将 Hamilton 量 $H$ 变为 $H'$,并使这种变换的路径构成闭合环路,因而能够产生一个相位。以环面 (torus) 拓扑为例,底流形上有两种自同构,一种是剪切 (shear) 变形,另一种是挤压 (squeezing) 变形,它们分别对应了 $S$ 矩阵和 $T$ 矩阵。这两个矩阵也是\emph{模变换} (modular transformation) 的投影表示,从中还可以得到拓扑序的更多性质,例如分数统计等。

从微观角度来看,拓扑序的来源是\emph{长程纠缠} (long-range entanglement),它可以通过\emph{局域幺正变换} (local unitary transformation) 来定义。如图~\ref{fig:lu-transformation} 所示,对于深度为 $M$ 的\emph{量子线路} (quantum circuit),其中的局域幺正变换为
\begin{equation}
  U^M = U^{(1)} U^{(2)} \cdots U^{(M)},
\end{equation}
式中 $U^{(k)}$ 为一系列互相独立的量子门的乘积。

\begin{figure}[htb]
  \centering
  \includegraphics[width=0.8\textwidth]{images/temp/lu-transformation.pdf}
  \caption[量子线路中的局域幺正变换]{(a) 量子线路中的局域幺正变换,它由 (b) 作用在大小为 $l$ 的区域上的幺正变换组成。绿色部分反应了其因果结构。图片来源:\parencite{wen2013topological}。}
  \label{fig:lu-transformation}
\end{figure}

两个量子态属于同一个相,当且仅当它们之间可以通过局域幺正变换相联系。对于\emph{有能隙} (gapped) 的量子系统,其中的相可以分为短程纠缠态和长程纠缠态两类。短程纠缠态可以通过局域幺正变换转变为直积态,因而所有的短程纠缠态都属于同一个相;长程纠缠态无法通过局域幺正变换转变为直积态,而无法通过局域幺正变换相联系的态就构成了不同的相,它们也对应了不同的拓扑序。

2+1 维的拓扑序可以用张量范畴进行系统地描述与分类,更高维的拓扑序则需要借助高阶范畴来描述。接下来我们先从基本的范畴论出发,并向其中加入各种新的结构。

\section{范畴论基础}

\emph{范畴论} (category theory)\cite{baez2011physics,maclane2013categories,beer2018categories}用以抽象地刻画一些数学结构之间的关系,它主要描述了“对象”之间的作用,即\emph{映射} (mapping)。拓扑序理论中所研究的,正是带有了某些附加结构的范畴。

一个\emph{范畴} $\mathcal{C}$ 由其中的\emph{对象} (object) $x\in\mathcal{C}$ 和这些对象之间的\emph{态射} (morphism) $f\colon x\to y$ 组成。对象之间的态射满足以下三个条件:
\begin{itemize}
  \item \emph{复合性} (composition):对于范畴 $\mathcal{C}$ 中的对象 $x$、$y$、$z$,若 $f\colon x\to y$ 和 $g\colon y\to z$ 为态射,则存在复合态射 $g\circ f\colon x\to z$;
  \item \emph{结合律} (associativity):若 $\mathcal{C}$ 中有态射 $f\colon x\to y$、$g\colon y\to z$、$h\colon z\to w$,则有
    \begin{equation}
      (h\circ g)\circ f = h\circ (g\circ f);
    \end{equation}
  \item \emph{单位元} (identity):对于 $\forall x\in\mathcal{C}$,都存在恒等态射 $\id_x\colon x\to x$,使得
    \begin{equation}
      f \circ \id_x = \id_x \circ f = f, \quad \forall f\colon x\to y.
    \end{equation}
\end{itemize}
态射也可记为 $f\in\Hom_{\mathcal{C}}(x,y)$,其中 $\Hom_{\mathcal{C}}(x,y)$ 称为同态集 (hom-set)。如果 $x=y$,则称 $f$ 为\emph{自同态} (endomorphism),记为 $f\in\End_{\mathcal{C}}(x)$。

\emph{函子} (functor) 是范畴之间保结构的映射。具体而言,对于范畴 $\mathcal{C}$、$\mathcal{D}$,函子 $F\colon\mathcal{C}\to\mathcal{D}$ 会将 $\mathcal{C}$ 中的对象 $x$ 映射到 $\mathcal{D}$ 中的对象 $F(x)$,而将 $\mathcal{C}$ 中的态射 $f\colon x\to y$ 映射到 $\mathcal{D}$ 中的态射 $F_f\colon F(x)\to F(y)$,并且保持复合性与单位元的成立,即
\begin{align}
  F_{\id_x} &= \id_{F(x)} \colon F(x) \to F(x), \quad \forall x\in\mathcal{C}, \\
  F_{g\circ f} &= F_g \circ F_f \colon F(x) \to F(z), \quad \forall f\colon x\to y, \, g\colon y\to z.
\end{align}

在函子之上可进一步定义\emph{自然变换} (natural transformation)。对于两个函子 $F\colon\mathcal{C}\to\mathcal{D}$ 和 $G\colon\mathcal{C}\to\mathcal{D}$,自然变换 $\tau\colon F\Rightarrow G$ 由其分量 $\tau_x$(这是范畴 $\mathcal{D}$ 中的一个态射)定义:
\begin{equation}
  \tau_x\colon F(x)\to G(x), \quad \forall x\in\mathcal{C},
\end{equation}
它满足
\begin{equation}
  \tau_y \circ F_f = G_f \circ \tau_x, \quad \forall f\colon x\to y.
\end{equation}
如图~\ref{fig:natural-transformation},自然变换可以用交换图来直观地表示。若 $\tau$ 的每一分量 $\tau_x$ 均可逆(即存在 $\tau_x^{-1}$ 使得 $\tau^{-1}_x\circ\tau_x=\id_{F(x)}$ 且 $\tau_x\circ\tau^{-1}_x = \id_{G(x)}$),则称其为\emph{自然同构} (natural isomorphism),我们用 $\similarrightarrow$ 来表示。

\begin{figure}[htb]
  \centering
  \begin{tabular}{c@{\qquad}c}
    \includegraphics{images/natural-transformation-1.pdf} &
    \includegraphics{images/natural-transformation-2.pdf} \\
    (a) & (b)
  \end{tabular}
  \caption[自然变换对应的交换图]{自然变换对应的交换图。(a) 此图是“可交换的”,即从 $F(x)$ 到 $G(y)$ 的两条路径等价;(b) 对于结合律的“提升”,图中三棱柱的三个侧面都是可交换的。}
  \label{fig:natural-transformation}
\end{figure}

% \begin{example}
%   对于任意的交换环 $K$,以其中元素构成的 $n\times n$ 的非奇异矩阵组成了一般线性群 $GL_n(K)$。对于环同态 $f\colon K\to L$,显然可以构建群同态 $GL_n f\colon GL_n(K)\to GL_n(L)$。因此,$GL_n$ 即为交换环范畴 $\mathbf{CRng}$ 到群范畴 $\mathbf{Grp}$ 的函子。

%   设非奇异矩阵 $M\in GL_n(K)$,则其行列式 $\det_K(M)$ 为 $K$ 中的可逆元(即单位),因而 $\det_K$ 是群同态 $GL_n(K)\to K^\times$,其中 $K^\times$ 为 $K$ 的可逆元群。另一方面,当把环同态 $f$ 限制在可逆元群上时,可得群同态 $f^\times\colon K^\times\to L^\times$,因而 $(\cdot)^\times$ 同样也是 $\mathbf{CRng}$ 到 $\mathbf{Grp}$ 的函子。根据定义,$\det\colon GL_n\Rightarrow(\cdot)^\times$ 即为自然变换,它满足 $\det_L\circ\,GL_n f=f^\times\circ \det_K$。
% \end{example}

% \begin{example}
%   在 Haskell 编程语言中,类型为范畴 $\mathbf{Hask}$ 中的对象,而纯函数为态射。函子可由类型类 (type class) 来定义,例如 \verb|List| 可以将类型 \verb|T| 构造为对应的数组 \verb|[T]|,并且通过 \verb|fmap| 将以 \verb|T| 类型为参数的纯函数转换为以 \verb|[T]| 类型为参数的纯函数。自然变换则由参数多态 (parametric polymorphism) 实现。例如,一个安全的(不引发异常)返回数组首元素的函数可按下面的方式定义:
% \begin{verbatim}
%   head :: [T] -> Maybe T
%   head []     = Nothing
%   head (x:xs) = Just x
% \end{verbatim}
%   因此 \verb|head| 函数是从 \verb|List| 到 \verb|Maybe| 的自然变换。
% \end{example}

\emph{弦图} (string diagram) 可以更直观地描述范畴的概念。其中,带箭头的直线或曲线代表对象,盒子代表态射\cite{selinger2011survey,baez2011physics}:
\begin{equation}
  f\colon x\to y \quad \coloneq \quad
  \tikzinput{morphism} \, .
\end{equation}
态射的复合可以表示为连在一起的两个盒子:
\begin{equation}
  \tikzinput{composition}.
\end{equation}

\section{张量范畴与融合范畴}
\label{sec:tensor-category-fusion-category}

\subsection{张量范畴}

我们可以在范畴中引入一些结构,使其具有新的特性。引入了张量积的范畴为\emph{张量范畴} (tensor category)\cite{bakalov2001lectures,muger2008tensor,maclane2013categories,beer2018categories,kong2022invitation}。它最基本也最重要的例子是向量空间(或对应的范畴 $\mathbf{Vec}$),其中的张量积结构即为两个向量空间和相应线性变换的直积。一个张量范畴 $\mathcal{C}$ 由下面的条件定义:
\begin{itemize}
  \item \emph{张量积} (tensor product) $\otimes\colon\mathcal{C}\times\mathcal{C}\to\mathcal{C}$ 和\emph{单位对象} (unit object) $\1\in\mathcal{C}$;
  \item \emph{结合子} (associator) $\alpha$,它是一个自然同构:
    \begin{equation}
      \alpha_{x,y,z} \colon (x\otimes y)\otimes z \similarrightarrow x\otimes(y\otimes z), \quad \forall x,y,z \in \mathcal{C};
    \end{equation}
  \item \emph{左右单位子} (left/right unitor),同样也是自然同构:
    \begin{equation}
      \lambda_x \colon \1\otimes x \similarrightarrow x, \quad
      \rho_x    \colon x\otimes\1  \similarrightarrow x, \quad
      \forall x \in \mathcal{C}.
    \end{equation}
\end{itemize}
它们需要满足五边形方程和三角形方程(图~\ref{fig:pentagon-triangle-equation})。

\begin{figure}[htb]
  \centering
  \begin{tabular}{cc}
    \includegraphics{images/pentagon-equation.pdf} &
    \includegraphics{images/triangle-equation.pdf} \\
    (a) & (b)
  \end{tabular}
  \caption[五边形方程和三角形方程对应的交换图]{(a) 五边形方程和 (b) 三角形方程对应的交换图。}
  \label{fig:pentagon-triangle-equation}
\end{figure}

如果上述定义中的 $\similarrightarrow$ 可以取为等号,则称该张量范畴是\emph{严格} (strict) 的,此时 $\alpha$、$\lambda$ 和 $\rho$ 均为恒等变换。根据 \emph{Mac Lane 一致性定理} (Mac Lane coherence theorem),每个张量范畴都等价于一个严格张量范畴\cite{maclane2013categories}。因此我们之后可以只考虑严格张量范畴的情况。此时张量积表达式中的括号和单位对象都可以忽略。

一个张量范畴也是一个\emph{幺半群} (monoid),因为范畴的单位对象可以作为群的单位元,而张量积可以作为群乘法。所以张量范畴也被称为\emph{幺半范畴} (monoidal category)。

\subsection{对偶}

类似于对偶空间的概念,我们可以在张量范畴中引入\emph{对偶} (dual) 的概念。$x\in\mathcal{C}$ 的\emph{右对偶} (right dual) $x^\vee$ 通过以下两个态射定义:
\begin{equation}
  e_x\colon x^\vee\otimes x\to\1, \quad i_x\colon\1\to x\otimes x^\vee,
\end{equation}
它们需要满足\emph{刚性公理} (rigidity axioms):
\begin{equation}
  \begin{aligned}
    (\id_x\otimes e_x) \circ (i_x\otimes\id_x) &= \id_x, \\
    (e_x\otimes\id_{x^\vee}) \circ (\id_{x^\vee}\otimes i_x) &= \id_{x^\vee}.
  \end{aligned}
\end{equation}
如果在弦图中省略单位元(对应于物理中的真空态),则 $e_x$ 和 $i_x$ 可以表示为
\begin{equation}
  \tikzinput{dual-1},
  \qquad
  \tikzinput{dual-2}.
\end{equation}
这可以类比于量子力学中的湮灭与产生算符。(右对偶的)刚性公理则可表示为
\begin{equation}
  \tikzinput{rigidity-axioms-1},
  \qquad
  \tikzinput{rigidity-axioms-2}.
  \label{eq:rigidity-axioms-diagrams}
\end{equation}

同理,我们还可以定义\emph{左对偶} (left dual):
\begin{equation}
  e'_x \colon x\otimes \ldual{x}\to\1, \quad i'_x\colon \1\to\ldual{x}\otimes x.
  \label{eq:left-dual}
\end{equation}
这实际上只需要对上面的图沿水平方向做一下镜像操作。如果 $\mathcal{C}$ 中的每个对象既有左对偶也有右对偶,则称其为是\emph{刚性} (rigid) 的或\emph{自治} (autonomous) 的;而如果 $\forall x\in\mathcal{C}$,都有 $x=x^\vee$,则称 $\mathcal{C}$ 是\emph{自对偶} (self-dual) 的。

在有限维的情形中,向量空间 $V$ 的二重对偶 $V^{\vee\vee}$ 同构于 $V$。这在张量范畴中的推广即为\emph{中枢} (pivotal) 结构,它是由以下的自然同构给出的:
\begin{equation}
  \delta_x \colon x \similarrightarrow x^{\vee\vee}, \quad \forall x\in\mathcal{C},
\end{equation}
且需满足
\begin{equation}
  \delta_{x\otimes y} = \delta_x\otimes\delta_y, \quad
  \delta_{\1} = \id_{\1}, \quad
  \delta_{x^\vee} = (\delta_x^\vee)^{-1}.
\end{equation}
根据右对偶的定义,可有
\begin{equation}
  e_x\colon x^\vee\otimes x \similarrightarrow x^\vee\otimes x^{\vee\vee}\to\1, \quad
  i_x\colon \1\to x\otimes x^\vee \similarrightarrow x^{\vee\vee}\otimes x^\vee.
\end{equation}
对比式~\eqref{eq:left-dual},可以发现 $x^{\vee\vee}$ 也是 $x^\vee$ 的左对偶。若令 $y=x^\vee$,即有 $\ldual{y}=y^\vee$,也就是说在中枢范畴中可以不再区分左右对偶。

对于中枢范畴 $\mathcal{C}$ 中的自同态 $f\in\End_{\mathcal{C}}(x)$,可以定义\emph{左右迹} (left/right trace):
\begin{equation}
  \begin{aligned}
    \tr_{\text{L}} f &\colon \1
      \xrightarrow{i_{x^\vee}} x^\vee\otimes x^{\vee\vee}
      \xrightarrow{\id_{x^\vee}\otimes\delta_x^{-1}} x^\vee\otimes x
      \xrightarrow{\id_{x^\vee}\otimes f} x^\vee\otimes x
      \xrightarrow{e_x} \1, \\
    \tr_{\text{R}} f &\colon \1
      \xrightarrow{i_x} x\otimes x^\vee
      \xrightarrow{f\otimes\id_{x^\vee}} x\otimes x^\vee
      \xrightarrow{\delta_x\otimes\id_{x^\vee}} x^{\vee\vee}\otimes x^\vee
      \xrightarrow{e_{x^\vee}} \1.
  \end{aligned}
\end{equation}
当 $f=\id_x$ 时,还可以定义\emph{左右维数} (left/right dimension):
\begin{equation}
  \dim_{\text{L}} x \coloneq \tr_{\text{L}}\id_x, \quad
  \dim_{\text{R}} x \coloneq \tr_{\text{R}}\id_x.
  \label{eq:left-right-dimension}
\end{equation}
迹和维数可以用下图来描述:
\begin{equation}
  \begin{gathered}
    \tr_{\text{L}}  f = \tikzinput{trace-1} \, , \quad
    \tr_{\text{R}}  f = \tikzinput{trace-2} \, ; \\
    \dim_{\text{L}} x = \tikzinput{dimension-1} \!, \quad
    \dim_{\text{R}} x = \tikzinput{dimension-2} \!.
  \end{gathered}
\end{equation}
如果对于任意的 $f\in\End_{\mathcal{C}}(x)$ 都有 $\tr_{\text{L}}f=\tr_{\text{R}}f$,则称 $\mathcal{C}$ 是\emph{球状} (spherical) 的。

\subsection{融合范畴}

范畴中还可以引入\emph{直和} (direct sum) 的结构。若范畴 $\mathcal{C}$ 中的对象均可分解为\emph{简单对象} (simple object) 的直和:
\begin{equation}
  x = \bigoplus_{i\in I} n_i x_i, \quad \forall x \in \mathcal{C},
\end{equation}
其中 $x_i\in\mathcal{C}$ 是简单对象,$I$ 是由非零简单对象(等价类)构成的指标集,而系数 $n_i\in\mathbb{Z}_+$,则称 $\mathcal{C}$ 是一个\emph{半单范畴} (semi-simple category)。简单对象的例子包括向量空间范畴 $\mathbf{Vec}$ 中的一维空间(直线),以及 Abel 群范畴 $\mathbf{Ab}$ 中的单群。

若张量范畴 $\mathcal{C}$ 同时也是半单的,并且简单对象只有有限多个,那么这样的 $\mathcal{C}$ 称为\emph{融合范畴} (fusion category)\cite{bakalov2001lectures,kitaev2006anyons,bruillard2016rank,aasen2020topological,lou2021dummy}。此时简单对象的张量积可以写成:
\begin{equation}
  x_a \otimes x_b = \bigoplus_c N_{ab}^c x_c,
\end{equation}
其中 $N_{ab}^c\in\mathbb{Z}_+$,称为\emph{融合系数} (fusion coefficient)。在物理学中一般可以假设 $N_{ab}^c$ 只能取到 0 或 1,即是否允许该融合发生。所有允许的融合称为\emph{融合规则} (fusion rule)。融合范畴还要与张量范畴的结构相容,这意味着
\begin{equation}
  N_{\1 a}^b = N_{a\1}^b = \delta_{ab}, \quad
  \sum_x N_{ax}^z N_{bc}^x = \sum_y N_{ab}^y N_{yc}^z.
\end{equation}
我们还要求融合范畴是刚性且自对偶的,此时可以证明
\begin{equation}
  N_{ab}^c = N_{ba}^c = N_{ac}^b = N_{ca}^b = N_{bc}^a = N_{cb}^a,
\end{equation}
即融合系数关于所有指标均对称。

定义矩阵 $(N_a)_{bc}\coloneq N_{ab}^c$,根据 Perron--Frobenius 定理,可知 $N_a$ 存在最大的非负特征值,这定义为简单对象 $a$ 的\emph{量子维数} (quantum dimension)或 Perron--Frobenius 维数 $d_a$。可以证明,量子维数与式~\eqref{eq:left-right-dimension} 中通过迹定义的维数是相同的。融合范畴 $\mathcal{C}$ 的\emph{总量子维数} (total quantum dimension) 定义为
\begin{equation}
  D = \sqrt{\sum_{i\in I} d_i^2}.
\end{equation}

融合的逆运算是\emph{分裂} (splitting)。简单对象的融合与分裂都可以用弦图来表示:
\begin{equation}
  \tikzinput{fusion-tree-1}
  \in \Hom_{\mathcal{C}}(a\otimes b, c) \eqcolon V^{ab}_c, \quad
  \tikzinput{fusion-tree-2}
  \in \Hom_{\mathcal{C}}(c, a\otimes b) \eqcolon V_{ab}^c.
\end{equation}
由于 $\mathcal{C}$ 的刚性和自对偶性,我们可以省略弦图中的箭头。此时,$\Hom_{\mathcal{C}}(a\otimes b,c)$ 和 $\Hom_{\mathcal{C}}(c,a\otimes b)$ 都是向量空间,分别记为 $V^{ab}_c$ 和 $V_{ab}^c$,上面的“融合树”正是对应的基向量。这两个向量空间的维数相等,都等于融合系数 $N_{ab}^c$。

下面我们考虑向量空间 $V^{abc}_d$,它表示从 $a\otimes b\otimes c$ 到 $d$ 的融合。由于 $\mathcal{C}$ 是严格的,$(a\otimes b)\otimes c$ 和 $a\otimes(b\otimes c)$ 的结果相同(都等于 $d$),但却会给出两种融合树的分支结构:
\begin{equation}
  \tikzinput{f-symbol-1}
  \in \bigoplus_x V^{ab}_x \otimes V^{xc}_d \simeq V^{abc}_d, \quad
  \tikzinput{f-symbol-2}
  \in \bigoplus_x V^{ay}_d \otimes V^{bc}_y \simeq V^{abc}_d.
\end{equation}
联系这两组基的变换称为 \emph{$F$ 移动} ($F$ move):
\begin{equation}
  \tikzinput{f-symbol-1}
  = \sum_y \, \bigl[ F^{abc}_d \bigr]_{xy}
  \tikzinput{f-symbol-2}.
  \label{eq:f-move}
\end{equation}
式中的系数 $[F^{abc}_d]_{xy}$ 称为 \emph{$F$ 符号} ($F$ symbol),它一共有 6 个指标。$\mathcal{C}$ 中的 $F$ 符号并不是独立的。如图~\ref{fig:f-symbols-pentagon-equation} 所示,它们所满足的约束条件同样也是五边形方程。

\begin{figure}[htb]
  \centering
  \includegraphics{images/f-symbols-pentagon-equation.pdf}
  \caption[$F$ 符号所满足的五边形方程]{$F$ 符号所满足的五边形方程,对应的融合空间是 $\Hom_{\mathcal{C}}(a\otimes b\otimes c\otimes d,e)$,即 $V^{abcd}_e$。}
  \label{fig:f-symbols-pentagon-equation}
\end{figure}

一个融合范畴可由以下几组数据完全确定:
\begin{itemize}
  \item 简单对象(等价类)的集合 $\{a,b,c,\dots\}$;
  \item 融合系数 $N_{ab}^c$ 或者对应的融合规则;
  \item $F$ 符号 $[F^{abc}_d]_{xy}$。
\end{itemize}
利用这些数据可以对任意弦图进行化简或求值。如果一个弦图没有端点(即“外腿”),那么它的化简结果将是一个复数。根据上文,我们可以进行的操作有:
\begin{itemize}
  \item 任意的连续变形,这是由刚性公理式~\eqref{eq:rigidity-axioms-diagrams} 和自对偶性保证的;
  \item $F$ 移动,即式~\eqref{eq:f-move};
  \item 环路消除,这会给出相应的量子维数:
    \begin{equation}
      \tikzinput{loop-removal-1}
      \! = d_a, \quad
      \tikzinput{loop-removal-2}
      \! = \delta_{ac} \sqrt{\frac{d_b d_{b'}}{d_a}} \enspace
      \tikzinput{loop-removal-3}.
    \end{equation}
\end{itemize}

\subsection{融合范畴的例子}
\label{subsec:fusion-category-examples}

下面我们给出一些融合范畴的例子\cite{rowell2009classification}。这里只列出了非平凡的 $F$ 符号,对于其他合法的融合树,它们之间基变换的 $F$ 符号都等于 1。

\paragraph{Semion}

\begin{itemize}
  \item 简单对象:$\{\1, s\}$;
  \item 量子维数:$d_{\1}=d_s=1$;
  \item 总量子维数:$D=\sqrt2$;
  \item 融合规则:
    \begin{fusionrules}{|c|cc|}
      $\otimes$ & $\1$ & $s$  \\ \hline
      $\1$      & $\1$ & $s$  \\
      $s$       & $s$  & $\1$ \\
    \end{fusionrules}
  \item $F$ 符号:$[F^{sss}_s]_{\1\1}=-1$。
\end{itemize}

\paragraph{Fibonacci}

\begin{itemize}
  \item 简单对象:$\{\1, \tau\}$;
  \item 量子维数:$d_{\1}=1, \, d_\tau=\varphi$;
  \item 总量子维数:$D=\sqrt{\frac{5+\sqrt5}{2}}=2\cos\frac{\pi}{10}$;
  \item 融合规则:
    \begin{fusionrules}{|c|cc|}
      $\otimes$ & $\1$   & $\tau$ \\ \hline
      $\1$      & $\1$   & $\tau$ \\
      $\tau$    & $\tau$ & $\1\oplus\tau$ \\
    \end{fusionrules}
  \item $F$ 符号:
    $
      [F^{\tau\tau\tau}_\tau]_{ij} = \dfrac1\varphi \begin{pmatrix} 1 & \sqrt\varphi \\ \sqrt\varphi & -1 \end{pmatrix}, \,
      i,j \in \{\1, \tau\}
    $。
\end{itemize}

这里的 $\varphi=(1+\sqrt5)/2$ 是黄金比\cite{trebst2008short}。

\paragraph{Ising}

\begin{itemize}
  \item 简单对象:$\{\1, \sigma, \psi\}$;
  \item 量子维数:$d_{\1}=d_\psi=1, \, d_\sigma=\sqrt2$;
  \item 总量子维数:$D=2$;
  \item 融合规则:
    \begin{fusionrules}{|c|ccc|}
      $\otimes$ & $\1$     & $\sigma$       & $\psi$   \\ \hline
      $\1$      & $\1$     & $\sigma$       & $\psi$   \\
      $\sigma$  & $\sigma$ & $\1\oplus\psi$ & $\sigma$ \\
      $\psi$    & $\psi$   & $\sigma$       & $\1$     \\
    \end{fusionrules}
  \item $F$ 符号:
    $
      [F^{\psi\sigma\psi}_\sigma]_{\sigma\sigma} = [F^{\sigma\psi\sigma}_\psi]_{\sigma\sigma} = -1, \,
      [F^{\sigma\sigma\sigma}_\sigma]_{ij} = -\dfrac{1}{\sqrt2} \begin{pmatrix} 1 & 1 \\ 1 & -1 \end{pmatrix}, \,
      i,j \in \{\1, \psi\}
    $。
\end{itemize}

\paragraph{Toric code}

\begin{itemize}
  \item 简单对象:$\{\1, e, m, f\}$;
  \item 量子维数:$d_{\1}=d_e=d_m=d_f=1$;
  \item 总量子维数:$D=2$;
  \item 融合规则:
    \begin{fusionrules}{|c|cccc|}
      $\otimes$ & $\1$ & $e$  & $m$  & $f$  \\ \hline
      $\1$      & $\1$ & $e$  & $m$  & $f$  \\
      $e$       & $e$  & $\1$ & $f$  & $m$  \\
      $m$       & $m$  & $f$  & $\1$ & $e$  \\
      $f$       & $f$  & $m$  & $e$  & $\1$ \\
    \end{fusionrules}
  \item $F$ 符号:都等于 1。
\end{itemize}

\subsection{更复杂的例子:\texorpdfstring{$\mathcal{A}_{k+1}$}{𝒜ₖ₊₁} 范畴}
\label{subsec:A-k+1-category}

$\mathcal{A}_{k+1}$ 范畴\cite{coquereaux2007racah,aasen2020topological,chen2022galois}也称 $\mathfrak{su}(2)_k$ 模型,它最初来自于对 Lie 代数表示的研究。该范畴中共有 $k+1$ 个简单对象,标记为 $0,\frac12,1,\dots,\frac k2$,对应的量子维数为
\begin{equation}
  d_i = \frac{\sin\frac{(2i+1)\pi}{k+2}}{\sin\frac{\pi}{k+2}}, \quad i = 0,1,2,\dots,k.
\end{equation}
计算可得总量子维数
\begin{equation}
  D = \frac12 \left( \cot^2 \frac{\pi}{2k+4} - 1 \right).
\end{equation}
融合规则由下式确定:
\begin{equation}
  N_{ab}^c = \begin{cases}
    1, & a+b\geqslant c, \, b+c\geqslant a, \, c+a\geqslant b, \, a+b+c\leqslant k, \, a+b+c\in\mathbb{Z}; \\
    0, & \text{otherwise}.
  \end{cases}
\end{equation}
例如当 $k=1$ 和 2 时,融合规则分别为
\begin{fusionrules}{|c|cc|}
  $\otimes$ & 0 & ½ \\ \hline
  0         & 0 & ½ \\
  ½         & ½ & 0 \\
\end{fusionrules}
和
\begin{fusionrules}{|c|ccc|}
  $\otimes$ & 0 & ½          & 1 \\ \hline
  0         & 0 & ½          & 1 \\
  ½         & ½ & $0\oplus1$ & ½ \\
  1         & 1 & ½          & 0 \\
\end{fusionrules}。

\begin{equation}
  G^{abc}_{def} = \begin{Bmatrix} a & b & c \\ d & e & f \end{Bmatrix}_q (-1)^p,
\end{equation}
式中
\begin{equation}
  p = \frac{3(a+b+c+d+e+f)^2 - (a+d)^2 - (b+e)^2 - (c+f)^2}{2},
\end{equation}
而 $\{\dots\}_q$ 称为 \emph{$q$-变形 Wigner $6j$ 符号} ($q$-deformed Wigner $6j$ symbol),其定义为
\begin{align}
  \begin{Bmatrix} j_1 & j_2 & j_3 \\ j_4 & j_5 & j_6 \end{Bmatrix}_q
  &= \Delta(j_1,j_2,j_3) \, \Delta(j_1,j_5,j_6) \, \Delta(j_4,j_2,j_6) \, \Delta(j_4,j_5,j_3) \notag \\
  &\times \sum_{z=m}^M \frac{(-1)^z [z+1]!}{[z-a_1]![z-a_2]![z-a_3]! [b_1-z]![b_2-z]![b_3-z]!},
\end{align}
其中
\begin{align}
  q    &= \exp\frac{2\pi\ii}{k+2}, \\
  [n]  &= \frac{q^{m/2} - q^{-m/2}}{q^{1/2} - q^{-1/2}} = \frac{\sin\frac{n\pi}{k+2}}{\sin\frac{\pi}{k+2}}, \\
  [n]! &= \begin{cases}
    \prod_{i=1}^{n} [n], & n \geqslant 1; \\
    1                    & n = 0,
  \end{cases} \\
  \Delta(x,y,z) &= \sqrt{\frac{[x+y-z]! [y+z-x]! [z+x-y]!}{[x+y+z+1]!}},
\end{align}
此外
\begin{equation}
  \begin{gathered}
    a_1=j_1+j_2+j_3, \quad a_2=j_1+j_5+j_6, \quad a_3=j_2+j_4+j_6, \quad a_4=j_3+j_4+j_5, \\
    b_1=j_1+j_2+j_4+j_5, \quad b_2=j_2+j_3+j_5+j_6, \quad b_3=j_1+j_3+j_4+j_6,
  \end{gathered}
\end{equation}
而对 $z$ 的求和为从 $m=\max\{a_1,a_2,a_3,a_4\}$ 到 $M=\min\{b_1,b_2,b_3\}$ 的整数。

\subsection{模张量范畴与共形场论}

在二维共形场论 (conformal field theory, CFT) 的最小模型 (minimal model) 中,初级场 (primary field) 之间也存在融合规则\cite{ginsparg1988applied,francesco2012conformal}:
\begin{equation}
  \phi_{(r,s)} \times \phi_{(m,n)}
  = \sum_{\substack{k=1+|r-m| \\ k+r+m=1 \bmod 2}}^{k_{\max}} \,
    \sum_{\substack{l=1+|s-n| \\ l+s+n=1 \bmod 2}}^{l_{\max}} \phi_{(k,l)},
\end{equation}
式中
\begin{equation}
  \begin{aligned}
    k_{\max} &= \min(r+m-1, 2p'-1-r-m), \\
    l_{\max} &= \min(s+n-1, 2p-1-s-n),
  \end{aligned}
\end{equation}
而 $p$ 和 $p'$ 是两个互素的整数,它们确定了最小模型 $\mathcal{M}(p,p')$。以二维临界 Ising 模型为例,它是最小模型 $\mathcal{M}(4,3)$,其中的初级场除了恒等算符 $\1$ 之外,还包括自旋 $\sigma$ 和能量密度 $\epsilon$。它们满足的融合规则为:
\begin{equation}
  \sigma\times\sigma = \1+\epsilon, \quad
  \sigma\times\epsilon = \sigma, \quad
  \epsilon\times\epsilon = \1.
\end{equation}
这与上面 Ising 融合范畴中(非平凡)的融合规则是完全一致的。

事实上,我们在 \ref{subsec:fusion-category-examples} 和 \ref{subsec:A-k+1-category} 小节中给出的这几个例子不仅是融合范畴,还是所谓\emph{模张量范畴} (modular tensor category, MTC)\cite{bakalov2001lectures,kitaev2006anyons,bruillard2016rank,beer2018categories,kong2022invitation}。它在融合范畴的基础上加入了\emph{编织} (braiding)、\emph{丝带} (ribbon) 的结构以及\emph{非简并} (non-degeneracy) 等条件。编织范畴或称辫子范畴定义了如下的自然变换:
\begin{equation}
  \sigma_{x,y} = \tikzinput{braiding} \, , \quad x,y \in \mathcal{C}.
\end{equation}
丝带范畴则定义了\emph{扭转} (twist):
\begin{equation}
  \theta_x = \tikzinput{twist} \, , \quad x \in \mathcal{C}
\end{equation}
和对应的 \emph{$T$ 矩阵} ($T$ matrix):
\begin{equation}
  T_{ij} = \theta_i \delta_{ij}, \quad x \in \mathcal{C},
\end{equation}
以及 \emph{$S$ 矩阵} ($S$ matrix):
\begin{equation}
  S_{ij} = \frac1D \, \tikzinput{s-matrix}, \quad i,j \in I,
\end{equation}
非简并条件要求 $S$ 矩阵可逆。此外还可定义\emph{Gauss 和} (Gauss sums):
\begin{equation}
  p^{\pm} = \sum_{i\in I} \theta_i^{\pm1} d_{i}^2.
\end{equation}
它与 $\mathcal{C}$ 的总量子维数相关:
\begin{equation}
  p^+ p^- = D^2,
\end{equation}
而其比值定义为
\begin{equation}
  \zeta = \left( \frac{p^+}{p^-} \right)^{1/6}.
\end{equation}
可以证明模张量范畴中融合系数与 $S$ 矩阵之间存在如下关系\cite{verlinde1988fusion,bakalov2001lectures,huang2005vertex,bruillard2016rank}:
\begin{equation}
  N_{ij}^k = \sum_r \frac{S_{ir} S_{jr} S_{k^\vee r}}{S_{\1 r}},
\end{equation}
这被称为 \emph{Verlinde 公式} (Verlinde formula)。此外,还存在 \emph{Vafa 定理} (Vafa's theorem),即 $\theta_i$ 和 $\zeta$ 都是单位根\cite{bakalov2001lectures}。此时令
\begin{equation}
  \theta_i = \ee^{2\pi\ii\Delta_i}, \quad
  \zeta = \ee^{2\pi\ii c/24},
\end{equation}
可知式中 $\Delta_i$ 和 $c$ 都是有理数。它们正是\emph{有理共形场论} (rational CFT, RCFT) 中的\emph{标度维数} (scaling dimension) 和\emph{中心荷} (central charge)(参考 \ref{sec:cft-review} 节)。

\section{Toric code 模型}

拓扑序的一个基本且重要的例子是 toric code 模型\cite{kitaev2003fault,kong2022invitation}。它定义在一个二维正方形网格 (square lattice) 上,每条边上都存在一个自旋 $1/2$ 的自由度。其 Hamilton 量定义为
\begin{equation}
  H = -\sum_v A_v - \sum_p B_p,
  \label{eq:toric-code-hamiltonian}
\end{equation}
其中
\begin{equation}
  A_v = \prod_{i\in\operatorname{star}(v)} \sigma_i^x, \quad
  B_p = \prod_{i\in\operatorname{boundary}(p)} \sigma_i^z,
\end{equation}
称为\emph{稳定化子} (stabilizer),而 $\sum_v$ 和 $\sum_p$ 分别表示对所有的顶点 (vertex) 和元格 (plaquette) 求和,如图~\ref{fig:toric-code} 所示。

\begin{figure}[htb]
  \centering
  \tikzinput{toric-code}
  \caption[正方形网格上的 toric code 模型]{正方形网格上的 toric code 模型,其中顶点 $v$ 和元格 $p$ 各包含 4 个自旋自由度。图片来源:\parencite{kong2022invitation}。}
  \label{fig:toric-code}
\end{figure}

\subsection{基态}

利用 Pauli 矩阵的运算规则很容易知道 $A_v$、$B_p$ 之间互相都是对易的,而又因为 $A_v$、$B_p$ 的本征值为 $\pm1$,所以整个系统的基态将是 $A_v$、$B_p$ 共同的本征态,并且对应的本征值为 1。显然,在基态中翻转任意自旋,都将会使相邻 $A_v$、$B_p$ 算符的本征值由 $+1$ 变为 $-1$,因此无论系统尺寸大小,toric code 的 Hamilton 量始终具有大小为 2 的能隙。

Toric code 中基态简并度与空间底流形的拓扑有关。设整个网格由 $V$ 个顶点、$E$ 条边和 $F$ 个元格(面)构成,根据 Euler 公式,有
\begin{equation}
  V - E + F = 2 - 2g,
\end{equation}
其中 $g$ 为空间的\emph{亏格} (genus)。对于周期性边界条件(即环面拓扑)而言,亏格 $g=1$。由于自由度定义在边上,所以系统 Hilbert 空间的维度为 $2^E$。而基态子空间则由约束 $A_v=1$ 和 $B_p=1$ 确定。注意到
\begin{equation}
  \prod_v A_v = \prod_p B_p = 1,
\end{equation}
因此总的约束数目为 $V+F-2$,而基态简并度 $d$ 为
\begin{equation}
  d = 2^{E-(V+F-2)} = 2^{2g} = 4^g.
\end{equation}

\subsection{激发态与任意子}

在 toric code 中,对于顶点 $v_0$,存在激发态 $\ket*{\psi_{v_0}}$,它满足
\begin{equation}
  A_{v_0} \ket*{\psi_{v_0}} = -\ket*{\psi_{v_0}}, \quad
  A_v \ket*{\psi_{v_0}} = B_p \ket*{\psi_{v_0}} = \ket*{\psi_{v_0}}, \quad \forall v \neq v_0, p;
\end{equation}
同样,对于元格 $p_0$,也存在 $\ket*{\psi_{p_0}}$,满足
\begin{equation}
  B_{p_0} \ket*{\psi_{p_0}} = -\ket*{\psi_{p_0}}, \quad
  A_v \ket*{\psi_{p_0}} = B_p \ket*{\psi_{p_0}} = \ket*{\psi_{p_0}}, \quad \forall p \neq p_0, v.
\end{equation}
$\ket*{\psi_{v_0}}$ 和 $\ket*{\psi_{p_0}}$ 分别定义了\emph{电荷} (electric charge) $e$ 和\emph{磁通} (magnetic flux) $m$ 两种拓扑激发态。如图~\ref{fig:toric-code-string-operators} 所示,它们也可以由 Pauli 矩阵 $\sigma^z$ 和 $\sigma^x$ 组成的非局域弦算符(成对)产生。

\begin{figure}[htb]
  \centering
  \tikzinput{toric-code-string-operators}
  \caption[弦算符产生拓扑激发]{拓扑激发可由非局域的弦算符(成对)产生。图片来源:\parencite{kong2022invitation}。}
  \label{fig:toric-code-string-operators}
\end{figure}

$e$ 和 $m$ 分别绕对方一圈(例如使用 $\sigma^z$ 组成的 $e$ 闭弦包围一个 $m$ 激发),会给出 $-1$ 的相位。绕一圈相当于两次交换,所以每次交换相当于为波函数增加了 $\ii$ 的相位,这显然是无法用 Bose 或 Fermi 统计来描述的,因此这些拓扑激发也称为\emph{任意子} (anyon)。

除了 $e$ 和 $m$ 之外,toric code 中还存在 $\1$ 和 $f$ 两种任意子。$\1$ 相当于一个空的激发态,即基态;而 $f=e\otimes m$ 则相当于邻近的 $e$、$m$ 所构成的复合粒子。这些任意子一旦相互邻近,它们之间也可以相互复合,或称\emph{融合} (fuse),并且也满足\emph{融合规则} (fusion rules):
\begin{equation}
  \begin{gathered}
    e \otimes e = m \otimes m = f \otimes f = \1, \\
    e \otimes m = m \otimes e = f, \quad
    e \otimes f = f \otimes e = m, \quad
    m \otimes f = f \otimes m = e.
  \end{gathered}
\end{equation}

% TODO: quantum double
可以发现,这里的融合规则与 \ref{subsec:fusion-category-examples} 小节中 toric code 范畴中的融合规则是完全相同的。实际上,toric code 模型是基于 $\mathbb{Z}_2$ 群的 quantum double 模型,其进一步推广则可以得到基于张量融合范畴的弦网模型。

\section{弦网模型}

\emph{弦网模型} (string-net model)\cite{levin2005string,levin2006detecting}可以视为 toric code 模型的一种推广,它是通过 \ref{sec:tensor-category-fusion-category} 节介绍的张量融合范畴来定义的\footnote{准确来说要求模张量范畴。}。弦网模型一般定义在一个三分支 (trivalent) 的网格上,网格的边上带有“弦”的自由度。对于一个具体的弦网模型,我们需要给出下列几组数据:
\begin{itemize}
  \item \emph{弦的类型},按照量子力学的语言也被称为\emph{超选择分区} (superselection sector),它相当于融合范畴中简单对象的集合;
  \item \emph{弦的取向},相当于指定了简单对象的对偶;
  \item \emph{分支规则},相当于融合规则,规定了哪些弦允许汇聚于一点。
\end{itemize}

根据张量融合范畴的精神,弦网模型的基态波函数 $\Phi$ 应该满足下面的约束条件:
\begin{equation}
  % TODO: update image and conventions
  \includegraphics{images/temp/string-net-wave-function.pdf}
\end{equation}
这里的 $F^{ijm}_{kln}$ 是 $F$ 符号的另一种记法,它与 $F[...]$ 的关系将在 [...] 中给出。

在六边形网格 (hexagonal lattice) 上定义的弦网模型,其 Hamilton 量由下式给出:
\begin{equation}
  H = -\sum_v A_v - \sum_p B_p.
  \label{eq:string-net-hamiltonian}
\end{equation}
与式~\eqref{eq:toric-code-hamiltonian} 类似,这里 $v$ 和 $p$ 也表示顶点和元格。$A_v$ 称为\emph{电荷算符} (electric charge operator):
\begin{equation}
  % A_v \ket{...} = \delta_{ijk} \ket{...}
  \includegraphics{images/temp/string-net-electric-charge.pdf}
\end{equation}
$B_p$ 称为\emph{磁通量算符} (magnetic flux operator):
\begin{equation}
  B_p = \sum_{s=0}^N \frac{d_s}{D} B_p^s,
\end{equation}
其中
\begin{equation}
  \begin{gathered}
    % B_p^s \ket{...} = \sum ... \ket{...}
    \includegraphics{images/temp/string-net-magnetic-flux.pdf}, \\
    B_{p,ghijkl}^{s,g'h'i'j'k'l'}(abcdef) =
      F^{al^*g}_{s^*g'l'^*}
      F^{bg^*h}_{s^*h'g'^*}
      F^{ch^*i}_{s^*i'h'^*}
      F^{di^*j}_{s^*j'i'^*}
      F^{ej^*k}_{s^*k'j'^*}
      F^{fk^*l}_{s^*l'k'^*}.
  \end{gathered}
  \label{eq:string-net-bp}
\end{equation}
可以验证 $A_v$ 与 $B_p$ 是对易的,因而 Hamilton 量式~\eqref{eq:string-net-hamiltonian} 严格可解。

\section{本章小结}

我们在表~\ref{tab:tensor-category-ingredients} 中总结了张量范畴中的各种结构。

\begin{table}[htb]
  \centering
  \begin{tabular}{cc}
    \toprule
      名称 & 含义 \\
    \midrule
      张量 / 幺半 (tensor / monoidal)   & 定义了张量积 $\otimes$  \\
      编织 / 辫子 (braided)             & 定义了编织结构 $\sigma$ \\
      融合        (fusion)              & 张量积可以分解为直和    \\
      对称        (symmetric)           & $\sigma=\sigma^{-1}$    \\
      刚性 / 自治 (rigid / autonomous)  & 存在左右对偶            \\
      中枢        (pivotal / sovereign) & 左右对偶相等            \\
      球状        (spherical)           & 左右迹相等              \\
      丝带        (ribbon)              & 定义了扭转结构 $\theta$ \\
      模          (modular)             & $S$ 矩阵可逆            \\
    \bottomrule
  \end{tabular}
  \caption[张量范畴中的各种结构]{张量范畴中的各种结构。}
  \label{tab:tensor-category-ingredients}
\end{table}


\tikzset{x=1em, y=1em, node font=\footnotesize}

\chapter{张量网络方法介绍}
\label{chap:tensor-network}

\section{基本概念}

\emph{张量网络} (tensor network)\cite{orus2014practical,bridgeman2017hand,biamonte2017tensor,orus2019tensor,ran2020tensor,evenbly2022practical} 为凝聚态物理、量子信息、机器学习等领域提供了一套统一的描述框架。顾名思义,张量网络就是根据一定规则连接起来张量单元。这里的\emph{张量} (tensor) 可以简单理解为多维数组,即标量(0 维张量)、向量(1 维张量)、矩阵(2 维张量)的推广。

如图~\ref{fig:tensors} 所示,张量网络常利用图形方式来描述,其中圆圈(也可用其他图形)表示张量,延伸出来的腿表示张量的指标。

\begin{figure}[htb]
  \centering
  \includegraphics[width=0.6\textwidth]{images/temp/tensors.pdf}
  \caption[张量单元]{三种张量单元,分别为向量、矩阵和三阶张量。}
  \label{fig:tensors}
\end{figure}

\subsection{基本张量运算}

常用的张量运算包括张量积、缩并、变形等。

\emph{张量积} (tensor product) 其实就是把若干个张量并排放置,并保持指标不变:
\begin{equation}
  (A \otimes B)_{i_1,\dots,i_r,j_1,\dots,j_s} \coloneq A_{i_1,\dots,i_r} B_{j_1,\dots,j_s}.
\end{equation}
图形描述为
\begin{equation}
  [[TODO:]]
\end{equation}

张量的\emph{缩并} (contraction) 是向量内积、矩阵乘法的推广,即对两个张量的某些指标进行求和,得到一个新的张量:
\begin{equation}
  C_{abc} = \sum_k A_{abk} B_{kc} \eqcolon A_{abk} B_{kc}
\end{equation}
这里根据 Einstein 求和约定省略了求和号。张量缩并的图形描述为
\begin{equation}
  \mbox{\includegraphics[width=0.75\textwidth]{images/temp/contraction.pdf}}
\end{equation}
即把需要缩并的腿(指标)连接起来。

同一个张量的指标也可以缩并,这样就得到了\emph{迹} (trace) 或\emph{偏迹} (partial trace):
\begin{equation}
    (\tr_{x,y} A)_{i_1,\dots,i_{x-1},i_{x+1},\dots,i_{y-1},u_{y+1},\dots,i_r}
  = A_{i_1,\dots,i_{x-1},k,i_{x+1},\dots,i_{y-1},k,i_{y+1},\dots,i_r}.
\end{equation}
其图形描述与缩并类似:
\begin{equation}
  [[TODO:]]
\end{equation}
利用图形语言很容易验证 $\tr(AB)=\tr(BA)$:
\begin{equation}
  [[TODO:]]
\end{equation}

张量的\emph{变形} (reshape) 相当于指标的重新组合,例如
\begin{equation}
  \mbox{\includegraphics[width=0.5\textwidth]{images/temp/reshape.pdf}}
\end{equation}
在数值计算中,把一般形状的张量变形为矩阵,往往可以利用为矩阵优化的算法来加速计算。

\subsection{张量分解}

张量的\emph{分解} (decomposition) 可以理解为缩并的逆运算,最常用的是\emph{奇异值分解} (singular value decomposition, SVD),即
\begin{equation}
  M_{ij} = \sum_{k=1}^{\min(m,n)} U_{ik} \Lambda_{kk} V_{kj} = \sum_k U_{ik} \Lambda_{kk} \bigl( V^\dagger \bigr)_{jk},
\end{equation}
其中 $M$ 为 $m\times n$ 矩阵,$\Lambda$ 为对角矩阵(如果行数、列数不等,则相应补零),$U$ 和 $V$ 为幺正矩阵,$V^\dagger$ 表示共轭转置。利用张量的变形,可以很容易地将 SVD 推广为一般形状的张量,用图形描述为
\begin{equation}
  [[TODO:]]
\end{equation}

奇异值分解可以用来获得张量的近似表示。将奇异值($\Lambda$ 矩阵的对角元)按大小排列后,只保留前 $r$ 个值,就得到了秩为 $r$ 的近似张量:
\begin{equation}
  M_{ij} \approx \sum_{k=1}^r U_{ik} \Lambda_{kk} V_{kj}, \quad r < \min(m,n).
\end{equation}
并且可以证明,这种近似表示是所有秩为 $r$ 的张量中最优的。

对于一个一般的张量,原则上我们总可以把它改写为若干个小张量的缩并形式,并保证自由度数目不变。这种改写并不能降低计算复杂度,但我们可以利用奇异值分解把这些小张量替换为对应的近似表示,这样就可以大幅减少总自由度的数目。

在量子力学的语境中,奇异值分解也可表述为 \emph{Schmidt 分解} (Schmidt decomposition):对于 Hilbert 空间 $\mathcal{H}^{\text{L}}\otimes\mathcal{H}^{\text{R}}$,其中的任意量子态 $\ket{\Psi}$ 均可被分解为
\begin{equation}
  \ket{\Psi} = \sum_\alpha \lambda_\alpha \ket{\Phi^{\text{L}}_\alpha} \otimes \ket{\Phi^{\text{R}}_\alpha},
  \label{eq:schmidt-decomposition}
\end{equation}
其中 $\lambda_\alpha$ 称为 \emph{Schmidt 系数} (Schmidt coefficients),而 $\ket{\Phi^{\text{L}}_\alpha}$、$\ket{\Phi^{\text{R}}_\alpha}$ 分别是 $\mathcal{H}^{\text{L}}$、$\mathcal{H}^{\text{R}}$ 中的单位正交基,称为 \emph{Schmidt 向量} (Schmidt vectors)。可以发现 Schmidt 系数和奇异值是等价的,而 Schmidt 向量和幺正矩阵 $U$、$V$ 也是等价的。

\section{矩阵乘积态算法}

接下来介绍几种具体的张量网络及其算法。本文中,我们把这些算法分为两类:一类算法以矩阵乘积态为代表,主要用来处理一维量子多体系统;另一类算法则以张量重整化群为代表,顾名思义,主要用在二维网络的重整化(粗粒近似)操作中。

\subsection{波函数的构造}
\label{subsec:mps-construction}

% \citet{cirac2021matrix}

如图~\ref{fig:mps} 所示,\emph{矩阵乘积态} (matrix product state, MPS)\cite{perez2007matrix,verstraete2008matrix,orus2014practical} 由一系列张量单元首尾相连而成,每个张量单元包含三个指标,没有缩并的称为\emph{物理指标} (physical index),另外两个则称为\emph{虚拟指标} (virtual index) 或\emph{辅助指标} (auxiliary index)。根据具体问题的需要,MPS 可取开放或周期性边界条件。当所研究的系统具有平移对称性时,可以将所有张量单元取为相同值,此时有
\begin{equation}
  % TODO: 周期性边界条件
  \Psi_{i_1 i_2 \cdots i_n} = A^{j_1 j_2}_{i_1} A^{j_2 j_3}_{i_2} \cdots A^{j_n j_1}_{i_n}.
\end{equation}
这样的张量网络称为\emph{无限 MPS} (infinite MPS, iMPS)。我们之后的讨论主要针对 iMPS。

\begin{figure}[htb]
  \centering
  \includegraphics[width=0.6\textwidth]{images/temp/mps.png}
  \caption[矩阵乘积态的示意图]{矩阵乘积态的示意图。图 (a)、(b) 分别对应开放和周期性边界条件。}
  \label{fig:mps}
\end{figure}

考虑一个量子态 $\ket{\Psi}$,我们对它作用一次 SVD,便可以获得由两个张量组成的 MPS,此时连接维数 $\chi$ 等于奇异值的数量。这一过程可以不断重复,以构造出任意长度的 MPS。在做 SVD 时,如果不对奇异值进行截断,可以发现 $\chi$ 会随着分解的次数(也即张量单元的数量)$N$ 指数级增长。然而对于一维\emph{有能隙} (gapped) Hamilton 量的基态以及低能激发态,$\chi$ 可以取为常数;对于\emph{无能隙} (gapless) 的临界系统,则会以多项式形式发散。

% TODO: \emph{面积定律} (area law)

由上述方法构造出的 MPS 并不是唯一的,而是存在所谓\emph{规范自由度} (gauge freedom)\cite{bridgeman2017hand}。如图~\ref{fig:mps-gauge-freedom} 所示,两个张量单元之间总可以插入单位矩阵 $I=XX^{-1}$,并把 $X$ 和 $X^{-1}$ 分配到两边。

\begin{figure}[htb]
  \centering
  % \includegraphics[width=0.6\textwidth]{images/temp/mps.png}
  % \caption[]{}
  \caption{规范自由度}
  \label{fig:mps-gauge-freedom}
\end{figure}

% TODO: 密度矩阵
为了方便计算,我们一般会把 MPS 取为\emph{正则形式} (canonical form)\cite{orus2008infinite,schollwock2011density,orus2014practical},它要求辅助指标由 Schmidt 分解式~\eqref{eq:schmidt-decomposition} 决定。张量单元 $A$ 会被拆分成 $\Gamma$ 和 $\lambda$ 两部分,分别对应 Schmidt 向量与系数。定义左右矩阵
\begin{equation}
  \begin{aligned}
       R_{(\alpha\alpha'), \, (\beta\beta')}
    &= \sum_{i=1}^d \left( \Gamma^i_{\alpha\beta} \lambda_\beta \right) \left( \Gamma^i_{\alpha'\beta'} \lambda_{\beta'} \right)^*, \\
       L_{(\alpha\alpha'), \, (\beta\beta')}
    &= \sum_{i=1}^d \left( \lambda_\alpha \Gamma^i_{\alpha\beta} \right) \left( \lambda_{\alpha'} \Gamma^i_{\alpha'\beta'} \right)^*,
  \end{aligned}
\end{equation}
此时正则形式要求 $R$、$L$ 的特征向量均为单位矩阵(在张量变形的意义下),且对应的主特征值\footnote{即绝对值最大的特征值。它对应的特征向量也称主特征向量。} $\eta$ 相同。将一般的 $\Gamma$ 和 $\lambda$(例如随机初始化的)进行正则化的方法为:

\begin{enumerate}
  \item 计算 $R$、$L$ 的主特征向量,并变形为矩阵 $V_R$、$V_L$\footnote{在实际计算中,$V_R$、$V_L$ 可能会带有冗余的相位,这会影响接下来的矩阵分解操作。此相位可通过规定 $V_R$、$V_L$ 的迹为实数来移除。},对应的特征值均为 $\eta$。将 $V_R$、$V_L$ 进一步分解为 $X$、$Y$,使其满足
    \begin{equation}
      V_R = X X^\dagger, \quad V_L = Y^\dagger Y.
    \end{equation}
    这里可以用特征值分解,也可以用 Cholesky 分解。

  \item 利用规范自由度在 $\lambda$ 两侧分别插入 $I=(Y^\trans)^{-1}Y^\trans$ 和 $I=XX^{-1}$,并对得到的 $Y^\trans\lambda X$ 进行奇异值分解:
    \begin{equation}
      Y^\trans \lambda X = U \lambda V^\dagger.
    \end{equation}

  \item 将新得到的张量重排为 $\Gamma'$
    \begin{equation}
      \Gamma' = V^\dagger X^{-1} \Gamma \left( Y^\trans \right)^{-1} U.
    \end{equation}
\end{enumerate}

可以证明通过以上步骤得到的 $\Gamma'$ 和 $\lambda'$ 的确是 iMPS 的正则形式。

\begin{figure}[htb]
  \centering
  \includegraphics[width=\textwidth]{images/temp/canonical-form.png}
  % \caption[]{}
  \caption{正则形式}
  \label{fig:mps-canonical-form}
\end{figure}

\subsection{基态的确定}

Hamilton 量的基态 $\Psi_0$ 可以通过变分法求得:
\begin{equation}
  \ket{\Psi_0} = \argmin_{\ket{\Psi}} \frac{\langle\Psi|H|\Psi\rangle}{\langle\Psi|\Psi\rangle}.
\end{equation}
即使 $\Psi$ 已经表示成了 MPS 的形式,一般来说对 $\Psi$ 整体进行优化也是无法进行的。

DMRG\cite{white1992density,white1993density,schollwock2005density,mcculloch2007density,schollwock2011density}

iDMRG\cite{mcculloch2008infinite}

\subsection{时间演化}

下面我们介绍\emph{无限时间演化块消减} (infinite time-evolving block decimation, iTEBD)\cite{vidal2007classical,orus2008infinite} 算法,它主要用来处理波函数的时间演化
\begin{equation}
  \ket{\Psi_t} = \ee^{-\ii Ht} \ket{\Psi_0}
\end{equation}
或虚时演化
\begin{equation}
  \ket{\Psi_\tau} = \ee^{-H\tau} \ket{\Psi_0},
\end{equation}
其核心在于通过 Suzuki--Trotter 分解\cite{sornborger1999higher}
\begin{equation}
  \ee^{-\tau(A+B)} = \ee^{-\tau A} \ee^{-\tau B} + \mathcal{O}(\tau^2)
\end{equation}
将演化算符表示成 MPO 的形式,这样就可以很方便地与 MPS 形式的波函数进行缩并。

对于时间演化或虚时演化算符而言,由于经过 Suzuki--Trotter 分解后它们都很接近幺正算符,不会破坏 iMPS 的正则形式。而一般的算符并不具备这一性质,所以需要额外进行正则化操作。一般的 iTEBD 算法如下:

\begin{enumerate}
  \item 取随机的 iMPS $\{\Gamma,\lambda\}$,并按照 \ref{subsec:mps-construction} 小节中介绍的方法将其正则化。

  \item 把 $\{\Gamma,\lambda\}$ 与 MPO 的张量单元进行缩并:
    \begin{equation}
      \tilde{\Gamma}_{j\tilde{\alpha}\tilde{\beta}} = \sum_{i=1}^d \Gamma_{i\alpha\beta} O_{ij\mu\nu}, \quad
      \tilde{\lambda}_{\tilde{\beta}} = \lambda_\beta.
    \end{equation}
    这里指标 $\tilde{\alpha}=(\alpha,\mu)$、$\tilde{\beta}=(\beta,\nu)$,因此得到的 iMPS 对应连接维数 $\tilde{\chi}=\kappa\chi$。

  \item 对 $\{\tilde{\Gamma},\tilde{\lambda}\}$ 进行正则化,得到 $\{\tilde{\Gamma}',\tilde{\lambda}'\}$。

  \item 利用奇异值分解对 $\{\tilde{\Gamma}',\tilde{\lambda}'\}$ 进行截断,即只保留前 $\chi$ 个奇异值,使得连接维数保持在 $\chi$。

  \item 重复步骤 2--4,直到 iMPS 收敛。
\end{enumerate}

\begin{figure}[htb]
  \centering
  \includegraphics[width=0.75\textwidth]{images/temp/itebd-evolution.png}
  % \caption[]{}
  \caption{iTEBD 算法}
  \label{fig:itebd-evolution}
\end{figure}

\subsection{配分函数的计算}
\label{subsec:partition-function}

一个不同于时间(虚时)演化的例子,是把二维经典格点模型的配分函数视为 iMPS 在转移矩阵作用下的演化:
\begin{equation}
  Z(\beta) = \lim_{p,q\to\infty} \omega^{pq},
\end{equation}
其中 $\omega$ 是 $W$ 矩阵的主特征值:
\begin{equation}
  \includegraphics[width=0.75\textwidth]{images/temp/itebd-w-matrices.png}
\end{equation}

此时我们还可以计算单点函数和两点(关联)函数的期望值:
\begin{equation}
  \begin{aligned}
    \langle f(\sigma^{\bm{r}}) \rangle
      &= \frac{1}{Z(\beta)} \sum_{\{\sigma\}} f(\sigma^{\bm{r}}) \, \ee^{-\beta H(\{\sigma\})}, \\
    \langle f(\sigma^{\bm{r}}) g(\sigma^{\bm{r}'}) \rangle
      &= \frac{1}{Z(\beta)} \sum_{\{\sigma\}} f(\sigma^{\bm{r}}) g(\sigma^{\bm{r}'}) \, \ee^{-\beta H(\{\sigma\})}.
  \end{aligned}
\end{equation}
主要思路是将原配分函数中的张量单元 $A$ 替换为包含 $f(s)$ 或 $g(s)$ 的 $B$ 或 $B'$,并用同样的方法进行缩并。如图~\ref{fig:expectation-value} 所示,由于 iMPS 已被正则化,我们只需处理一个较小的张量网络。

\begin{figure}[htb]
  \centering
  \includegraphics[height=6.5cm]{images/temp/itebd-one-point-function.png} \quad
  \includegraphics[height=6.5cm]{images/temp/itebd-two-point-function.png}
  \caption{单点函数和两点(关联)函数的计算}
  \label{fig:expectation-value}
\end{figure}

\subsection{推广}

PEPS

MERA, entanglement renormalization\cite{vidal2007entanglement,evenbly2009algorithms,konig2009exact,evenbly2014algorithms,evenbly2015tensor2}

\section{重整化算法}

接下来我们主要考察二维张量网络。与 \ref{subsec:partition-function} 小节相同,我们关注的一个核心问题仍然是格点模型配分函数的计算。类似于 Kadanoff 的实空间重整化群\cite{pathria2011statistical},张量重整化算法的主要思路是对张量网络进行粗粒近似,直到得到一个不动点张量。

\subsection{张量重整化群}

最基本的一种重整化算法称为\emph{张量重整化群} (tensor renormalization group, TRG)\cite{levin2007tensor}。其步骤为:

\begin{enumerate}
  \item 利用 SVD 对原始的张量单元 $A^{(0)}$ 进行分解:
    \begin{center}
      \includegraphics[width=0.75\linewidth]{images/temp/trg-factorizing.png} \\
      \includegraphics[width=\linewidth]{images/temp/trg-factorizing-svd.png}
    \end{center}
    在做 SVD 时需要对奇异值进行截断,即只保留 $\chi$ 个最大奇异值,这样可以保证 TRG 的计算开销始终在可控范围内。

  \item 把得到的三角形张量缩并为新的张量单元 $A^{(1)}$:
    \begin{center}
      \includegraphics[width=0.4\linewidth]{images/temp/trg-group.png}
    \end{center}
    此时的张量网络相当于旋转了 $45^\circ$,而总的张量数目减少了一半。

  \item 重复以上步骤,直至收敛到不动点张量。

  \item 对最终得到的不动点张量进行求迹操作即可得到配分函数 $Z$:
    \begin{equation}
      Z = \sum_{i,j} A^{(N)}_{ijij} = \raisebox{-2em}{\includegraphics[width=2.5cm]{images/temp/trg-double-trace.png}}
    \end{equation}
\end{enumerate}

在实际计算中直接对 $A^{(n)}$ 进行缩并会很快使数值溢出。一种常用的技术是在每一步中对 $A^{(n)}$ 进行归一化,即令其 Frobenius 范数
\begin{equation}
  \bigl| A^{(n)} \bigr|_{\mathrm{F}} = \left( \sum_{i,j,k,l} A^{(n)}_{ijkl} \right)^{1/2} = 1,
\end{equation}
并记录每步所得的范数。最终的配分函数就相当于不动点张量 $A^{(N)}$ 与这些范数的乘积。

\begin{figure}[htb]
  \centering
  \includegraphics[width=\textwidth]{images/temp/trg-ising.png}
  \caption{利用 TRG 算法计算二维 Ising 模型的配分函数、能量及热容。可以发现在临界点处 TRG 的计算结果与精确值有一定差异。}
  \label{fig:trg-ising}
\end{figure}

\subsection{张量网络重整化}

TRG 算法对于临界的格点模型效果并不好。这主要是由于临界系统中存在长程关联,纠缠熵
\begin{equation}
  S = -\tr(\rho\log\rho)
\end{equation}
会随着系统尺寸以对数级增长,然而通过 SVD 给出的张量分解并不能很好地保留这些信息。

对于 TRG 的一种改进算法称为\emph{张量网络重整化} (tensor network renormalization, TNR)\cite{evenbly2015tensor1,evenbly2017algorithms},它在粗粒近似的过程中引入了一组幺正变换
\begin{equation}
  u \colon \mathbb{V} \otimes \mathbb{V} \to \mathbb{V} \otimes \mathbb{V}, \quad
  u u^\dagger = I^{\otimes2}
\end{equation}
和投影算符
\begin{equation}
  v \colon \mathbb{V} \to \mathbb{V} \otimes \mathbb{V}, \quad
  v^\dagger v = I,
\end{equation}
其中 $\mathbb{V}=\mathbb{C}^\chi$ 是 $\chi$ 维复向量空间。这里的 $u$ 和 $v$ 分别称为\emph{解纠缠子} (disentangler) 和\emph{等距子} (isometry),它们的具体取值可以通过最小化截断误差
\begin{equation}
  \delta = \raisebox{-2em}{\includegraphics[width=6cm]{images/temp/trg-truncation-error.png}}
\end{equation}
来获得。通过角双线 (corner double line, CDL) 张量的方法可以证明\cite{evenbly2015tensor1},TNR 算法中的 $u$ 和 $v$ 可以消除短程纠缠的影响,使得最终获得的不动点张量的确是标度不变的。

\begin{figure}[htb]
  \centering
  \includegraphics[width=0.6\textwidth]{images/temp/tnr.png}
  \caption{TNR 算法}
  \label{fig:tnr}
\end{figure}

除了 TNR 之外,还有其他一些工作试图改进原始的 TRG 算法,如引入过滤操作以消除短程纠缠影响的\emph{张量纠缠过滤重整化} (tensor entanglement-filtering renormalization, TEFR) 算法\cite{gu2009tensor1}、基于高阶奇异值分解的\emph{高阶 TRG} (higher order TRG, HOTRG) 算法\cite{xie2012coarse}以及通过将小张量组合成环路并加以优化的\emph{环路 TNR} (loop TNR) 算法\cite{yang2017loop}等。这些方法相比 TRG 和 TNR 在精度与计算效率上各有优劣,需要根据具体问题加以权衡。

% CTMRG\cite{nishino1996corner,orus2012exploring}
% Differentiable\cite{liao2019differentiable,geng2022differentiable}

\section{具体实现}

本文后续介绍的算法主要使用 Python 语言实现,其中的张量运算则利用 NumPy\cite{harris2020array}、SciPy\cite{virtanen2020scipy} 编写。它们提供了高效的张量缩并、变形以及 SVD、特征值求解等算法,并且还能通过线性算符 (linear operator) 的方法处理较大规模的稀疏矩阵与张量。此外,在硬件支持的情况下,还可以借助 GPU 甚至 TPU\cite{ganahl2023density} 进一步加速计算过程。

近年来,人们使用多种语言编写了各类张量网络程序包,例如基于 MATLAB 的 NCON\cite{pfeifer2014ncon},基于 Python 的 TeNPy\cite{hauschild2018efficient}、TensorNetwork\cite{roberts2019tensornetwork} 以及基于 Julia 的 TensorOperations.jl\cite{jutho2023tensoroperations} 等。特别值得注意的是基于 C++ 和 Julia 实现的 ITensor\cite{fishman2022itensor} 包,它能够根据指标本身的性质自动完成张量缩并,无需手动选择求和指标。文献 \parencite{psarras2021landscape} 对这些程序包的功能和特点进行了总结和比较。

在处理较大的系统时,张量缩并往往会成为计算瓶颈。为此人们提出了一系列优化方案\cite{pfeifer2014faster,evenbly2014improving},以寻找最优的缩并路径。NCON、opt\_einsum\cite{daniel2018opteinsum} 等程序包中均给出了实现。

\section{本章小结}

\chapter{奇异关联子与全息张量网络}
\label{chap:strange-correlator}

在前两章中,我们分别介绍了拓扑序和张量网络的基本概念。作为二维拓扑序的重要模型,弦网模型的 Hamilton 量可以表示为一系列局域算符之和,因此用张量网络来表示其基态也成为了一种很自然的想法。而将弦网模型的基态波函数与某些特定的直积态做内积,可以直接得到相应临界格点模型的配分函数。这种称为\emph{奇异关联子} (strange correlator) 的构造,不仅使我们能够计算相应 CFT 的中心荷和能谱,还给出了一种重整化算符的构造方案,进而使我们得以将体 (bulk) 和边界 (boundary) 联系在一起,并以此来构建全息张量网络 (holographic tensor network)。

\section{弦网模型基态的张量网络表示}
\label{sec:string-net-peps}

\subsection{四面体对称性}
\label{subsec:tetrahedral-symmetry}

为了叙述与计算的方便,我们先来澄清一下 $F$ 符号的定义,并介绍其对称性。在式~\eqref{eq:f-move} 和 \eqref{eq:string-net-local-rules} 中分别给出了 $F$ 符号的两种定义:
\begin{equation}
  \begin{aligned}
       \tikzinput{category/f-symbol-1}
    &= \sum_y \, \bigl[ F^{abc}_d \bigr]_{xy} \tikzinput{category/f-symbol-2}, \\
       \tikzinput{category/f-symbol-3}
    &= \sum_m F^{ijm}_{kln} \tikzinput{category/f-symbol-4}.
  \end{aligned}
\end{equation}
容易知道 $[F^{ijk}_l]_{xy}=F^{jix}_{lky}$。在本文中我们主要使用第一种定义。另一方面,根据式~\eqref{eq:f-move},我们有
\begin{equation}
    \bigl[ F^{abc}_d \bigr]_{xy}
  = \frac{\tr \, \Biggl[
      \Biggl( \tikzinput{category/f-symbol-small-2} \Biggr)^\dagger \,
      \tikzinput{category/f-symbol-small-1}
    \Biggr]}{\tr \, \Biggl[
      \Biggl( \tikzinput{category/f-symbol-small-2} \Biggr)^\dagger \,
      \tikzinput{category/f-symbol-small-2}
    \Biggr]}
  = \frac{\tikzinput{category/f-symbol-trace-1}}{\tikzinput{category/f-symbol-trace-2}}.
\end{equation}
这里分母部分可以利用环路消除 \eqref{eq:loop-removal} 式化简:
\begin{equation}
  \tikzinput{category/f-symbol-trace-2} = \sqrt{d_a d_b d_c d_d},
\end{equation}
而分子部分则等价于一个正四面体,于是
\begin{equation}
    \bigl[ F^{abc}_d \bigr]_{xy}
  = \sqrt{d_x d_y} \, \begin{bmatrix} a & b & x \\ c & d & y \end{bmatrix}
  = \frac{1}{\sqrt{d_a d_b d_c d_d}} \, \Tetrahedron xbdyca.
  \label{eq:tetrahedra-symbol}
\end{equation}
式中
\begin{equation}
    \begin{bmatrix} a & b & c \\ i & j & k \end{bmatrix}
  = \frac{1}{\sqrt{d_a d_b d_c d_i d_j d_k}} \, \Tetrahedron cbjkia
\end{equation}
称为\emph{四面体符号} (tetrahedral symbol) 或 \emph{$6j$ 符号} ($6j$-symbol),它的对称性(以及相应 $F$ 符号的对称性)可以从四面体的几何性质中获得:
\begin{align}
     \begin{bmatrix} a & b & c \\ i & j & k \end{bmatrix}
  &= \begin{bmatrix} b & c & a \\ j & k & i \end{bmatrix}
   = \begin{bmatrix} c & a & b \\ k & i & j \end{bmatrix}, \\
     \begin{bmatrix} a & b & c \\ i & j & k \end{bmatrix}
  &= \begin{bmatrix} a & c & b \\ i & k & j \end{bmatrix}
   = \begin{bmatrix} c & b & a \\ k & j & i \end{bmatrix}
   = \begin{bmatrix} b & a & c \\ j & i & k \end{bmatrix}, \\
     \begin{bmatrix} a & b & c \\ i & j & k \end{bmatrix}
  &= \begin{bmatrix} a & j & k \\ i & b & c \end{bmatrix}
   = \begin{bmatrix} i & b & k \\ a & j & c \end{bmatrix}
   = \begin{bmatrix} i & j & c \\ a & b & k \end{bmatrix}.
\end{align}
以上三组等式分别对应了四面体对称群 $A_4$ 中绕顶角的旋转(列轮换)、镜像(交换两列)和绕对边连线的旋转(交换两行的部分)。注意整行的交换一般来说并不相等。这一性质为\emph{四面体对称性} (tetrahedral symmetry)\cite{aasen2020topological,fuchs2023tetrahedral}。利用 $F$ 移动和环路消除,还可以得到下面的关系式:
\begin{equation}
    \tikzinput{string-net/vertex-2}
  = \sqrt{d_\alpha d_\beta d_\gamma} \, \begin{bmatrix} a & b & c \\ \alpha & \beta & \gamma \end{bmatrix} \Vertex kij
  = \frac{1}{\sqrt{d_i d_j d_k}} \, \Tetrahedron ik\gamma\alpha\beta j \Vertex kij,
\end{equation}
这也被称为\emph{星—三角变换} (star-triangle transformation)。

\subsection{弦网模型的基态}
\label{subsec:string-net-ground-state}

接下来我们来给出弦网模型基态的张量网络表示\cite{gu2009tensor2,buerschaper2009explicit}。由于弦网模型的 Hamilton 量
\begin{equation}
  H = -\sum_v A_v - \sum_p B_p
\end{equation}
是严格可解的,即 $H$ 可以表示成一系列相互对易的投影算符之和,因而它的基态可以通过这些投影算符的本征子空间给出。式中,电荷算符 $A_v$ 指定了每个顶点处的融合规则,磁通算符 $B_p$ 是 $B_p^s$ 按量子维数的线性组合,而 $B_p^s$ 则可以图形化地视为能在方块 $p$ 处生成一个孤立环路的算符(具体定义见 \ref{subsec:string-net-hamiltonian} 小节)。

弦网模型的基态波函数 $\ket{\psi_0}$ 可以通过在真空态 $\ket{\varnothing}$ 上作用全部的 $B_p$ 算符得到。如上所言,每个 $B_p$ 算符都是一个方块内部由对象 $R$ 构成的环路,而
\begin{equation}
  \tikzinput{string-net/string-1}
  \enspace = \sum_i \frac{d_i}{D^2}
  \tikzinput{string-net/string-2}
\end{equation}
则是任意子按量子维数的加权叠加。因此有
\begin{equation}
    \ket{\psi_0} = \prod_p B_p \ket{\varnothing}
  = \prod_p B_p \, \ket[\Bigg]{\, \tikzinput{string-net/loop-1} \,}
  = \ket[\Bigg]{\, \tikzinput{string-net/loop-2} \,}.
\end{equation}

为了给出 $\ket{\psi_0}$ 一种\emph{对称化}的张量网络表示,我们必须从一开始就同等对待六边形的每个边以保持对称性。需要指出的是,文献 \parencite{buerschaper2009explicit} 中通过引入奇偶子网格来构造张量网络的手段不能很好地保留旋转对称性。这里,我们需将权重(量子维数)$d_i$ 平均分配到 $R$ 环路的六个边上,使得它们在互相连接时也有良好定义:
\begin{equation}
  \begin{aligned}
       \VirutalHexagon{\draw [very thick, draw=MaterialRed] (-30:0.8) -- (30:0.8)} \enspace
    &= \sum_i \left( \frac{d_i}{D^2} \right)^{1/6} \enspace
       \VirutalHexagon{\draw [thick] (-30:0.8) node [left] {$i$} -- (30:0.8)} \, , \\
       \VirutalHexagon{\draw [very thick, draw=MaterialRed] (-30:0.8) -- (30:0.8) -- (90:0.8)} \enspace
    &= \sum_i \left( \frac{d_i}{D^2} \right)^{1/3} \enspace
       \VirutalHexagon{\draw [thick] (-30:0.8) node [left] {$i$} -- (30:0.8) -- (90:0.8)} \, , \\
       \VirutalHexagon{\draw [very thick, draw=MaterialRed] (0,0) circle [radius=0.7]} \enspace
    &= \sum_i \frac{d_i}{D^2} \enspace
       \VirutalHexagon{\draw [thick] (0,0) circle [radius=0.7] node [right=-0.2em] {$i$}} \, .
  \end{aligned}
\end{equation}
对于邻接的六边形,我们可以在两个相邻的 $R$ 环路间执行 $F$ 移动:
\begin{align}
     \tikzinput{string-net/adjacent-loop-1} \enspace
  &= \sum_{i,j} \biggl( \frac{d_i d_j}{D^4} \biggr)^{1/6} \enspace
     \tikzinput{string-net/adjacent-loop-2} \notag \\
  &= \sum_{i,j} \biggl( \frac{d_i d_j}{D^4} \biggr)^{1/6} \sum_k \sqrt{\frac{d_k}{d_i d_j}} \enspace
     \tikzinput{string-net/f-move-1} \notag \\
  &= \sum_{i,j,k} \bigl( D^2 d_i d_j \bigr)^{-1/6} d_k^{1/4} \cdot \bigl( D^2 d_i d_j \bigr)^{-1/6} d_k^{1/4} \enspace
     \tikzinput{string-net/f-move-1} \,
   = \tikzinput{string-net/f-move-2} \, .
\end{align}
式中,蓝线和紫线表示各任意子按照对应量子维数的加权叠加,并且分别带有 $(D^2 d_i d_j)^{-1/6}$ 和 $d_k^{1/4}$ 的因子。在对所有的 $R$ 环路执行 $F$ 移动之后,$\ket{\psi_0}$ 现在可以表示为
\begin{equation}
  \ket{\psi_0} = \ket[\Bigg]{\, \tikzinput{string-net/vertices}}.
\end{equation}
它可以通过对位于顶点处的局域构建块进行缩并得到,而这些构建块可以用任意子基 (anyon basis) 表示为
\begin{align}
     \tikzinput{string-net/vertex-1} \enspace
  &= \sum_{i,j,k} \sum_{\alpha,\beta,\gamma} D^{-1} (d_i d_j d_k)^{1/4} (d_\alpha d_\beta d_\gamma)^{-1/3} \,
     \tikzinput{string-net/vertex-2} \notag \\
  &= \sum_{i,j,k} \sum_{\alpha,\beta,\gamma} D^{-1} (d_i d_j d_k)^{-1/4} (d_\alpha d_\beta d_\gamma)^{-1/3}
     \Tetrahedron ik\gamma\alpha\beta j \Vertex kij.
\end{align}
这样我们就获得了六边形网格中三角形张量的对称形式:
\begin{align}
  \Triangle jki\alpha\beta\gamma
  &= D^{-1} (d_i d_j d_k)^{-1/4} (d_\alpha d_\beta d_\gamma)^{-1/3} \,
    \Tetrahedron ik\gamma\alpha\beta j \notag \\
  &= D^{-1} (d_\alpha d_\beta d_\gamma)^{1/6} (d_i d_j d_k)^{-1/4} (d_\alpha d_\beta d_\gamma)^{-1/2} \,
    \Tetrahedron ik\gamma\alpha\beta j.
  \label{eq:unit-of-string-net-honeycomb}
\end{align}
对于更一般的情形,因子 $1/6$ 需要根据闭合环路中每条边的贡献进行修正。在任意三价图(每个顶点与三条边相连)中,归一化的三角形张量可以写成
\begin{align}
  \Triangle jki\alpha\beta\gamma
  &= D^{-2 (1/n_\alpha + 1/n_\beta + 1/n_\gamma)}
    \bigl( d_\alpha^{1/n_\alpha} d_\beta^{1/n_\beta} d_\gamma^{1/n_\gamma} \bigr) \notag \\
  &\qquad \cdot (d_i d_j d_k)^{-1/4} (d_\alpha d_\beta d_\gamma)^{-1/2} \,
    \Tetrahedron ik\gamma\alpha\beta j,
  \label{eq:unit-of-string-net-general}
\end{align}
这些三角形张量每条边带有三个指标,外面的两个是\emph{虚拟指标}或\emph{辅助指标},它们会在平面内互相缩并掉;中间的一个则是\emph{物理指标},它们会伸出平面外,并且不会被缩并掉。可以看出,这实际上也是一种 PEPS 张量网络(见 \ref{subsec:mps-generalization} 小节)。

\begin{figure}[htb]
  \centering
  \includegraphics[width=0.6\textwidth]{images/string-net-peps.pdf}
  \caption[弦网模型基态的 PEPS 张量网络表示]{弦网模型基态的 PEPS 张量网络表示。图片来源:\parencite{buerschaper2009explicit}。}
  \label{fig:string-net-peps}
\end{figure}

\section{奇异关联子}
\label{sec:strange-correlator}

\emph{奇异关联子} (strange correlator) 最早用来在对称保护拓扑序 (symmetry protected topological order, SPT)\footnote{尽管名称中带有“拓扑序”,但 SPT 态中只有短程纠缠,这与本文所说的拓扑序(具有长程纠缠,见 \ref{sec:topological-order} 节)实际上是不同的。} 中寻找不同的相\cite{you2014wave},之后则被推广到了弦网模型中\cite{vanhove2018mapping,lootens2019cardy,vanhove2022topological}。通过将 2+1 维弦网模型的 PEPS 基态波函数 $\ket{\Psi_\text{SN}}$ 与某些特定的直积态 $\bra{\Omega}$ 做内积,可以获得二维临界格点模型的配分函数
\begin{equation}
  Z = \langle\Omega|\Psi_\text{SN}\rangle.
\end{equation}
在热力学极限下,配分函数 $Z$ 可由对应的共形场论 (CFT) 来描述。如图~\ref{fig:peps-strange-correlator} 所示,对于由三角形张量单元构成的 PEPS 张量网络,直积态 $\bra{\Omega}$ 相当于为其提供了特定的边界条件,即把物理指标固定到某些值上,而角落上的自由度则需求和。

\begin{figure}[htb]
  \centering
  \tikzinput{strange-correlator}
  \caption[奇异关联子的构造]{奇异关联子的构造。其中绿线表示投影到特定值(即与某直积态做内积)之后的物理指标,灰线表示需要求和的辅助指标。整个张量网络完全缩并后即为对应的配分函数。}
  \label{fig:peps-strange-correlator}
\end{figure}

\subsection{例子:Fibonacci 模型}
\label{subsec:strange-correlator-fib}

接下来,我们利用 \ref{subsec:fusion-category-examples} 小节中所给出的 Fibonacci 和 Ising 两种融合范畴的数据来构造相应的奇异关联子。Fibonacci 弦网模型定义在六边形网格上,它包含 $\1$ 和 $\tau$ 两种简单对象(任意子),量子维数分别为 $d_{\1}=1$、$d_\tau=\varphi$,其中 $\varphi=(1+\sqrt5)/2$ 是黄金比。融合规则为
\begin{equation}
  \1 \times \1 = \1, \quad
  \1 \times \tau = \tau \times \1 = \tau, \quad
  \tau \times \tau = \1 + \tau,
\end{equation}
$F$ 符号为
\begin{equation}
  [F^{\tau\tau\tau}_\tau]_{ij} = \dfrac1\varphi \begin{pmatrix} 1 & \sqrt\varphi \\ \sqrt\varphi & -1 \end{pmatrix}, \quad
  i,j \in \{\1, \tau\}.
\end{equation}
唯一非平凡的四面体以及对应的三角形张量为
\begin{align}
     \Tetrahedron \tau mn\tau\tau\tau
  &= \sqrt{d_\tau d_\tau d_\tau d_\tau} \bigl[ F^{\tau\tau\tau}_\tau \bigr]_{mn}
   = \varphi^2 \bigl[ F^{\tau\tau\tau}_\tau \bigr]_{mn}, \displaybreak[0] \\
     \Triangle \tau\tau mn\tau\tau
  &= (d_\tau d_\tau d_m)^{-\frac14} (d_\tau d_\tau d_n)^{-\frac13} \Tetrahedron \tau mn\tau\tau\tau \\
  &= \varphi^{\frac56} d_m^{-\frac14} d_n^{-\frac13} \bigl[ F^{\tau\tau\tau}_\tau \bigr]_{mn}, \enspace
     m,n \in \{\1,\tau\}.
\end{align}
在我们的奇异关联子构造中,所有的物理指标都会被选取为 $\tau$,因而这时将得到两种三角形张量单元:
\begin{equation}
    \Triangle \tau\tau\tau\tau\tau\tau
  = \varphi^{\frac14} \bigl[ F^{\tau\tau\tau}_\tau \bigr]_{\tau\tau} = -\varphi^{-\frac34}, \quad
    \Triangle \tau\tau\tau\1\tau\tau
  = \varphi^{\frac{7}{12}} \bigl[ F^{\tau\tau\tau}_\tau \bigr]_{\tau\1} = \varphi^{\frac{1}{12}}.
\end{equation}

\subsection{例子:Ising 模型}

为了与经典二维 Ising 模型相对应,我们考虑一个定义在正方形和八边形交错的网格上、由 $\1$、$\sigma$ 和 $\psi$ 三种简单对象构成的 Ising 弦网模型。各任意子对应的量子维数分别为 $d_{\1}=d_\psi=1$、$d_\sigma=\sqrt2$,融合规则为
\begin{equation}
  \begin{gathered}
  \1 \times \1 = \psi \times \psi = \1, \quad
  \1 \times \sigma = \sigma \times \1 = \sigma \times \psi = \psi \times \sigma = \sigma, \\
  \1 \times \psi = \psi \times \1 = \psi, \quad
  \sigma \times \sigma = \1 + \psi,
  \end{gathered}
\end{equation}
$F$ 符号为
\begin{equation}
  [F^{\psi\sigma\psi}_\sigma]_{\sigma\sigma} = [F^{\sigma\psi\sigma}_\psi]_{\sigma\sigma} = -1, \quad
  [F^{\sigma\sigma\sigma}_\sigma]_{ij} = -\dfrac{1}{\sqrt2} \begin{pmatrix} 1 & 1 \\ 1 & -1 \end{pmatrix}, \quad
  i,j \in \{\1, \psi\}.
\end{equation}
于是四面体和三角形张量为:
\begin{align}
     \Triangle \sigma\sigma\1\1\sigma\sigma
  &= (d_{\1} d_\sigma d_\sigma)  ^{-\frac14} (d_{\1} d_\sigma d_\sigma)  ^{-\frac13}
     \Tetrahedron \1\sigma\sigma\1\sigma\sigma
     = 2^{-\frac{1}{12}}, \notag \displaybreak[0] \\
     \Triangle \sigma\sigma\1\psi\sigma\sigma
  &= (d_{\1} d_\sigma d_\sigma)  ^{-\frac14} (d_\sigma d_\sigma d_\psi)^{-\frac13}
     \Tetrahedron \1\sigma\sigma\psi\sigma\sigma
     = 2^{-\frac{1}{12}}, \notag \displaybreak[0] \\
     \Triangle \sigma\sigma\psi\1\sigma\sigma
  &= (d_\sigma d_\sigma d_\psi)^{-\frac14} (d_{\1} d_\sigma d_\sigma)  ^{-\frac13}
     \Tetrahedron \psi\sigma\sigma\1\sigma\sigma
     = 2^{-\frac{1}{12}}, \notag \displaybreak[0] \\
     \Triangle \sigma\sigma\psi\psi\sigma\sigma
  &= (d_\sigma d_\sigma d_\psi)^{-\frac14} (d_\sigma d_\sigma d_\psi)^{-\frac13}
     \Tetrahedron \psi\sigma\sigma\psi\sigma\sigma
     = -2^{-\frac{1}{12}}.
\end{align}
在 Ising 模型的奇异关联子构造中,我们需要将物理指标投影到直积态
\begin{equation}
  \bra{\omega(\beta)} = \sqrt2 \, \bigl( \cosh\beta \bra{\1} + \sinh\beta \bra{\psi} \bigr)
\end{equation}
上面。为了得到正方形网格的结构,我们用 8 个三角形组成一个带有 4 个指标的八边形张量单元:
\begin{equation}
  A_{ijkl} = \tikzinput{ising/octagon} \quad \text{where} \quad
  i, j, k, l = \1 \text{ or } \psi, \quad
  \tikzinput{ising/line-sigma} \, = \sigma, \quad
  \tikzinput{ising/line-omega} \, = \omega.
  \label{eq:ising-octagon}
\end{equation}
注意此时物理指标已被求和,而辅助指标 $i$、$j$、$k$、$l$ 得以保留。为了得到 $A_{ijkl}$ 的值,我们先计算每一对三角形:
\begin{equation}
  \tikzinput{ising/triangles-1} = \tikzinput{ising/triangles-2} =
  \begin{cases}
    2^{\frac13} \bigl( \cosh\beta + \sinh\beta \bigr) = 2^{\frac13} \ee^\beta,
      & (i,j) = (\1,\1) \text{ or } (\psi, \psi); \\
    2^{\frac13} \bigl( \cosh\beta - \sinh\beta \bigr) = 2^{\frac13} \ee^{-\beta},
      & (i,j) = (\1,\psi) \text{ or } (\1, \psi).
  \end{cases}
\end{equation}
于是
\begin{equation}
  A_{\1\1\1\1} = A_{\psi\psi\psi\psi} = 2^{\frac43} \ee^{4\beta}, \quad
  A_{\1\psi\1\psi} = A_{\psi\1\psi\1} = 2^{\frac43} \ee^{-4\beta}, \quad
  \text{others} = 2^{\frac43}.
  \label{eq:ising-tensor-from-sc}
\end{equation}
这和围绕一个方块的 Ising 相互作用(4 个自旋)的张量表示[式~\eqref{eq:ising-tensor}]是一致的,只相差一个整体系数:
\begin{equation}
  A_{ijkl}
    := A_{\sigma_i \sigma_j \sigma_k \sigma_l}
    = \ee^{\beta(\sigma_i\sigma_j + \sigma_j\sigma_k + \sigma_k\sigma_l + \sigma_l\sigma_i)}.
  \label{eq:ising-tensor}
\end{equation}
把这些八边形张量移动 $1/2$ 格可以获得对偶网格的张量表示
\begin{equation}
  \tilde{A}_{ijkl} = \, \tikzinput{ising/dual-octagon} \quad \text{where} \quad
  i, j, k, l = \1 \text{ or } \psi.
\end{equation}
% TODO: dual lattice
% 此时
% \begin{equation}
%   \tikzinput{ising/dual-triangles-1} = \tikzinput{ising/dual-triangles-2} =
%   \begin{cases}
%     2^{\frac13} \bigl( \cosh\beta + \sinh\beta \bigr) = 2^{\frac13} \ee^\beta,
%       & (i,j) = (\1,\1) \text{ or } (\psi, \psi); \\
%     2^{\frac13} \bigl( \cosh\beta - \sinh\beta \bigr) = 2^{\frac13} \ee^{-\beta},
%       & (i,j) = (\1,\psi) \text{ or } (\1, \psi).
%   \end{cases}
% \end{equation}
利用 Kramers--Wannier 对偶,即令 $A=\tilde{A}$,就可以得到临界温度
\begin{equation}
  \ee^{2\beta_{\text{c}}} = \sqrt{2} + 1 \implies \beta_{\text{c}} = \frac12 \log(1 + \sqrt{2}).
\end{equation}

\subsection{中心荷的计算}
\label{subsec:central-charge}

得到奇异关联子后,我们可以用标准的张量网络方法对其进行分析。例如,我们可以利用在 \ref{subsec:mps-time-evolution} 小节中介绍的 iTEBD 算法来计算中心荷,其基本思路是:

\begin{enumerate}
  \item 利用奇异关联子构造正方形张量单元 $A_{ijkl}$。在 Ising 的情况中,我们已经得到了这样的表示[式~\eqref{eq:ising-octagon}];而对于 Fibonacci 模型,我们需要把两个三角张量形拼合成一个正方形:
    \begin{equation}
      A_{ijkl} = \tikzinput{fibonacci/square-1} = \tikzinput{fibonacci/square-2}.
    \end{equation}
  \item 按照 iTEBD 算法,用 $A_{ijkl}$ 作为 iMPO 对随机的初始态进行演化,得到正则化的不动点 iMPS $\{\Gamma,\lambda\}$。
  \item 注意 $\lambda$ 是一个对角矩阵,其对角元可用来计算关联长度 $\xi$ 和 von Neumann 纠缠熵 $S_A$:
    \begin{equation}
      \xi = -\frac{1}{\log|\lambda_2 / \lambda_1|}, \quad
      S_A = \sum_i \lambda_i^2 \log \lambda_i^2
    \end{equation}
  \item 利用关联长度与纠缠熵之间满足的关系式
    \begin{equation}
      S_A \sim \frac{c}{6} \log \xi,
    \end{equation}
    可以对使用不同的连接维数 $\chi$ 所获得的数据进行拟合,即可得到中心荷 $c$。
\end{enumerate}

我们使用连接维数 $\chi\in\{8,9,\dots,27\}$ 来进行计算,拟合结果如图~\ref{fig:central-charge} 所示。得到的中心荷为
\begin{equation}
  c_{\text{Ising}} \simeq 0.499 \pm 0.004, \quad
  c_{\text{Fib}}   \simeq 0.792 \pm 0.004.
\end{equation}
这与对应 CFT 给出的理论值 $1/2$ 和 $4/5$ 是非常接近的,说明通过奇异关联子的确可以从弦网模型基态波函数出发得到正确的 CFT 不动点张量。

\begin{figure}[htb]
  % TODO: update image
  \centering
  \subcaptionbox{Ising 模型}{%
    \includegraphics[width=0.45\textwidth]{images/ising-central-charge.png}} \quad
  \subcaptionbox{Fibonacci 模型}{%
    \includegraphics[width=0.45\textwidth]{images/fib-central-charge.png}}
  \caption[中心荷拟合结果]{中心荷拟合结果。横坐标:关联长度的对数 $\log\xi$,纵坐标:纠缠熵 $S_A$。使用的连接维数为 $\chi\in\{8,9,\dots,27\}$。}
  \label{fig:central-charge}
\end{figure}

% \subsection{高度模型}
% TODO: \emph{高度模型} (height model)

\section{MPO 对称性}

在弦网模型的 PEPS 张量网络表示中,弦激发可以理解为由满足\emph{推拉条件} (pulling-through condition) 的矩阵乘积算符 (MPO) 产生。这些 MPO 具有一定的代数结构,任意子则可通过在 MPO 两端放置缺陷张量来产生。MPO 的对称性给出了另外的 $C^*$ 代数结构,其\emph{中心幂等元} (central idempotent) 也对应了输出范畴的不同\emph{拓扑分区} (topological sector),即不同的任意子类型\cite{bultinck2017anyons,williamson2017symmetry,lootens2019cardy,aasen2020topological,sahinoglu2021characterizing}。

\subsection{推拉条件}

弦网模型中的 MPO 张量单元为\footnote{不同文献中张量的具体选取会略有不同,这里与 \ref{sec:string-net-peps} 节相比差了一些整体的系数,但并不影响其结论。式~\eqref{eq:string-net-mpo}--\eqref{eq:tube-algebra-hermitian-conjugate} 中图片来源:\parencite{williamson2017symmetry}。}:
\begin{equation}
    \vcenter{\hbox{\includegraphics{images/mpo/TCMPO.pdf}}}
  = \delta_{\alpha\alpha'} \delta_{\beta\beta'} \delta_{\mu\mu'} \delta_{\nu\nu'} \delta_{ii'} \delta_{aa'}
    \frac{F^{a\mu i}_{\beta\alpha\nu}}{\sqrt{d_\alpha d_\nu}}.
  \label{eq:string-net-mpo}
\end{equation}
张量的指标都是融合范畴 $\mathcal{C}$ 中的简单对象。由此组成的 MPO 则为
\begin{equation}
    \mathrm{MPO}_a = \, \vcenter{\hbox{\includegraphics[scale=1.5]{images/mpo/smallTCmpo.pdf}}} \, ,
\end{equation}
其中虚线表示周期性边界条件,而
\begin{equation}
    \vcenter{\hbox{\includegraphics[scale=1.2]{images/mpo/TCMPOclosure.pdf}}}
  = \delta_{\alpha\alpha'} \delta_{\beta\beta'} \delta_{bb'} \delta_{ab}.
\end{equation}
它们之间满足融合规则:
\begin{equation}
  \mathrm{MPO}_a \, \mathrm{MPO}_b = \sum_c N_{ab}^c \, \mathrm{MPO}_c,
\end{equation}
其中 $N_{ab}^c$ 和输入范畴所给出的数据是一致的。通过插入融合张量
\begin{equation}
    \vcenter{\hbox{\includegraphics{images/mpo/TCMPOfusion.pdf}}}
  = \delta_{\alpha\alpha'} \delta_{\beta\beta'} \delta_{\gamma\gamma'}
    \left( \frac{d_a d_b}{d_c} \right)^{1/4}
    \frac{F^{ab\gamma}_{\alpha c\beta}}{\sqrt{d_\beta}},
\end{equation}
融合规则也可以表述为
\begin{equation}
    \vcenter{\hbox{\includegraphics[scale=1.2]{images/mpo/TCMPOmultLHS.pdf}}}
  = \sum_c \delta^c_{ab} \sqrt{\frac{d_c}{d_a d_b}}
    \vcenter{\hbox{\includegraphics[scale=1.2]{images/mpo/TCMPOmultRHS.pdf}}}.
\end{equation}
不同的缩并顺序对结果的影响由 $F$ 符号给出:
\begin{equation}
    \vcenter{\hbox{\includegraphics{images/mpo/TCFsymbolLHS.pdf}}} \enspace
  = \sum_f F^{abc}_{def} \enspace
    \vcenter{\hbox{\includegraphics{images/mpo/TCFsymbolRHS.pdf}}}.
\end{equation}

弦网模型的 PEPS 张量网络表示是 \emph{MPO 单射} (MPO-injective) 的,它满足\emph{推拉条件} (pulling-through condition):
\begin{equation}
  \vcenter{\hbox{\includegraphics[scale=1.2]{images/mpo/smallTCPullthruLHS.pdf}}}
  \enspace = \enspace
  \vcenter{\hbox{\includegraphics[scale=1.2]{images/mpo/smallTCPullthruRHS.pdf}}},
\end{equation}
即 MPO 可以自由地穿过 PEPS 张量单元。注意到 MPO 和 PEPS 的张量单元可通过同一组 $F$ 符号来确定,因此推拉条件实际上是融合范畴中五边形方程(图~\ref{fig:f-symbols-pentagon-equation})的推论。此时,PEPS 张量网络可以视为背景,而其中的 $\mathrm{MPO}_\psi$ 则表示对应的弦算符(见图~\ref{fig:peps-mpo})。

\begin{figure}[htb]
  \[
    \vcenter{\hbox{\includegraphics[scale=1.2]{images/mpo/TCpsiPEPS.pdf}}}
    \quad \to \quad
    \vcenter{\hbox{\includegraphics[scale=1.2]{images/mpo/TCpsibackground.pdf}}}
  \]
  \caption[PEPS 张量网络与其中的 MPO]{PEPS 张量网络与其中的 MPO。图片来源:\parencite{williamson2017symmetry}。}
  \label{fig:peps-mpo}
\end{figure}

% TODO: MPO and topological properties

\subsection{管状代数与中心幂等元}
\label{subsec:tube-algebra-idempotents}

考虑一个圆柱上的 PEPS 张量网络,对于其上的弦算符 (MPO),我们可以取下面的基来描述:
\begin{equation}
    \mathcal{T}^{s}_{pqr}
  = \, \vcenter{\hbox{\includegraphics{images/mpo/Tube0.pdf}}} \,
  = \vcenter{\hbox{\includegraphics{images/mpo/TCtensortube.pdf}}}.
\end{equation}
它们的乘法和 Hermitian 共轭分别为
\begin{align}
     \mathcal{T}^{s}_{pqr} \mathcal{T}^{s'}_{p'q'r'}
  &= \delta_{rp'} \, \vcenter{\hbox{\includegraphics{images/mpo/TCTubemult.pdf}}} \notag \\
  &= \delta_{rp'} \sum_{q''s''} F^{sqs'}_{q'rq''} F^{q''s's}_{pqs''} F^{s'sq''}_{r'q's''}
     \sqrt{\frac{d_s d_{s'}}{d_{s''}}} \mathcal{T}^{s''}_{pq''r'}, \\
     \bigl( \mathcal{T}^{s}_{pqr} \bigr)^\dagger
  &= \, \vcenter{\hbox{\includegraphics{images/mpo/TCTubedagger.pdf}}} \,
   = \sqrt{\frac{d_p}{d_r}} \sum_{q'} F^{srs}_{pqq'} \mathcal{T}^{s}_{rq'p}.
  \label{eq:tube-algebra-hermitian-conjugate}
\end{align}
由此可见,这些基关于乘法和 Hermitian 共轭都是封闭的,因而它们构成了一个 $C^*$ 代数,称为 \emph{Ocneanu 管状代数} (Ocneanu's tube algebra)\cite{evans1995ocneanu,evans1998quantum}。

管状代数可以被分块对角化,其中的每一块上的投影算符对应一个\emph{不可约中心幂等元} (irreducible central idempotent)。从范畴论的角度来看,这些中心幂等元构成了一个新的范畴,它对应了输入范畴 $\mathcal{C}$ 的 \emph{Drinfel'd 中心} (Drinfel'd center) $\mathcal{Z}(\mathcal{C})$。而从拓扑序的角度来看,它们则对应了不同的\emph{拓扑分区} (topological sectors)\cite{lan2014topological,bultinck2017anyons,williamson2017symmetry,vanhove2018mapping}。

中心幂等元可以表示为管状代数中基的线性组合:
\begin{equation}
  \mathcal{P}_{i} = \frac{1}{D^2} \sum_{pqrs} t^i_{pqrs} \mathcal{T}^{s}_{pqr},
\end{equation}
其中 $D$ 为总量子维数,而系数 $t^i_{pqrs}$ 可根据 $\mathcal{P}_{i}$ 满足的约束条件来求解:
\begin{equation}
  \mathcal{P}_{i} \mathcal{P}_{j} = \delta_{ij} \mathcal{P}_{i}, \quad
  \mathcal{P}_{i}^\dagger = \mathcal{P}_{i}, \quad
  \mathcal{T}^{s}_{pqr} \mathcal{P}_{i} = \mathcal{P}_{i} \mathcal{T}^{s}_{pqr}.
\end{equation}
以包含 $\1$、$\tau$ 两种任意子的 Fibonacci 模型为例\cite{bultinck2017anyons},它的中心幂等元共有 4 个:
\begin{equation}
  \begin{aligned}
    \mathcal{P}_{1} &= \frac{1}{\sqrt5} \left(
      \frac{1}{\varphi} \mathcal{T}^{\1}_{\1\1\1} + \sqrt{\varphi} \mathcal{T}^{\tau}_{\1\tau\1}
    \right), \\
    \mathcal{P}_{2} &= \frac{1}{\sqrt5} \left(
      \frac{1}{\varphi} \mathcal{T}^{\1}_{\tau\tau\tau} + \frac{1}{\sqrt{\varphi}} \ee^{-4\pi\ii/5} \mathcal{T}^{\tau}_{\tau\1\tau} + \ee^{ 3\pi\ii/5} \mathcal{T}^{\tau}_{\tau\tau\tau}
    \right), \\
    \mathcal{P}_{3} &= \frac{1}{\sqrt5} \left(
      \frac{1}{\varphi} \mathcal{T}^{\1}_{\tau\tau\tau} + \frac{1}{\sqrt{\varphi}} \ee^{ 4\pi\ii/5} \mathcal{T}^{\tau}_{\tau\1\tau} + \ee^{-3\pi\ii/5} \mathcal{T}^{\tau}_{\tau\tau\tau}
    \right), \\
    \mathcal{P}_{4} &= \frac{1}{\sqrt5} \left(
      \varphi \mathcal{T}^{\1}_{\1\1\1} + \mathcal{T}^{\1}_{\tau\tau\tau} - \sqrt{\varphi} \mathcal{T}^{\tau}_{\1\tau\1} + \sqrt{\varphi} \mathcal{T}^{\tau}_{\tau\1\tau} + \frac{1}{\varphi} \mathcal{T}^{\tau}_{\tau\tau\tau}
    \right),
  \end{aligned}
  \label{eq:fib-idempotents}
\end{equation}
分别对应了 doubled Fibonacci(即 Fibonacci 的 Drinfel'd 中心)中的 4 种任意子:
\begin{equation}
  \mathcal{P}_{1} \mapsto (\1, \1), \quad
  \mathcal{P}_{2} \mapsto (\1, \bar{\tau}), \quad
  \mathcal{P}_{3} \mapsto (\tau, \1), \quad
  \mathcal{P}_{4} \mapsto (\tau, \bar{\tau}),
\end{equation}
相应的拓扑自旋分别为
\begin{equation}
  h_1 = 0, \quad h_2 = -\frac45, \quad h_3 = \frac45, \quad h_4 = 0.
\end{equation}

\section{全息张量网络}
\label{sec:holographic-tensor-network}

% 首先,我们将奇异关联子的概念推广到了任意维度($1\leqslant D\leqslant 3$),并且给出了一种系统的方法用以构造配分函数和搜寻临界点。我们的方法基于对 $\bra{\Omega}$ 的理解。根据融合范畴 $\mathcal{C}$ 及其对应的范畴对称性 $\mathcal{Z}(\mathcal{C})$,我们可以构造一套重整化算符,并将其作为全息张量网络。它可以对拓扑序的基态波函数 $\ket{\Psi}$ 进行粗粒近似。为了使 $\langle\Omega|\Psi\rangle$ 能够表示对应共形场论的配分函数,$\bra{\Omega}$ 需要是重整化算符的本征态。实际上,带有范畴对称性 $\mathcal{Z}(\mathcal{C})$ 的 $D$ 维共形场论,也就相当于由 $D+1$ 维拓扑序所确定的重整化算符的本征态。我们发现,对于中心为 $\mathcal{Z}(\mathcal{C})$ 的输入范畴 $\mathcal{C}$,其中的每一个 \emph{Frobenius 代数} (Frobenius algebra) 都确定了一个 $D+1$ 维的格点拓扑模型,它们对应了重整化算符的本征态,能给出一个 $D$ 维的对称 TQFT,而 CFT 就位于这些 TQFT 的相变点处。我们可以在相变点两侧各取一个 Frobenius 代数,而描述 CFT 的边界条件即可通过这两个 Frobenius 代数对应的边界条件插值得到。而在远离 CFT 不动点处,重复作用这些重整化算符将会给出 $\mathcal{C}$ 的 Frobenius 代数。从数值角度上看,当我们对三维的配分函数不断作用重整化算符时,实际上也就相当于实现了类似于三维 TRG 算法的操作。这种全息张量网络与 $p$-adic 张量网络也存在相似之处。后者中的张量单元由 $p$-adic CFT 融合范畴中结合代数的乘积/余积结构给出,而前者则需要利用 $D-1$ 维融合范畴中的 Frobenius 代数结构,这正是结合代数的自然推广。

我们在 \ref{sec:strange-correlator} 节中所介绍的奇异关联子也可以被推广到任意维度。对于一个 $D+1$ 维的拓扑序,它可以用一个 $D+1$ 维球 $B^{D+1}$ 上的融合范畴 $\mathcal{C}$ 来描述,其基态波函数为 $\ket{\Psi}$。指定 $D$ 维球面 $S^D$ 上的边界条件,就相当于将 $\ket{\Psi}$ 与某一直积态 $\bra{\Omega}$ 作内积,即
\begin{equation}
  Z(\Omega, \mathcal{C}) = \langle \Omega|\Psi \rangle.
\end{equation}
由于 $\ket{\Psi}$ 是一个拓扑序的基态波函数,它应该在标度变换下保持不变。设标度变换可由重整化算符 $\mathcal{H}_{\mathcal{C}}$ 生成,则可有
\begin{equation}
  Z(\Omega, \mathcal{C}) = \langle \Omega | \exp(z\mathcal{H}_{\mathcal{C}}) | \Psi \rangle,
\end{equation}
其中 $z$ 是重整化坐标。考虑到 $Z(\Omega, \mathcal{C})$ 描述了一个 $D$ 维的拓扑 / 共形系统,因此
\begin{equation}
  \bra{\Omega} \exp(z\mathcal{H}_{\mathcal{C}}) = \bra{\Omega},
  \label{eq:rg-operator}
\end{equation}
即配分函数的构建可以转化为关于 $\mathcal{H}_{\mathcal{C}}$ 的本征值问题。重整化算符 $\mathcal{H}_{\mathcal{C}}$ 可以将体($D+1$ 维球 $B^{D+1}$)与不同格点标度下的边界($D$ 维球面 $S^D$)联系起来,因而也就能够用来构建一套\emph{全息张量网络} (holographic tensor network)。下面我们讨论几个不同维度中的具体例子。

\subsection{1+1 维:基于 Dijkgraaf--Witten 理论}
\label{subsec:holographic-tensor-network-1+1d}

我们首先考察由群 $G$ 所刻画的 1+1 维 Dijkgraaf--Witten 模型\cite{dijkgraaf1990topological}。为了计算二维流形 $M^2$ 上的路径积分,我们先对其进行三角剖分,每个三角形对应于
\begin{equation}
  \alpha_2(g_1, g_2) \in H^2 \bigl( G, U(1) \bigr),
\end{equation}
其中 $H^2$ 是群上同调,而 $g_i\in G$ 满足群乘法
\begin{equation}
  g_1 \times g_2 = g_3.
\end{equation}
如图~\ref{fig:dw-triangle} 所示,群乘法也决定了三角形的定向。我们选取圆盘作为底流形,则路径积分将会在圆 $S^1$ 上给出相应二维模型的基态波函数。最简单的三角剖分如图~\ref{fig:dw-triangulation} 所示,它可以表示为一个矩阵乘积态:
\begin{equation}
    \ket{\Psi}_G
  = \sum_{\{g_i\}} \sum_{\{h_i\}}
    \bigl[ \cdots \tau_{h_i h_{i+1}}(g_i) \, \tau_{h_{i+1} h_{i+2}}(g_{i+1}) \cdots \bigr]
    \ket{\dots, g_i, g_{i+1}, \dots},
\end{equation}
其中
\begin{equation}
  \tau_{h_i h_{i+1}}(g) = \alpha^\varepsilon(h_i, h_{i+1}) \, \delta_{g, h_i h_{i+1}},
\end{equation}
而 $\varepsilon=\pm1$ 则根据三角形的定向来确定。

\begin{figure}[htb]
  \centering
  \subcaptionbox{\label{fig:dw-triangle}}{%
    \includegraphics[width=0.2\textwidth]{images/holographic/dw-triangle.png}} \quad
  \subcaptionbox{\label{fig:dw-triangulation}}{%
    \includegraphics[width=0.5\textwidth]{images/holographic/dw-triangulation.png}}
  \caption[1+1 维 Dijkgraaf--Witten 模型的三角剖分]{1+1 维 Dijkgraaf--Witten 模型的三角剖分。(a) Dijkgraaf--Witten 模型中,三角形对应了 $\alpha_2(g_1, g_2)$,而其定向则由群乘法决定。(b) 圆盘上最简单的三角剖分及对应 $\bra{\Omega}$ 的张量网络表示。图片来源:\parencite{chen2022exact}。}
\end{figure}

这些三角形满足结合条件:
\begin{equation}
  \alpha(g_1, g_2) \, \alpha(g_1 g_2, g_3) = \alpha(g_1, g_2 g_3) \, \alpha(g_2, g_3).
  \label{eq:associativity-condition}
\end{equation}
据此可将边界 $S^1$ 上的 $2N$ 条边变为 $N$ 条边。于是我们就得到了树状结构的算符 $U_N(G,\alpha)$,它联系起了长度为 $2N$ 和 $N$ 的两条边界。我们取
\begin{equation}
  \exp(z\mathcal{H}) \to U(G,\alpha) \coloneq \lim_{N\to\infty} U_N(G,\alpha).
\end{equation}
根据式~\eqref{eq:rg-operator},$U(G,\alpha)$ 的本征态 $\bra{\Omega}$ 即给出了标度不变的配分函数 $Z(\Omega,G)$,并且具有全局对称性 $G$(图~\ref{fig:rg-1+1d})。

\begin{figure}[htb]
  \centering
  \includegraphics[width=0.6\textwidth]{images/holographic/rg-1+1d.png}
  \caption[结合条件与树状结构的张量网络]{重复利用式~\eqref{eq:associativity-condition} 中的结合条件(图片下半部分),圆盘的三角剖分可以被改写为一个树状结构的张量网络。图片来源:\parencite{chen2022exact}。}
  \label{fig:rg-1+1d}
\end{figure}

简单起见,我们只考虑 $H^2$ 中的平凡元素,即 $\alpha(g_1, g_2)=1$。此时,$U(G,1)$ 的本征态 $\bra{\Omega}$ 可以写成 MPS 的形式,
\begin{equation}
    \bra{\Omega}
  = \sum_{\{g\}} \tr\bigl[ \cdots \rho(g_i) \, \rho(g_{i+1}) \cdots \bigr]
    \bra{\dots, g_i, g_{i+1}, \dots},
\end{equation}
其中 $\rho$ 是群 $G$ 的表示矩阵,它满足
\begin{equation}
  \bigl[ \rho(g_1) \, \rho(g_2) \bigr]_{ij} = \sum_s \rho_{is}(g_1) \, \rho_{sj}(g_2) = \rho_{ij}(g_1 g_2).
\end{equation}
此外还可以证明,重整化算符的一般形式可以分解为 $G$ 表示的直积:
\begin{equation}
  \rho(g) = \bigoplus_\mu \rho^\mu(g),
\end{equation}
式中 $\mu$ 标记了群 $G$ 的不同不可约表示。这样我们得到的配分函数 $Z=\langle\Omega|\Psi\rangle_G$ 就可以表示为一组张量
\begin{equation}
  T_{IJ}(g) \coloneq \rho_{ij}(g) \, \alpha^\varepsilon(h,k) \, \delta_{g, hk}, \quad
  I = (i,h), \, J = (j,k)
\end{equation}
的缩并形式,而其中所有的关联函数都会以指数级衰减,这与一维系统的性质也是相符的。

\subsection{2+1 维:基于 Turaev--Viro 理论}
\label{subsec:holographic-tensor-network-2+1d}

2+1 维的情况实际上就对应了奇异关联子的原始定义(\ref{sec:strange-correlator} 节)。我们考虑一个对应于融合范畴 $\mathcal{C}$ 的 Turaev--Viro TQFT。对于球面 $S^2$ 上的基态波函数,它可由一个三维球 $B^3$ 上的路径积分给出,利用三角剖分则可将 $B^3$ 变为四面体的形式。方便起见,我们不妨令剖分得到的截面与图~\ref{fig:dw-triangulation} 中的情形相同,即球面 $S^2$ 被三角形所覆盖,而三角形的每个顶点都有一条边延伸至球心。每条边被赋予了 $\mathcal{C}$ 中的一个对象,而每个四面体则被赋予一个四面体符号的值,它由 $\mathcal{C}$ 中对象的融合关系所定义。这样,我们就获得了基态波函数的张量网络表示(见图~\ref{fig:ground-state-wave-function},也可参考 \ref{subsec:string-net-ground-state} 小节)。这些四面体符号之间满足的关系在 \ref{subsec:tetrahedral-symmetry} 小节中也已经讨论过了,此处我们重新整理在图~\ref{fig:tetrahedra-symbols} 中。

\begin{figure}[htb]
  \centering
  \includegraphics[width=0.4\textwidth]{images/holographic/ground-state-wave-function.png}
  \caption[2+1 维 Turaev--Viro TQFT 中基态波函数的张量网络表示]{2+1 维 Turaev--Viro TQFT 中基态波函数的张量网络表示。蓝线代表物理指标,而红点代表垂直于 $S^2$ 面的辅助指标,它们对应了 $\mathcal{C}$ 中对象按量子维数的加权和。图片来源:\parencite{chen2022exact}。}
  \label{fig:ground-state-wave-function}
\end{figure}

\begin{figure}[htb]
  \centering
  \includegraphics[width=0.9\textwidth]{images/holographic/tetrahedra-symbols.png}
  \caption[四面体符号]{张量单元的取值由四面体符号给出,而五边形方程则相当于在原来四面体的表面(即两个三角形)上叠加一个新的四面体,从而改变了表面的三角剖分。图片来源:\parencite{chen2022exact}。}
  \label{fig:tetrahedra-symbols}
\end{figure}

与 1+1 维的情况类似,在 2+1 维中重整化算符同样通过连接两个不同格点间距的边界来构造。如图~\ref{fig:rg-2+1d} 所示,我们可以利用一系列的 $F$ 移动及相应的变换关系(图~\ref{fig:tetrahedra-symbols})来得到重整化算符 $U(\mathcal{C})$\cite{vanhove2018mapping}。它可以表示为一系列四面体的组合,并会将第 $N$ 次粗粒近似的波函数映射到第 $N+1$ 次上。可以发现,$U(\mathcal{C})$ 完全由融合范畴 $\mathcal{C}$ 中的数据所决定,而重复作用 $U(\mathcal{C})$ 将得到 Euclidean AdS 空间的一个离散化表示。

\begin{figure}[htb]
  \centering
  \includegraphics[width=0.9\textwidth]{images/holographic/rg-2+1d.png}
  \caption[利用五边形方程来实现粗粒近似]{利用五边形方程来实现粗粒近似。重整化算符可以视为一系列四面体的组合,例如第一幅图中的四个三角形在作用一组四面体后(标记为灰色),就变为了两个三角形。图片来源:\parencite{chen2022exact}。}
  \label{fig:rg-2+1d}
\end{figure}

% TODO: Frobenius Algebra

特别值得关注的是位于临界点处的配分函数。当非对易的拓扑对称性同时守恒时,即为临界点。由于配分函数 $Z(\mathcal{C})$ 是局域的,因而 $\bra{\Omega}$ 可以表示为一个 PEPS 张量网络。我们将张量单元 $T^{a_1 a_2 a_3}_{I_1 I_2 I_3}$ 取在三角形上,其中 $a_i\in\mathcal{C}$ 为物理指标,而 $I_i$ 为辅助指标,连接维数为 $\chi$。在重整化开始前,$\bra{\Omega}$ 是一个直积态,即所有 $I_i$ 都是平凡的;随着重整化的进行,相应的数据会“流入” $I_i$,得到非平凡的 PEPS 结构。

原则上,$T^{a_1 a_2 a_3}_{I_1 I_2 I_3}$ 的具体形式可以根据拓扑序中开弦算符按幂律衰减的性质来确定。然而,幂律衰减意味着 $\chi$ 会随着系统尺寸的增大而增大;对于无限系统而言,应有 $\chi\to\infty$。这在实际计算中显然是无法实现的。为此,我们采取另一套方案来获取临界点数据。考虑单个重整化步骤 $\bra{\Omega(T)}U(\mathcal{C})$。如图~\ref{fig:rg-2+1d} 所示,四个三角形被映射到了两个三角形,而它们之间带有纠缠的边界条件。对于这种边界,我们可以通过奇异值分解 (SVD) 把它编码到辅助指标 $I_i$ 上,并且把相应的连接维数控制在合理范围内。如图~\ref{fig:rg-2+1d-blocking},这时四个 $T$ 的乘积就可以被改写为两个新张量 $\tilde{T}$ 的缩并。

\begin{figure}[htb]
  \centering
  \includegraphics[width=0.8\textwidth]{images/holographic/rg-2+1d-blocking.png}
  \caption[边界态 $\bra{\Omega}$ 的 PEPS 张量网络表示]{上图为边界态 $\bra{\Omega}$ 的 PEPS 张量网络表示,其指标位于三角形的边上。每条边都有两个指标,一个是物理指标 $a_i\in\mathcal{C}$,另一个是连接维数为 $\chi$ 的辅助指标 $I_i$。在下图中,我们利用 SVD 把张量 $M(k)$ 分解为两个张量 $\tilde{T}$ 的缩并。指标 $H$(标记为红色)在生成时,连接维数为 $\chi^2$,但我们会将其截断为 $\chi$ 以使得重整化过程得以持续。图片来源:\parencite{chen2022exact}。}
  \label{fig:rg-2+1d-blocking}
\end{figure}

需要注意的是,对于有限大小的 $\chi$,临界点是不稳定的,而从重整化过程中获得的 $T$ 序列最终会流向某个拓扑不动点。因此我们认为,在重整化过程中,如果观察到 $T$ 开始向拓扑不动点的方向收敛,即达到了临界点。而对于固定的 $\chi$,总能利用这种方式找到一组 $T$ 使其最接近临界态。

% TODO: \mathfrak{su}(2)_k / SU(2)_k / A_{k+1}

接下来我们以 $\mathcal{A}_{k+1}$ 模型(见 \ref{subsec:a-k+1-category} 小节)为例给出一些具体的数值计算结果。如图~\ref{fig:a-k+1-boundary-condition} 所示,$\mathcal{A}_{k+1}$ 中的边界态可以取为
\begin{equation}
  \bra{\Omega} = \sum_{\{a\}} \prod_{a_1, a_2, a_3} T^{a_1, a_2, a_3} \bra{\{a\}},
\end{equation}
其中三角形上的 $T$ 张量为
\begin{equation}
  T^{a_1, a_2, a_3} = \begin{cases}
    1, & a_1 = a_3 = \frac12, \, a_2 = 0; \\
    r, & a_1 = a_3 = \frac12, \, a_2 = 1; \\
    0, & \text{otherwise}.
  \end{cases}
\end{equation}

\begin{figure}[htb]
  \centering
  \includegraphics[width=0.5\textwidth]{images/holographic/a-k+1-boundary-condition.png}
  \caption[$\mathcal{A}_{k+1}$ 模型中的边界条件]{$\mathcal{A}_{k+1}$ 模型中的边界条件。$k=2$ 时 $\mathcal{A}_{k+1}$ 等价于 Ising 模型,因而有 $r=\ee^{-2\beta}$。与图~\ref{fig:rg-2+1d-blocking} 相对比,此处的辅助指标 $I_i$ 是平凡的。图片来源:\parencite{chen2022exact}。}
  \label{fig:a-k+1-boundary-condition}
\end{figure}

通过重复作用重整化算符,我们可以让 $T$ 收敛到不动点。作为最粗糙的近似,这里只取连接维数 $\chi=1$。计算结果表明,对于较小的 $r$,不动点张量包含唯一非零分量 $T^{000}$,这对应了可分 Frobenius 代数 $\mathcal{A}_0=\{0\}$;而对于较大的 $r$,不动点张量的非零分量为 $T^{000}=T^{110}=T^{101}=T^{011}$,$k>2$ 时还存在 $T^{111}$,它们的比值为
\begin{equation}
  \frac{T^{000}}{T^{111}} \simeq \begin{cases}
    1.43463, & k = 3; \\
    1.18921, & k = 4.
  \end{cases}
\end{equation}
这与通过可分 Frobenius 代数 $\mathcal{A}_1=\{0,1\}$ 得到的精确值
\begin{equation}
  \frac{T^{000}}{T^{111}} = \frac{\sqrt[4]{2\cos\left(\frac{2\pi}{k+2}\right) + 1}}{\sqrt{2\cos\left(\frac{2\pi}{k+2}\right)}}, \quad
  k > 2
\end{equation}
是相符的。我们还可以得到临界耦合常数 $r_{\text{c}}(k)$(图~\ref{fig:a-k+1-coupling}),这与利用 Kramers--Wannar 对偶得到的精确值
\begin{equation}
  r_{\text{c}} = \frac{\sqrt[4]{2\cos\left(\frac{2\pi}{k+2}\right) + 1}}{\sqrt{2\cos\left(\frac{\pi}{k+2}\right) + 1}}
\end{equation}
也是基本符合的(至一位有效数字)。

\begin{figure}[htb]
  \centering
  \includegraphics[width=0.6\textwidth]{images/holographic/a-k+1-coupling.png}
  \caption[$\mathcal{A}_{k+1}$ 模型中的临界耦合常数]{$\mathcal{A}_{k+1}$ 模型中的临界耦合常数,这里 $r^{\text{a}}_{\text{c}}$ 和 $r^{\text{n}}_{\text{c}}$ 分别代表理论结果和数值模拟结果。它们符合的程度大致为一位有效数字。图片来源:\parencite{chen2022exact}。}
  \label{fig:a-k+1-coupling}
\end{figure}

\subsection{3+1 维概述}

类似地,三维配分函数也可通过 3+1 维 TQFT 基态波函数与对应的奇异关联子获得。对于 3+1 维 TQFT 而言,输入数据是 \emph{2-融合范畴} (2-fusion category)\cite{lan2018classification,lan2019classification,johnson2021dimensional,johnson2022classification},而相应张量网络的构建块是\emph{四维单形} (4-simplex)。此时,边会被赋予 2-范畴中的对象,面会被赋予 1-态射,而三维单形则会被赋予 2-态射。

3+1 维 TQFT 基态波函数的张量网络表示是 2+1 维的自然推广,它可由以三维球面 $S^3$ 为边界的四维球 $B^4$ 中的路径积分获得。边界 $S^3$ 可被三角剖分为一系列的三维单形,它们共享的指标则位于四维球 $B^4$ 的球心[即图~\ref{fig:dw-triangulation} 的高维推广]。在张量网络中,这些构建块可以表示为四面体的形式,其边和面上都带有指标。波函数则相当于把这些四面体铺满三维边界,并对共享的指标求和。

具体来说,三维边界由一系列小正方体填充,每个正方体可继续分为 6 个四面体,而顶点则被 8 个正方体共享[图~\ref{fig:3+1d-cube-1}]\cite{delcamp2021tensor}。注意这样的正方体单元并不是平移不变的,最小可重复的填充单元需要 $2\times2\times2$ 个小正方体[图~\ref{fig:3+1d-cube-2}]。

\begin{figure}[htb]
  \centering
  \subcaptionbox{\label{fig:3+1d-cube-1}}{%
    \includegraphics[width=0.25\textwidth]{images/holographic/3+1d-cube-1.png}} \quad
  \subcaptionbox{\label{fig:3+1d-cube-2}}{%
    \includegraphics[width=0.45\textwidth]{images/holographic/3+1d-cube-2.png}}
  \caption[3+1 维中的三角剖分]{3+1 维中的三角剖分。(a) 每个正方体被分为了 6 个四面体。(b) 每个 $2\times2\times2$ 的大正方体包含了 8 个小正方体,三角剖分的方式由小正方体的对角(红色圆点)来决定。图片来源:\parencite{chen2022exact}。}
\end{figure}

如图~\ref{fig:rg-3+1d} 所示,3+1 维中的重整化大致通过以下三个步骤来实现:

\begin{enumerate}
  \item 消去边中间位置的顶点 $m$、$n$ 等,得到新的边 $\overline{12}$、$\overline{23}$ 等;
  \item 消去面中间位置的顶点 $\alpha$、$\beta$ 等,得到新的边 $\overline{13}$、$\overline{16}$ 等;
  \item 消去体心 $o$,得到新的边 $\overline{17}$。
\end{enumerate}

\begin{figure}[htb]
  \centering
  \subcaptionbox{重整化的总体效果}{%
    \includegraphics[width=0.45\textwidth]{images/holographic/rg-3+1d.png}} \quad
  \subcaptionbox{消去棱心 $m$,得到边 $\overline{12}$}{%
    \includegraphics[width=0.45\textwidth]{images/holographic/rg-3+1d-step-1.png}} \\
  \subcaptionbox{消去面心 $\alpha$,得到边 $\overline{13}$}{%
    \includegraphics[width=0.45\textwidth]{images/holographic/rg-3+1d-step-2.png}} \quad
  \subcaptionbox{消去体心 $o$,得到边 $\overline{17}$}{%
    \includegraphics[width=0.45\textwidth]{images/holographic/rg-3+1d-step-3.png}}
  \caption[3+1 维中重整化算符的构造]{3+1 维中重整化算符的构造。$2\times2\times2$ 个正方体将被粗粒近似为 $1\times1\times1$ 个正方体。图片来源:\parencite{chen2022exact}。}
  \label{fig:rg-3+1d}
\end{figure}

考虑 3+1 维 toric code 模型(对应于 3+1 维 $\mathbb{Z}_2$ Dijkgraaf--Witten 模型)\cite{hamma2005string,zhao2022string},其边界态可以取为
\begin{equation}
  \bra{\Omega} = \sum_{\{a\}} \Omega_{\{a\}} \bra{\{a\}} = \bigotimes_e \bra{w_e},
\end{equation}
其中 $\bra{w_e}$ 是棱 $e$ 上的态。如图~\ref{fig:3+1d-cube-3} 所示,$\bra{w_e}$ 可分为两类:
\begin{equation}
  \bra{w_e} = \begin{cases}
    \bra{1} + \bra{-1}, & \text{red edges}; \\
    \ee^\beta \bra{1} + \ee^{-\beta} \bra{-1}, & \text{blue edges},
  \end{cases}
\end{equation}
由此得到的奇异关联子为
\begin{equation}
  Z = \langle\Omega|\Psi\rangle = \sum_{\{\sigma\}} \prod_{\ev{ij}} \ee^{\beta \sigma_i \sigma_j}
\end{equation}
可以发现它和三维 Ising 模型的配分函数是一致的。将正方体单元进一步拆分为四面体,可以给出 $\bra{\Omega}$ 的张量网络表示:
\begin{equation}
  \bra{\Omega(u_\beta)} = \sum_{\{a\}} \sum_{i,j,k,l,\dots} \cdots u_\beta^{ijkl} (a_1,a_2,a_3) \cdots \bra{\{a\}}.
\end{equation}
其中 $\sum_{\{a\}}$ 表示对所有可能的构型求和,而 $\sum_{i,j,k,l,\dots}$ 则表示对面上的指标求和,而张量单元
\begin{equation}
  u_\beta^{ijkl} (a,b,c) = \begin{cases}
    \ee^{\beta(a+2b+c)/8}, & i=j=k=l=1; \\
    0, & \text{otherwise}
  \end{cases}
\end{equation}
对应图~\ref{fig:3+1d-tetrahedron} 中的小四面体,其中 $a,b,c=\pm1$。

\begin{figure}[htb]
  \centering
  \subcaptionbox{\label{fig:3+1d-cube-3}}{%
    \includegraphics[width=0.3\textwidth]{images/holographic/3+1d-cube-3.png}} \qquad
  \subcaptionbox{\label{fig:3+1d-tetrahedron}}{%
    \includegraphics[width=0.25\textwidth]{images/holographic/3+1d-tetrahedron.png}}
  \caption[3+1 维 toric code 模型中的边界态]{3+1 维 toric code 模型中的边界态。(a) 边界中的棱分为红色和蓝色两类。(b) $a$、$b$、$c$ 表示棱上的自由度,$i$、$j$、$k$、$l$ 则是面上的指标。图片来源:\parencite{chen2022exact}。}
\end{figure}

按照上面介绍的重整化方案,我们可以找到 $u_\beta$ 的两个不动点,分别为 $u_0$ 和 $u_\infty$,而临界点大致为
\begin{equation}
  0.27 \lesssim \beta_{\text{c}} \lesssim 0.28,
\end{equation}
这与通过其他方法得到的三维 Ising 模型数值结果 $\beta_{\text{c}}=0.22165463(8)$\cite{hasenbusch2010finite}同样也是基本符合的(至一位有效数字)。注意到与 2+1 维的情况相同,这里连接维数仍然只取为 $\chi=1$。可以想见,增大连接维数将提高计算的精度。

\section{AdS/CFT 的张量网络重建}
\label{sec:ads-cft-tensor-network}

在上一节中,我们通过 $D+1$ 维拓扑格点模型构造了重整化算符,并且在 $D=1,2,3$ 的各维度中都给出了例子。将这些重整化算符不断堆叠,即取 $\lim_{N\to\infty}U(\mathcal{C})^N$,我们将得到一个类似 MERA 的张量网络结构(见 \ref{subsec:mps-generalization} 小节)。MERA 是一类全息张量网络,在给出特定的边界条件并使其能够流向不动点之后,它可以精确地描述 CFT。我们也通过三维 Ising 模型的例子展示了这一点。

此外,我们还注意到,奇异关联子的构造与 $p$-adic 张量网络也具有相似之处,而后者能够复现 $p$-adic AdS/CFT 中的诸多结论\cite{hung2019padic,chen2021emergent,chen2021bending1,chen2021bending2}。$p$-adic 张量网络的体态通过结合代数(CFT 算符的融合范畴)描述,而我们这里所采用的(高阶)融合范畴实际上是它的推广。在 $p$-adic 张量网络中,$p$-adic CFT 的配分函数也是以类似奇异关联子的形式给出的。在不动点处,边界态也对应了体全息张量网络的本征态。因此,我们这里的构造也有望用来实现 AdS/CFT 对应。

考虑在 \ref{subsec:holographic-tensor-network-2+1d} 小节中所探讨的二维 Ising 模型(即 $k=2$ 时的 $\mathcal{A}_{k+1}$ 模型)。我们在紫外 (UV) 层(图~\ref{fig:a-k+1-boundary-condition})和全息张量网络的第 $n$ 层分别插入两个算符 $\mathcal{O}_1$ 和 $\mathcal{O}_n$,并数值地计算它们之间的关联函数,即
\begin{align}
  &\mathrel{\phantom{=}}
     \ev[\big]{\mathcal{O}_{n=1}(x=y=0) \, \mathcal{O}_n(x,y)} \notag \\
  &= \bigl\langle \Omega(T_{\Lambda_1}) \big|
     \sigma^z(x=y=0) \, U^{n-1}(\mathcal{C}) \, \sigma^z(x,y) \, U^n(\mathcal{C}) \cdots
     \big| \Psi \bigr\rangle \notag \\
  &= \bigl\langle \Omega(T_{\Lambda_1}) \big|
     \sigma^z(x=y=0) \, U^{n-1}(\mathcal{C}) \, \sigma^z(x,y)
     \big| \Psi_{\Lambda_n} \bigr\rangle.
\end{align}
计算结果如图~\ref{fig:bulk-boundary-propagator} 所示。可以发现,与上图相比,下图中对应于不同 $n$ 的点几乎都重叠在一起,这表明存在满足
\begin{equation}
  \ev{\mathcal{O}_1 \, \mathcal{O}_n} \sim \left[ \frac{z}{x^2+z^2} \right]^\Delta
\end{equation}
形式的\emph{体—边传播子} (bulk-boundary propagator)\cite{witten1998anti}。式中,$x$ 为两算符间的距离,$z$ 为体到边界的距离,而 $\Delta$ 则是相应的标度维数(见 \ref{sec:conformal-symmetry} 小节)。这里所展示的只是一个初步结果,我们还需要对其他模型以及不同标度维数的算符进行进一步的检验。

\begin{figure}[htb]
  % TODO: update image
  \centering
  \includegraphics[width=0.45\textwidth]{images/holographic/bulk-boundary-propagator-1.png}
  \includegraphics[width=0.45\textwidth]{images/holographic/bulk-boundary-propagator-2.png}
  \caption[体—边传播子的计算结果]{体—边传播子的计算结果。$\mathcal{O}_n$ 表示在第 $n$ 层插入的算符。为了计算方便,$\mathcal{O}_1$ 会和邻近的张量缩并,并推至第 $n$ 层,得到 $\mathcal{O}_{1n}$,因此 $\ev{\mathcal{O}_n\,\mathcal{O}_{1n}}$ 即相当于 $\ev{\mathcal{O}_n\,\mathcal{O}_1}$。(a)(b) 分别是关于 $z_n x_n^2$ 和 $z_n(x_n^2+1)$ 的 log-log 图像,其中 $z_n=(\sqrt{2})^{n-1}$,而 $x_n=x_1/z_n$ 是 $\mathcal{O}_n$ 和 $\mathcal{O}_{1n}$ 间的距离。在 AdS/CFT 中,有 $\ev{\mathcal{O}_n\,\mathcal{O}_1}\sim[z_n(x_n^2+1)]^{-\Delta}$,其中 $\Delta$ 是初级场 $\mathcal{O}_1$ 的标度维数。可以发现,在图 (b) 中对应于不同 $n$ 的点几乎都重叠在一起,这说明体—边传播子的形式是符合 AdS/CFT 预测的。图片来源:\parencite{chen2022exact}。}
  \label{fig:bulk-boundary-propagator}
\end{figure}

\section{本章小结}

本章首先给出了弦网模型基态的 PEPS 张量网络表示,并通过将其与特定的直积态做内积得到奇异关联子。在 Fibonacci 和 Ising 两种模型中,我们利用奇异关联子构造了相应配分函数的张量单元,并通过 iTEBD 算法数值计算了对应 CFT 的中心荷。接下来,我们简要介绍了 PEPS 张量网络表示中的 MPO 对称性,而 MPO 所满足的推拉条件实质也是融合范畴结构的一种体现。我们还给出了管状代数与中心幂等元的构造,后者作为投影算符,可以给出 Drinfel'd 中心中不同的拓扑分区。最后,我们详细讨论了利用奇异关联子构建全息张量网络的方案。由于拓扑序的基态波函数 $\ket{\Psi}$ 位于重整化的不动点处,因而边界条件 $\bra{\Omega}$ 的选取就可以转化为重整化算符的本征值问题。我们在 1+1、2+1 和 3+1 维中分别给出了重整化算符及边界态的构造,并计算得到了临界点位置的初步数值结果。在二维 Ising 模型中,我们还利用全息张量网络计算了体—边传播子,并发现其形式与 AdS/CFT 的预测基本符合。

\chapter{Virasoro 与 Kac--Moody 代数的张量网络实现}

\section{Virasoro 与 Kac--Moody 代数简介}

\emph{共形场论} (conformal field theory, CFT) 即满足\emph{共形对称性} (conformal symmetry) 的量子场论。二维共形场论可以用很自然地用复平面上的坐标 $z$ 和 $\bar{z}$ 来描述。在共形映照 $z\to w(z)$、$\bar{z}\to\bar{w}(\bar{z})$ 下,满足
\begin{equation}
  \phi'(w,\bar{w}) = \left( \dv{w}{z} \right)^{-h} \left( \dv{\bar{w}}{\bar{z}} \right)^{-\bar{h}} \phi(z,\bar{z})
  \label{eq:quasi-primary-field}
\end{equation}
这一变换关系的场 $\phi$ 称为\emph{准初级场} (quasi-primary field),其中
\begin{equation}
  h = \frac12 \bigl( \Delta+s \bigr), \quad \bar{h} = \frac12 \bigl( \Delta-s \bigr)
\end{equation}
称为\emph{共形维数} (conformal dimension),而 $\Delta$ 和 $s$ 分别称为\emph{标度维数} (scaling dimension) 和\emph{自旋} (spin),它们确定了 $\phi$ 在标度和旋转变换下的性质。如果对任意的局部共形变换,式~\eqref{eq:quasi-primary-field} 都成立,则称 $\phi$ 为\emph{初级场} (primary field)。

根据 Noether 定理,(连续)对称性对应了守恒流。因而我们可以为局部坐标变换定义\emph{能动量张量} (energy-momentum tensor),也称\emph{应力张量} (stress tensor)。在共形对称性的条件下,能动量张量 $T^{\mu\nu}$ 可以取为对称且无迹的,即只剩下 $T(z)\coloneq T_{zz}(z)$ 和 $\bar{T}(\bar{z})\coloneq T_{\bar{z}\bar{z}}(\bar{z})$。$T$ 和 $\bar{T}$ 的共形维数分别为 $(h_T,\bar{h}_T)=(2,0)$ 和 $(h_{\bar{T}},\bar{h}_{\bar{T}})=(0,2)$,即
\begin{equation}
  \Delta_T = \Delta_{\bar{T}} = 2, \quad s_T = 2, \quad s_{\bar{T}} = -2.
\end{equation}

对于共形维数为 $(h,\bar{h})$ 初级场 $\phi$,能动量张量与它的\emph{算子积展开} (operator product expansion, OPE) 具有如下形式:
\begin{equation}
  \begin{aligned}
    T(z) \, \phi(w,\bar{z}) &\sim
      \frac{h}{(z-w)^2} \phi(w,\bar{z}) + \frac{1}{z-w} \partial_w\phi(w,\bar{z}), \\
    \bar{T}(\bar{z}) \, \phi(w,\bar{z}) &\sim
      \frac{\bar{h}}{(\bar{z}-\bar{w})^2} \phi(w,\bar{z}) + \frac{1}{\bar{z}-\bar{w}} \partial_{\bar{w}}\phi(w,\bar{z}).
  \end{aligned}
  \label{eq:t-phi-ope}
\end{equation}
而能动量张量与自身的 OPE 则可写为
\begin{equation}
  \begin{aligned}
    T(z) \, T(w) &\sim
      \frac{c/2}{(z-w)^4} + \frac{2}{(z-w)^2} T(w) + \frac{1}{z-w} \partial_w T(w), \\
    \bar{T}(\bar{z}) \, \bar{T}(\bar{w}) &\sim
        \frac{\bar{c}/2}{(\bar{z}-\bar{w})^4}
      + \frac{2}{(\bar{z}-\bar{w})^2} \bar{T}(\bar{w})
      + \frac{1}{\bar{z}-\bar{w}} \partial_{\bar{w}}\bar{T}(\bar{w}).
  \end{aligned}
\end{equation}
其中 $(c,\bar{c})$ 称为\emph{中心荷} (central charge)。

把能动量张量进行模展开,可以得到
\begin{equation}
  \begin{aligned}
    T(z)             &= \sum_{n\in\mathbb{Z}} z^{-n-2} L_n, &\quad
    L_n              &= \frac{1}{2\pi\ii} \oint z^{n+1} T(z) \, \dd z; \\
    \bar{T}(\bar{z}) &= \sum_{n\in\mathbb{Z}} \bar{z}^{-n-2} \bar{L}_n, &\quad
    \bar{L}_n        &= \frac{1}{2\pi\ii} \oint \bar{z}^{n+1} \bar{T}(\bar{z}) \, \dd\bar{z}.
  \end{aligned}
\end{equation}
式中 $L_n$ 和 $\bar{L}_n$ 称为 \emph{Virasoro 算符} (Virasoro operators),它们构成了 \emph{Virasoro 代数} (Virasoro algebra):
\begin{equation}
  \begin{aligned}
    \bigl[ L_n, L_m \bigr]
      &= (n-m) L_{n+m} + \frac{c}{12} n \bigl( n^2-1 \bigr) \delta_{n+m,0}, \\
    \bigl[ \bar{L}_n, \bar{L}_m \bigr]
      &= (n-m) \bar{L}_{n+m} + \frac{\bar{c}}{12} n \bigl( n^2-1 \bigr) \delta_{n+m,0}, \\
    \bigl[ L_n, \bar{L}_m \bigr] &= 0.
  \end{aligned}
  \label{eq:virasoro-algebra}
\end{equation}

二维共形场论可以进行\emph{径向量子化} (radial quantization),此时圆柱面可以被映射到平面上;特别地,$t\to-\infty$ 时刻会被映射到坐标原点 $z=\bar{z}=0$。因此有\emph{态—算符对应} (state-operator correspondence):
\begin{equation}
  \ket{\phi} = \lim_{z,\bar{z}\to\infty} \phi(z,\bar{z}) \ket{0},
\end{equation}
这意味着每一个场算符都可以生成一个对应的态。真空态 $\ket{0}$ 需要在全局共形变换下保持不变,这要求
\begin{equation}
  L_n \ket{0} = \bar{L}_n \ket{0} = 0, \quad n \geqslant -1.
\end{equation}
设初级场 $\phi$ 对应的态为 $\ket*{h,\bar{h}}\coloneq\phi(0,0)\ket{0}$。根据式~\eqref{eq:t-phi-ope},可知
\begin{equation}
  L_0       \ket*{h,\bar{h}} = h       \ket*{h,\bar{h}}, \quad
  \bar{L}_0 \ket*{h,\bar{h}} = \bar{h} \ket*{h,\bar{h}}, \quad
  L_n \ket*{h,\bar{h}} = \bar{L}_n \ket*{h,\bar{h}} = 0 \enspace (n > 0).
\end{equation}
利用式~\eqref{eq:virasoro-algebra} 中的对易关系,有
\begin{equation}
  \bigl[ L_0, L_{-n} \bigr] = n L_{-n}, \quad
  \bigl[ \bar{L}_0, \bar{L}_{-n} \bigr] = n \bar{L}_{-n}.
\end{equation}
可以看出 $L_{-n}\ket{0}$ 和 $\bar{L}_{-n}\ket{0}$ 分别是 $L_0$ 和 $\bar{L}_0$ 本征值为 $n$ 的本征态,因而 $L_{-n}$、$\bar{L}_{-n}$ 即可作为升算符,使得共形维数 $h$、$\bar{h}$ 增加 $n$。产生的这些态称为 $\ket*{h,\bar{h}}$ 的\emph{后代} (descendant),它们实际上也可以通过对 $\phi$ 求导得到。

\section{离散全纯性}

\section{Virasoro 与 Kac--Moody 算符的构造}

\section{能动量张量的确定}

\chapter{全息张量网络中的算符推移}
\label{chap:operator-pushing}

\renewcommand{\arraystretch}{0.8}

在本章中,我们将沿着第 \ref{chap:strange-correlator} 章的思路,继续考察基于拓扑序的重整化张量网络。这类张量网络可以揭示出对应 CFT 的性质,并且在数值上还可以看到重整化张量与 AdS 体的相似性。因此,研究其中的算符推移及演生广义自由场的条件具有重要意义。

\section{广义自由场与算符推移}

AdS/CFT 对应\cite{maldacena1999large}中的一项重要特征是在 AdS 体 (bulk) 中演生出的\emph{自由场} (free field)。这使得我们可以为体态提供一套半经典的描述,同时也可以通过体来计算边界 CFT 中的物理量。这里所谓“自由场”,是指其关联函数可以通过 Wick 缩并来计算。一般来说,边界 CFT 中的\emph{单迹算符} (single trace operator) 近似对应了半经典的体态理论中的自由 Gauss 场。这就使得 CFT 中的关联函数可以在 AdS 一侧通过 Witten 图来计算\cite{witten1998anti,gubser1998gauge}。也有很多工作探讨了全息 CFT 中所演生出的广义自由场\cite{dutsch2003generalized,liu2019dimensional,collier2019quantum,nebabu2023bulk}。

另一方面,受 Ryu--Takayanagi 公式\cite{ryu2006holographic}的启发,也有人提出张量网络是构建 CFT 与 AdS 自由度之间线性映射的合适框架\cite{swingle2012entanglement}。如图~\ref{fig:mera-ads-cft} 所示,这类张量网络的图形表示也和 AdS 空间的离散版本非常相似。它描述了 CFT 边界自由度的重整化(粗粒近似)过程,而这些自由度同时也位于张量网络的渐进边界。体中的算符作用在张量网络的辅助指标上,而 CFT 中的算符则作用在张量网络的渐进边界上。因此,张量网络提供了这些算符之间的线性映射\cite{pastawski2015holographic,hayden2016holographic}。

\begin{figure}[htb]
  \centering
  \includegraphics[width=0.5\textwidth]{images/temp/mera-ads-cft.pdf}
  \caption[MERA 张量网络与 AdS/CFT]{MERA 张量网络可以被理解为 AdS/CFT 的一种离散实现。一维格点模型的基态波函数对应了 $\text{CFT}_{1+1}$ 的真空态,而由此“生长”出的二维 MERA 张量网络则给出了 $\text{AdS}_{2+1}$ 时间切片的一种离散表示。图片来源:\parencite{evenbly2011tensor}。}
  \label{fig:mera-ads-cft}
\end{figure}

由于张量网络实际上是局域化的,它可以分解为一系列张量的乘积,而这些张量只与邻近的张量相缩并(邻近张量的具体数目取决于体空间的维数),因而我们可以通过\emph{算符推移} (operator pushing) 的方法来找出相应的\emph{体—边映射} (bulk-boundary map)\cite{pastawski2015holographic,bhattacharyya2016exploring,bhattacharyya2018tensor}。

于是这就引出了一个自然的问题:对于一个弱耦合的体理论,其中的场几乎都是自由的,那么此时什么样的张量网络最能描述相应 AdS/CFT 中的体—边映射?文献 \parencite{bhattacharyya2018tensor} 对这个问题进行了一些讨论。如图~\ref{fig:operator-pushing} 所示,对于体中的算符 $\mathcal{O}$,它作用在体内部的自由度上;而当利用算符推移把 $\mathcal{O}$ 拉到边界时,这种作用应当等价于另一组算符 $\mathcal{O}'$ 在边界自由度上的作用。在利用边界算符对体算符进行重建时,我们有必要对入射、出射腿的方向做出选择。事实上,当张量网络对应了重整化(粗粒近似)操作时,这种选择是唯一的:作用在粗粒近似后的自由度上的算符,应当由作用在粗粒近似前的自由度上的算符来决定。此时可以证明,一个广义的自由场在利用算符推移穿过张量网络后,可以被分解为一系列\emph{简单算符} (simple operators) 之和\cite{bhattacharyya2018tensor}。

\begin{figure}[htb]
  \centering
  % TODO: operator-pushing
  \caption{算符推移}
  \label{fig:operator-pushing}
\end{figure}

考虑一个由重整化张量 $M^{i_1,\dots,i_K}_{j_1,\dots,j_H}\colon\mathbb{V}^K\to\mathbb{V}^H$ 构成的张量网络,其中 $K>H$,$i_l,j_l\in\mathbb{V}$,而 $\mathbb{V}$ 是一个 $d$ 维向量空间。那么算符推移对应于
\begin{equation}
  \mathcal{O} \bigl( \mathbb{V}^H \bigr) \cdot M = M \cdot \mathcal{O} \bigl( \mathbb{V}^K \bigr).
  \label{eq:operator-pushing}
\end{equation}
一个近似自由的体算符会作用在 $\mathbb{V}^H$ 其中一个指标上,即
\begin{equation}
    \mathcal{O} \bigl( \mathbb{V}^H \bigr)
  = \mathbb{I}_1 \otimes \cdots \otimes \mathbb{I}_{i-1} \otimes X_i \otimes \mathbb{I}_{i+1} \otimes \dots \otimes \mathbb{I}_{H}.
\end{equation}
当 $\mathcal{O}$ 满足式~\eqref{eq:operator-pushing} 时,则有
\begin{equation}
    \mathcal{O} \bigl( \mathbb{V}^K \bigr)
  = \sum_{i=1}^K \alpha_i \, \bigl(
      \mathbb{I}_1 \otimes \cdots \otimes \mathbb{I}_{i-1} \otimes X_i \otimes \mathbb{I}_{i+1} \otimes \dots \otimes \mathbb{I}_{K}
    \bigr),
  \label{eq:operator-pushing-coefficients}
\end{equation}
其中 $\{\alpha_i\}$ 为常数。此时,作用在体自由度 $l_B$ 上的体算符 $\mathcal{O}(l_B)$ 就可以用边界算符重建出来:
\begin{equation}
  \mathcal{O} \bigl( l_B \bigr) = \sum_b K^I \bigl( l_b, l_B \bigr) \, \mathcal{O}^I \bigl( l_b \bigr),
\end{equation}
其中 $\{\mathcal{O}^I\}$ 是作用在边界自由度 $l_b$ 上算符的一组完备基,而 $K^I(l_b, l_B)$ 是\emph{体—边核} (bulk-boundary kernel),它可用式~\eqref{eq:operator-pushing-coefficients} 中的系数 $\alpha_i$ 表示。可以发现,这一表达式与通过\emph{体—边传播子} (bulk-boundary propagator) 构造的 \emph{HKLL 核} (HKLL kernel)\cite{hamilton2006local,hamilton2006holographic}具有相同的形式。此外还可以证明,体算符关联函数的行为与广义自由场是类似的\cite{bhattacharyya2018tensor,hung2019padic}。由于 $M$ 是一个长方形的矩阵,所以这种重建不是唯一的。然而,只要式~\eqref{eq:operator-pushing-coefficients} 成立,那么广义自由场的解总是存在的。

\section{1+1 维中的算符推移}

我们首先讨论 1+1 维的情形,即由有限群 $G$ 所描述的 Dijkgraaf--Witten 理论。在 \ref{subsec:holographic-tensor-network-1+1d} 小节中,我们已经给出了重整化算符的构造。它对应了一个树状结构的张量网络,其中每个顶点都是三分支的。在无扭转的 Dijkgraaf--Witten 理论中,每个顶点对应了一个带有三个指标的张量 $M^{g_1g_2}_{g_3}$:
\begin{equation}
  \tikzinput{operator-pushing/tensor-1+1d},
\end{equation}
其中 $g_1$、$g_2$、$g_3$ 都是群 $G$ 中的元素。张量 $M$ 描述了 $G$ 中的群乘法(也相当于融合规则),即
\begin{equation}
  M^{ij}_k = \delta_{G(i,j), k}, \quad
  G(g_1, g_2) \coloneq g_1 \times g_2 = g_3.
\end{equation}

如图~\ref{fig:rg-1+1d} 所示,这一由重整化算富所构成的张量网络具有一维边界。当在体中插入一个算符 $B$ 时,它的作用应当等价于边界上插入的另外一些算符。找到能再现体算符作用的边界算符,也就是我们所说的算符推移或\emph{体算符重建} (bulk operator reconstruction) 问题。

由于张量网络是局域的,我们可以通过研究重整化张量网络中的一个张量来研究体算符重建。具体来说,对于一个张量单元而言,算符推移或重建相当于为给定的算符 $B$ 找到对应的算符 $A$,并使其满足
\begin{equation}
  \tikzinput{operator-pushing/constraint-1+1d}.
  \label{eq:constraint-1+1d}
\end{equation}
当算符 $B$ 是所谓广义自由场时,由 $A$ 给出的算符重建应当可以表示为“简单”形式(即 $\mathbb{I}\otimes\tilde{A}$ 或 $\tilde{A}\otimes\mathbb{I}$ 的线性组合)。为了确保 $B$ 的行为满足广义自由场的条件,我们还要求 $\tilde{A}$ 仍然属于算符集 $\{B\}$,也就是要求这一重建过程可以被不断重复至任意层的张量网络。式~\eqref{eq:constraint-1+1d} 可以表述为
\begin{equation}
  A_{(ij), (i'j')} M_{(i'j'), k} = M_{(ij), k'} B_{k', k}.
  \label{eq:constraint-equation-1+1d}
\end{equation}
为计算方便,我们把 $i$、$j$ 合并成了单个指标 $(ij)$,类似张量网络中的变形操作[式~\eqref{eq:tensor-reshape}]。接下来我们考虑对一般的算符 $B$ 求解式~\eqref{eq:constraint-equation-1+1d},并从中确定广义自由场的集合。还可以注意到,在张量网络中,如果算符 $A$ 能够表示为简单形式,则算符 $B$ 的集合将构成广义自由场的一组完备基。

在式~\eqref{eq:constraint-1+1d} 中张量的每条腿上,都有一个 $|G|$ 维的向量空间。作为 $\mathbb{Z}_2$ 情形中 Pauli 矩阵的推广,我们可以利用\emph{广义 Pauli 矩阵} (generalized Pauli matrices)\cite{patera1988pauli}作为基,来构建作用在每条腿上的算符。对于有限群 $G$ 且有 $|G|=n$,$G$ 中群元标记为 $0, 1, \dots, n-1$,则 $|G|$ 维的向量空间中的基可以表示为
\begin{equation}
  \ket{i} = \begin{pmatrix} 0 \\ \vdots \\ 1_{\text{$i$-th}} \\ \vdots \\ 0 \end{pmatrix}, \quad
  i \in \{ 0, 1, \dots, n-1 \}.
\end{equation}
在这组基下,广义 Pauli 矩阵可通过\emph{移位矩阵} (shift matrix) $X$ 和\emph{时钟矩阵} (clock matrix) $Z$ 生成:
\begin{equation}
  X = \begin{pmatrix}
    0      & 1      & 0      & \cdots & 0      \\
    0      & 0      & 1      & \cdots & 0      \\
    \vdots & \vdots & \vdots & \ddots & \vdots \\
    0      & 0      & 0      & \cdots & 1      \\
    1      & 0      & 0      & \cdots & 0
  \end{pmatrix}, \quad
  Z = \begin{pmatrix}
    1      & 0      & \cdots & 0            & 0      \\
    0      & \omega & \cdots & 0            & 0      \\
    \vdots & \vdots & \ddots & \vdots       & \vdots \\
    0      & 0      & \cdots & \omega^{n-2} & 0      \\
    0      & 0      & \cdots & 0            &\omega^{n-1}
  \end{pmatrix},
  \label{eq:generalized-pauli-matrices}
\end{equation}
这里 $\omega=\ee^{2\pi i/n}$ 是 $n$ 次单位根。于是广义 Pauli 矩阵为
\begin{equation}
  \sigma^\mu \coloneq \sigma^{ns+t} = X^t Z^s,
\end{equation}
其中 $s=\lfloor \mu/n\rfloor$、$t=\mu\bmod n$,而 $\mu\in\{0,\dots,n^2-1\}$。此时式~\eqref{eq:constraint-equation-1+1d} 可以写为
\begin{align}
     A_{(ij), (i'j')} \delta_{G(i',j'), k}
  &= \delta_{G(i,j), k'} B_{k', k} \notag \\
  &= \delta_{G(i,j), k'} \sigma^\mu_{k', k}
   = \sigma^\mu_{G(i,j), k}.
  \label{eq:1+1d-constraint-equation}
\end{align}
因此,一旦给定了群乘法 $G(i,j)$,我们就可以根据标准的线性代数方法,对于每一个 $\sigma_\mu$ 求解上式中的 $A$。完整的解系应包含通解和特解两个部分,其中通解部分对应于齐次方程
\begin{equation}
  A_{(ij), (i'j')} \delta_{G(i',j'), k} = 0
\end{equation}
的解,即相当于 $(\sigma^\mu_{G(i,j),k})^{\mathrm{T}}=\sigma^\mu_{k,G(i,j)}$ 的\emph{零空间} (null space)。而注意到
\begin{equation}
  G(i,0) = i, \quad G(0,j) = j,
\end{equation}
我们可以给出一组特解:
\begin{equation}
  A^{(\mu)}_{(ij), (i'j')} = \begin{cases}
    \sigma^\mu_{G(i,j), j'}, & i' = 0; \\
    0, & i' \neq 0.
  \end{cases}
  \label{eq:1+1d-specific-solution}
\end{equation}

我们还想找到能使算符 $A$ 为简单形式的体算符 $B$ 的子集,此时要求 $A$ 在张量一条腿上的作用是平凡的,即
\begin{equation}
  A_{(ij), (i'j')} = \tilde{A}_{ii'} \delta_{jj'} \quad \text{or} \quad
  A_{(ij), (i'j')} = \delta_{ii'} \tilde{A}_{jj'}.
\end{equation}
于是有
\begin{equation}
  \tilde{A}_{ii'} \delta_{G(i',j), k} = \sigma^\mu_{G(i,j), k} \quad \text{or} \quad
  \tilde{A}_{jj'} \delta_{G(i,j'), k} = \sigma^\mu_{G(i,j), k}, \quad
  \forall j \in \{ 0, 1, \dots, n-1 \}.
  \label{eq:1+1d-simple-form-constraint-equation}
\end{equation}
$\tilde{A}$ 仍然可以很容易地求解得到。根据线性代数中的 Rouch\'e--Capelli 定理,上式有解的充分必要条件是对于 $(\sigma^\mu_{G(i,j),k})^{\mathrm{T}}$ 的每一列,增广矩阵的秩都满足
\begin{equation}
    \rank\bigl[ (\delta_{G(i',j), k})^{\mathrm{T}} \big| (\sigma^\mu_{G(i,j),k})^{\mathrm{T}}_s \bigr]
  = \rank\bigl[ (\delta_{G(i',j), k})^{\mathrm{T}} \bigr].
\end{equation}
又因为
\begin{equation}
    \rank\bigl[ (\delta_{G(i',j), k})^{\mathrm{T}} \bigr]
  = \rank\bigl( \delta_{G(i',j), k} \bigr) = n,
\end{equation}
如果式~\eqref{eq:1+1d-simple-form-constraint-equation} 有解,也只存在一组解。

\subsection{Abelian 的例子:\texorpdfstring{$\mathbb{Z}_n$}{ℤₙ} 群}

接下来我们考察几个具体的例子。对于 $\mathbb{Z}_n$ 群,其融合规则由模算术给出:
\begin{equation}
  G(i,j) = (i+j)\bmod n,
  \label{eq:Z_n-fusion-rules}
\end{equation}
因此 $\delta_{G(i,j),k}=\delta_{(i+j)\bmod n,k}$ 是一个秩为 $n$ 的 $n^2\times n$ 矩阵,其转置矩阵的零空间可由 $n^2-n$ 个向量 $\mathbf{v}$ 张成:
\begin{equation}
  v^{(p)}_q = \delta_{(-p-\lfloor p/n\rfloor-2)\bmod n, \, q} - \delta_{n^2-p-1, \, q},
\end{equation}
其中
\begin{equation}
  p \in \left\{ 0, 1, \dots, n^2-n-1 \right\}, \quad
  q \in \left\{ 0, 1, \dots, n^2-1 \right\}.
\end{equation}
于是通解 $A^*$ 可由 $v^{(p)}$ 的线性组合给出:
\begin{equation}
  A^* = \begin{pmatrix}
    \beta_{0,0} \mathbf{v}^{(0)} + \dots + \beta_{0,n^2-n-1} \mathbf{v}^{(n^2-n-1)} \\
    \vdots \\
    \beta_{n^2-1,0} \mathbf{v}^{(0)} + \dots + \beta_{n^2-1,n^2-n-1} \mathbf{v}^{(n^2-n-1)}
  \end{pmatrix},
\end{equation}
其中 $\beta_{i,j}$ 是任意常数。而根据式~\eqref{eq:1+1d-specific-solution},特解部分为
\begin{equation}
  A^{(\mu)}_{(ij), (0j')} = \sigma^\mu_{(i+j)\bmod n, j'}.
\end{equation}

对于算符 $A$ 为简单形式的情形,我们注意到当且仅当 $\mu\in\{0,1,\dots,n-1\}$ 时式~\eqref{eq:1+1d-simple-form-constraint-equation} 有解,此时
\begin{equation}
  \tilde{A}^{(\mu)}_{ii'} = \sigma_\mu.
\end{equation}
这意味着当且仅当 $B=\sigma_k$ 而 $k\in\{1,2,\dots,n-1\}$ 时($\sigma_0=\mathbb{I}$ 对应了同样是平凡算符的 $A=\mathbb{I}\otimes\mathbb{I}$,因此将其忽略),对应的
\begin{equation}
  A = \sigma_k   \otimes \mathbb{I} \quad \text{or} \quad
  A = \mathbb{I} \otimes \sigma_k
\end{equation}
为简单算符。由于 $\tilde{A}=\sigma_k$ 可以作为新一层的体算符 $B$,这样的算符推移操作可以不断进行下去。考虑一个 $L$ 层的张量网络(见图~\ref{fig:rg-1+1d}),对于上述的 $B=\sigma_k$,边界算符仍然具有简单形式,并且可以写为:
\begin{equation}
  A_L = \mathbb{I}^{\otimes L-l} \otimes \sigma_k \otimes \mathbb{I}^{\otimes l-1}, \quad l = 0,\dots,L.
\end{equation}

下面我们以 $\mathbb{Z}_2$ 群(即 $n=2$)为例给出一个具体的解。$\delta_{k,(i+j)\bmod n}$ 的零空间可以由向量组
\begin{equation}
  \{ v^{(p)} \} = \Biggl\{ \,
    \begin{pmatrix} 1 \\ 0 \\ 0 \\ -1 \end{pmatrix}, \,
    \begin{pmatrix} 0 \\ 1 \\ -1 \\ 0 \end{pmatrix} \,
  \Biggr\}
\end{equation}
张成,这等价于通解
\begin{equation}
  A^* = \begin{pmatrix}
    \beta_{0,0} & \beta_{0,1} & -\beta_{0,1} & -\beta_{0,0} \\
    \beta_{1,0} & \beta_{1,1} & -\beta_{1,1} & -\beta_{1,0} \\
    \beta_{2,0} & \beta_{2,1} & -\beta_{2,1} & -\beta_{2,0} \\
    \beta_{3,0} & \beta_{3,1} & -\beta_{3,1} & -\beta_{3,0}
  \end{pmatrix},
\end{equation}
其中 $\beta_{i,j}$ 是任意常数。因而完整的解可以表示为
\begin{align}
  B = \sigma_0 = \mathbb{I}   &\implies A = A^* + \begin{pmatrix} 1 &  0 & 0 & 0 \\ 0 &  1 & 0 & 0 \\ 0 &  1 & 0 & 0 \\ 1 &  0 & 0 & 0 \end{pmatrix}, \notag \displaybreak[0] \\
  B = \sigma_1 = \sigma_x     &\implies A = A^* + \begin{pmatrix} 0 &  1 & 0 & 0 \\ 1 &  0 & 0 & 0 \\ 1 &  0 & 0 & 0 \\ 0 &  1 & 0 & 0 \end{pmatrix}, \notag \displaybreak[0] \\
  B = \sigma_2 = \sigma_z     &\implies A = A^* + \begin{pmatrix} 1 &  0 & 0 & 0 \\ 0 & -1 & 0 & 0 \\ 0 & -1 & 0 & 0 \\ 1 &  0 & 0 & 0 \end{pmatrix}, \notag \displaybreak[0] \\
  B = \sigma_3 = -\ii\sigma_y &\implies A = A^* + \begin{pmatrix} 0 & -1 & 0 & 0 \\ 1 &  0 & 0 & 0 \\ 1 &  0 & 0 & 0 \\ 0 & -1 & 0 & 0 \end{pmatrix}.
  \label{eq:Z_2-solution}
\end{align}
可以看到当且仅当体算符 $B=\sigma_x$ 时,边界算符 $A$ 可以取到简单形式
\begin{equation}
  A = \mathbb{I} \otimes \sigma_x = \begin{pmatrix}
    0 & 1 & 0 & 0 \\
    1 & 0 & 0 & 0 \\
    0 & 0 & 0 & 1 \\
    0 & 0 & 1 & 0
  \end{pmatrix}
  \quad \text{or} \quad
  A = \sigma_x \otimes \mathbb{I} = \begin{pmatrix}
    0 & 0 & 1 & 0 \\
    0 & 0 & 0 & 1 \\
    1 & 0 & 0 & 0 \\
    0 & 1 & 0 & 0
  \end{pmatrix}.
\end{equation}

\subsection{非 Abelian 的例子:\texorpdfstring{$S_3$}{𝑆₃} 群}

$S_3$ 群的乘法表为:
\begin{center}
  \begin{tabular}{c|cccccc}
    & $g_0$ & $g_1$ & $g_2$ & $g_3$ & $g_4$ & $g_5$ \\
    \hline
    $g_0$ & 0 & 1 & 2 & 3 & 4 & 5 \\
    $g_1$ & 1 & 0 & 3 & 2 & 5 & 4 \\
    $g_2$ & 2 & 4 & 0 & 5 & 1 & 3 \\
    $g_3$ & 3 & 5 & 1 & 4 & 0 & 2 \\
    $g_4$ & 4 & 2 & 5 & 0 & 3 & 1 \\
    $g_5$ & 5 & 3 & 4 & 1 & 2 & 0 \\
  \end{tabular}
\end{center}
我们取
\begin{equation}
  G(i,j) = g_i g_j,
\end{equation}
则可知 $G(1,2)=g_1 g_2=g_3\eqcolon3$,$G(2,1)=g_2 g_1=g_4\eqcolon4$,显然关于 $i$、$j$ 是不对称的。

\section{2+1 维中的算符推移}

\subsection{例子:\texorpdfstring{$\mathbb{Z}_n$}{ℤₙ} 模型}

\subsection{例子:Fibonacci 模型}

\section{本章小结}

\chapter{总结与展望}
\label{chap:conclusion}

在本文中,我们首先回顾了拓扑序以及描述它的数学框架——张量融合范畴,并介绍了张量网络这一能够有效处理量子多体问题的数值方法。接下来,我们给出了弦网模型基态波函数的张量网络表示,并通过奇异关联子将其与对应临界格点模型的配分函数联系起来。在这种基于拓扑序的张量网络中,本文进一步探讨了一些值得注意的问题,并得到了下面这些结论:

\begin{enumerate}
  \item 在奇异关联子的构造中,边界条件 $\bra{\Omega}$ 可以通过求解重整化算符的本征值问题来获得。这些重整化算符同时实现了一种全息张量网络:边界对应于共形场论,而体则对应了拓扑序(或拓扑量子场论)。我们利用这套方案数值计算了 2+1 维 $\mathcal{A}_{k+1}$ 模型中的耦合常数和 3+1 维 Ising 模型中的临界点,并发现它们与理论结果是基本一致的。
  \item 利用这种全息张量网络,我们还在二维 Ising 模型中计算了体—边传播子,其结果与 AdS/CFT 字典所给出的结论基本吻合。
  \item 在一般性的张量网络中,Virasoro 与 Kac--Moody 算符可以通过一种不依赖于 Hamilton 量的通用方式来构造:首先将转移矩阵表示为一个圆柱形的张量网络,对其进行精确对角化可以得到能动量张量(或流算符),将其变形后插回圆柱并添加对应的系数,即可给出相应的 Virasoro 与 Kac--Moody 生成元。Ising 和 dimer 模型中的数值结果表明,即使只使用很小的圆柱,这套方案也能达到较好的精度。
  \item 对于有限尺寸效应较大的系统(例如 Fibonacci 等拓扑序模型),拓扑投影算符(中心幂等元)可以帮助我们有效地计算并确定能动量张量,进而得到相应的 Virasoro 算符。
  \item 全息张量网络中的算符推移描述了穿越重整化算符时广义自由场的演化,这也相当于给出了利用体算符来重建边界算符的方案。算符推移的具体形式可以通过一组线性方程来描述,求解这组方程可以得到边界算符为广义自由场的条件。对于 1+1 维中的 $Z_n$ 群和 $S_3$ 群,我们还证明了其中广义自由场的数目等于群的阶数。
\end{enumerate}

本文关于全息张量网络的研究只是一个初步的框架,仍有许多方面可以改进和探索。理论方面,我们注意到奇异关联子、MPO 对称性等与范畴对称性\cite{ji2020categorical,kong2020algebraic,wu2021categorical,lootens2023dualities}\allowbreak 有关,而后者描述了拓扑缺陷的代数结构,这种联系值得进一步探讨。数值方面,本文中全息张量网络的构造比较粗糙,例如连接维数只取到了 1。通过优化重整化算符的结构以及相应的缩并算法,我们有望进一步提高计算规模和精度。此外,本文对于全息对偶的探索还非常基础,仅研究了体—边传播子和算符推移。我们希望能够在更多的模型中计算 AdS/CFT 字典的更多内涵,例如关联函数、纠缠楔重构等。对于算符推移,本文主要讨论了几个具体的模型,但对一般的群和张量范畴并没有给出一般性的结论,这也有待后续的研究。

关于 Virasoro 与 Kac--Moody 算符在张量网络中的构造,我们注意到,即使是通过较小的圆柱构造出的 Virasoro 生成元,在更大的圆柱中实际上也是有效的(即仍然可以近似作为升降算符),这也意味着其中可能存在有新的代数结构。另一方面,这套方案也可以被推广并用于研究其他的扩展对称性,例如 Potts 模型中的 $\mathcal{W}$ 代数\cite{fateev1987conformal}。对于更大的系统,我们也可以尝试利用 TNR 等重整化算法进行粗粒近似,以此进一步减少有限尺寸效应的影响。


\appendix

\chapter{图表与公式}

\subcaptionsetup[figure]{skip=0pt}

\begin{figure}[ht]
  \vspace*{-1.2cm}
  \centering
  \subcaptionbox{$      L_{-1}$}{\includegraphics[width=0.42\textwidth]{images/virasoro/ising-lm1.pdf}}    \qquad
  \subcaptionbox{$\bar{L}_{-1}$}{\includegraphics[width=0.42\textwidth]{images/virasoro/ising-lbarm1.pdf}} \\
  \subcaptionbox{$      L_{+1}$}{\includegraphics[width=0.42\textwidth]{images/virasoro/ising-l1.pdf}}     \qquad
  \subcaptionbox{$\bar{L}_{+1}$}{\includegraphics[width=0.42\textwidth]{images/virasoro/ising-lbar1.pdf}}  \\
  \subcaptionbox{$      L_{-2}$}{\includegraphics[width=0.42\textwidth]{images/virasoro/ising-lm2.pdf}}    \qquad
  \subcaptionbox{$\bar{L}_{-2}$}{\includegraphics[width=0.42\textwidth]{images/virasoro/ising-lbarm2.pdf}} \\
  \subcaptionbox{$      L_{+2}$}{\includegraphics[width=0.42\textwidth]{images/virasoro/ising-l2.pdf}}     \qquad
  \subcaptionbox{$\bar{L}_{+2}$}{\includegraphics[width=0.42\textwidth]{images/virasoro/ising-lbar2.pdf}}
  \caption[Virasoro 算符在 Ising 模型能谱上的作用]{Virasoro 算符在 Ising 模型能谱上的作用。图片来源:\parencite{wang2022virasoro}。}
  \vspace*{-1.5cm}
  \label{fig:ising-virasoro-all}
\end{figure}

\begin{figure}[ht]
  \centering
  \subcaptionbox{$      J_{-1}$}{\includegraphics[width=0.42\textwidth]{images/virasoro/dimer-jm1.pdf}}    \qquad
  \subcaptionbox{$\bar{J}_{-1}$}{\includegraphics[width=0.42\textwidth]{images/virasoro/dimer-jbarm1.pdf}} \\
  \subcaptionbox{$      J_{-2}$}{\includegraphics[width=0.42\textwidth]{images/virasoro/dimer-jm2.pdf}}    \qquad
  \subcaptionbox{$\bar{J}_{-2}$}{\includegraphics[width=0.42\textwidth]{images/virasoro/dimer-jbarm2.pdf}}
  \caption[Kac--Moody 算符在 dimer 模型能谱上的作用]{Kac--Moody 算符在 dimer 模型能谱上的作用。图片来源:\parencite{wang2022virasoro}。}
  \label{fig:dimer-kac-moody-all}
\end{figure}

\begin{figure}[ht]
  \centering
  \includegraphics[width=\textwidth]{images/fibonacci/fib-virasoro-all-1.pdf} \\
  \includegraphics[width=\textwidth]{images/fibonacci/fib-virasoro-all-2.pdf}
  \caption[Virasoro 算符在 Fibonacci 模型能谱上的作用]{Virasoro 算符在 Fibonacci 模型能谱上的作用。图片来源:\parencite{zeng2023virasoro}。}
  \label{fig:fib-virasoro-all}
\end{figure}

\begin{sidewaystable}[ht]
  % \special{pdf: put @thispage <</Rotate 90>>}
  \centering
  \renewcommand{\arraystretch}{1}
  \begin{tabular}{*{8}{c}>{\addfontfeatures{Fractions=On}}c}
    \toprule
      \diagbox{自旋}{标度维数}{圆柱尺寸}
                  & 4        & 8        & 12       & 16       & 20       & 24       & $\infty$     & 理论值 \\
    \midrule
      0           & 0.123499 & 0.124606 & 0.124823 & 0.124900 & 0.124936 & 0.124955 & 0.12522\,(6) & 1/8   \\
      0           & 1.026740 & 1.006488 & 1.002868 & 1.001610 & 1.001030 & 1.000715 & 0.9961\,(11) & 1     \\
      $\pm1$      & 1.245699 & 1.151345 & 1.136446 & 1.131388 & 1.129074 & 1.127824 & 1.107\,(5)   & 1+1/8 \\
      $\pm1,\pm2$ & 2.569513 & 2.098365 & 2.041544 & 2.022974 & 2.014590 & 2.010090 & 1.916\,(30)  & 2     \\
      0           & 2.367899 & 2.178085 & 2.148069 & 2.137876 & 2.133212 & 2.130692 & 2.089\,(11)  & 2+1/8 \\
      $\pm2$      & -        & 2.369005 & 2.223018 & 2.178379 & 2.158670 & 2.148201 & 2.043\,(19)  & 2+1/8 \\
    \bottomrule
  \end{tabular}
  \caption[Ising 模型的能谱数据]{Ising 模型的能谱数据。圆柱(转移矩阵)的尺寸 $n$ 由 4 取到 24,并且外推至无穷大。注意此处 $\Delta_\alpha$ 没有根据 $\Delta_T=2$ 的性质来进行标定。}
  \label{tab:ising-spectrum}
\end{sidewaystable}

\begin{sidewaystable}[ht]
  % \special{pdf: put @thispage <</Rotate 270>>}
  \centering
  \renewcommand{\arraystretch}{1}
  \begin{threeparttable}
    \catcode`\"=\active
    \def"#1"{\textbf{#1}}
    \def\TPTtagStyle#1{\textit{#1}}
    \begin{tabular}{*{10}{c}>{\addfontfeatures{Fractions=On}}c}
      \toprule
        \diagbox{自旋}{标度\\[-0.5ex]维数}{圆柱\\[-0.5ex]尺寸}
                     &  9                  &  12        &  15        &  18        &  21        &  24        &  27        &  $\infty$     &  调整值       &  理论值 \\
      \midrule
        "0"          & "0.0"               & "0.0"      & "0.0"      & "0.0"      & "0.0"      & "0.0"      & "0.0"      & "0.0"         & "0.0"         & "0"     \\
         0\tnote{a}  &  0.112899           &  0.113952  &  0.114474  &  0.114769  &  0.114950  &  0.115070  &  0.115152  &  0.11610\,(7) &  0.1513\,(32) &  2/15   \\
         0           &  0.722756           &  0.709100  &  0.703083  &  0.699890  &  0.697989  &  0.696765  &  0.695931  &  0.6856\,(9)  &  0.894\,(19)  &  4/5    \\
         $\pm1$      &  0.785553           &  0.817591  &  0.841528  &  0.859630  &  0.873686  &  0.884880  &  0.893989  &  0.9557\,(27) &  1.246\,(27)  &  1+2/15 \\
         $\pm1$      &  1.594955           &  1.346025  &  1.242196  &  1.184631  &  1.147942  &  1.122492  &  1.103796  &  0.914\,(11)  &  1.191\,(29)  &  1+2/15 \\
        "0"\tnote{a} & "1.290074"          & "1.222673" & "1.196184" & "1.182805" & "1.175050" & "1.170137" & "1.166820" & "1.123\,(4)"  & "1.464\,(32)" & "4/3"   \\
         $\pm1$      &  1.147427           &  1.222649  &  1.274342  &  1.312081  &  1.340887  &  1.363620  &  1.382029  &  1.510\,(4)   &  1.97\,(4)    &  1+4/5  \\
         $\pm1$      &  3.221770           &  2.490495  &  2.169235  &  2.016147  &  1.925379  &  1.865017  &  1.821876  &  1.31\,(5)    &  1.70\,(8)    &  1+4/5  \\
        "\textpm2"   & "2.727358"\tnote{b} & "2.139950" & "1.967143" & "1.887302" & "1.842910" & "1.815429" & "1.797149" & "1.535\,(33)" & "2.00\,(6)"   & "2"     \\
         $\pm2$      &  1.032723\tnote{b}  &  1.178227  &  1.276463  &  1.348394  &  1.403716  &  1.447739  &  1.483672  &  1.730\,(9)   &  2.25\,(5)    &  2+2/15 \\
         $\pm2$      &  -\tnote{b}         &  1.430687  &  1.484488  &  1.527193  &  1.561317  &  1.589035  &  1.611935  &  1.759\,(9)   &  2.29\,(5)    &  2+2/15 \\
      \bottomrule
    \end{tabular}
    \begin{tablenotes}
      \item[a] 这些能级带有二重简并。
      \item[b] 对于 $n=9$ 的情况,转移矩阵不够大,因此一些高能级的后代没有出现;同时,自旋 $\pm2$ 会退化为 $\pm1$。
    \end{tablenotes}
    \caption[Fibonacci 模型的能谱数据]{Fibonacci 模型的能谱数据。真空态及其后代(标记为粗体)通过对角化带有幂等元的转移矩阵确定,见 \ref{subsec:topological-projectors} 小节。$n=\infty$ 处的数据通过对尺寸由 12 到 27 的圆柱(转移矩阵)本征值外推得到。}
    \label{tab:fib-spectrum}
  \end{threeparttable}
\end{sidewaystable}

\cleardoublepage

在 \ref{subsec:operator-pushing-fib} 小节中,2+1 维 Fibonacci 模型的通解由式~\eqref{eq:2+1d-fib-null-space} 中向量 $v^{(p)}$ 的线性组合与特解部分 $A^{(\mu)}$ 共同组成,且分别对应不同的 $B=\sigma_\mu$。这些 $A^{(\mu)}$ 是 $8\times8$ 的矩阵,它们的最后 3 列均为零,即
\begin{equation}
  A^{(\mu)} = \begin{pmatrix}
    \tilde{A}^{(\mu)} & \mathbf{0} & \mathbf{0} & \mathbf{0}
  \end{pmatrix}.
  \label{eq:2+1d-fib-solution-1}
\end{equation}
这里 $8\times5$ 的矩阵 $\tilde{A}^{(\mu)}$ 由下式给出,其中 $\varphi=(1+\sqrt5)/2$,$\omega=\ee^{\ii\pi/5}$:
\begingroup
\small
\allowdisplaybreaks
\begin{align}
  \tilde{A}^{(0)} &= \begin{pmatrix}
    1 & 0 & 0 & 0 & 0 \\
    0 & 1 & 0 & 0 & 0 \\
    0 & 0 & 1 & 0 & 0 \\
    0 & 0 & 0 & 1 & 0 \\
    0 & 0 & 0 & 0 & 1 \\
    0 & 0 & 1 & 1 & -1 \\
    0 & 1 & 0 & 1 & -1 \\
    \varphi & \sqrt{\varphi} & \sqrt{\varphi} & -\frac{1}{\sqrt{\varphi}} & \frac{\varphi+1}{\sqrt{\varphi}} \\
  \end{pmatrix}, &
  \tilde{A}^{(1)} &= \begin{pmatrix}
    0 & 0 & 0 & 0 & \frac{1}{\varphi} \\
    0 & 0 & 1 & 0 & 0 \\
    0 & 0 & 0 & \frac{1}{\sqrt{\varphi}} & -\frac{1}{\sqrt{\varphi}} \\
    \varphi^{3/2} & 1 & 0 & 0 & 0 \\
    0 & 1 & 0 & 0 & 0 \\
    \varphi^{3/2} & 0 & 0 & \frac{1}{\sqrt{\varphi}} & -\frac{1}{\sqrt{\varphi}} \\
    \varphi^{3/2} & 0 & 1 & 0 & 0 \\
    -\varphi & \sqrt{\varphi} & \sqrt{\varphi} & 1 & 0 \\
  \end{pmatrix}, \notag \\[2ex]
  %
  \tilde{A}^{(2)} &= \begin{pmatrix}
    0 & \frac{1}{\varphi} & 0 & 0 & 0 \\
    0 & 0 & 0 & \frac{1}{\sqrt{\varphi}} & -\frac{1}{\sqrt{\varphi}} \\
    \varphi & 0 & 0 & 0 & 0 \\
    0 & 0 & 1 & 0 & \sqrt{\varphi} \\
    0 & 0 & 1 & 0 & 0 \\
    \varphi & 0 & 0 & 0 & \sqrt{\varphi} \\
    0 & 0 & 0 & \frac{1}{\sqrt{\varphi}} & \frac{\varphi-1}{\sqrt{\varphi}} \\
    \varphi^{3/2} & 1 & \sqrt{\varphi} & 1 & -2 \\
  \end{pmatrix}, &
  \tilde{A}^{(3)} &= \begin{pmatrix}
    0 & 0 & \frac{1}{\varphi} & 0 & 0 \\
    \varphi & 0 & 0 & 0 & 0 \\
    0 & 0 & 0 & 0 & 1 \\
    0 & \sqrt{\varphi} & 0 & \frac{1}{\sqrt{\varphi}} & -\frac{1}{\sqrt{\varphi}} \\
    0 & 0 & 0 & \frac{1}{\sqrt{\varphi}} & -\frac{1}{\sqrt{\varphi}} \\
    0 & \sqrt{\varphi} & 0 & 0 & 1 \\
    \varphi & \sqrt{\varphi} & 0 & 0 & 0 \\
    \varphi^{3/2} & -1 & 1 & 1 & \sqrt{\varphi}-1 \\
  \end{pmatrix}, \notag \\[2ex]
  %
  \tilde{A}^{(4)} &= \begin{pmatrix}
    0 & 0 & 0 & \frac{1}{\varphi^{3/2}} & -\frac{1}{\varphi^{3/2}} \\
    0 & 0 & 0 & 0 & 1 \\
    0 & 1 & 0 & 0 & 0 \\
    \varphi & 0 & \sqrt{\varphi} & 0 & 0 \\
    \varphi & 0 & 0 & 0 & 0 \\
    0 & 1 & \sqrt{\varphi} & 0 & 0 \\
    0 & 0 & \sqrt{\varphi} & 0 & 1 \\
    \varphi^{3/2} & \sqrt{\varphi} & -1 & \frac{1}{\sqrt{\varphi}} & \frac{\varphi-1}{\sqrt{\varphi}} \\
  \end{pmatrix}, &
  \tilde{A}^{(5)} &= \begin{pmatrix}
    1 & 0 & 0 & 0 & 0 \\
    0 & \omega^4 & 0 & 0 & 0 \\
    0 & 0 & -\omega & 0 & 0 \\
    0 & 0 & 0 & -\omega^3 & \omega^3+\omega^2 \\
    0 & 0 & 0 & 0 & \omega^2 \\
    0 & 0 & -\omega & -\omega^3 & \omega^3 \\
    0 & \omega^4 & 0 & -\omega^3 & \omega^3 \\
    \varphi & \omega^4 \sqrt{\varphi} & -\omega\sqrt{\varphi} & \frac{\omega^3}{\sqrt{\varphi}} & \frac{\omega^2 \varphi-\omega^3}{\sqrt{\varphi}} \\
  \end{pmatrix}, \notag \\[2ex]
  %
  \tilde{A}^{(6)} &= \begin{pmatrix}
    0 & 0 & 0 & 0 & \frac{\omega^2}{\varphi} \\
    0 & 0 & -\omega & 0 & 0 \\
    0 & 0 & 0 & -\frac{\omega^3}{\sqrt{\varphi}} & \frac{\omega^3}{\sqrt{\varphi}} \\
    \varphi^{3/2} & \omega^4 & 0 & 0 & 0 \\
    0 & \omega^4 & 0 & 0 & 0 \\
    \varphi^{3/2} & 0 & 0 & -\frac{\omega^3}{\sqrt{\varphi}} & \frac{\omega^3}{\sqrt{\varphi}} \\
    \varphi^{3/2} & 0 & -\omega & 0 & 0 \\
    -\varphi & \omega^4 \sqrt{\varphi} & -\omega\sqrt{\varphi} & -\omega^3 & \omega^3+\omega^2 \\
  \end{pmatrix}, &
  \tilde{A}^{(7)} &= \begin{pmatrix}
    0 & \frac{\omega^4}{\varphi} & 0 & 0 & 0 \\
    0 & 0 & 0 & -\frac{\omega^3}{\sqrt{\varphi}} & \frac{\omega^3}{\sqrt{\varphi}} \\
    \varphi & 0 & 0 & 0 & 0 \\
    0 & 0 & -\omega & 0 & \omega^2 \sqrt{\varphi} \\
    0 & 0 & -\omega & 0 & 0 \\
    \varphi & 0 & 0 & 0 & \omega^2 \sqrt{\varphi} \\
    0 & 0 & 0 & -\frac{\omega^3}{\sqrt{\varphi}} & \frac{\omega^3+\omega^2 \varphi}{\sqrt{\varphi}} \\
    \varphi^{3/2} & \omega^4 & -\omega\sqrt{\varphi} & -\omega^3 & \omega^3-\omega^2 \\
  \end{pmatrix}, \notag \\[2ex]
  %
  \tilde{A}^{(8)} &= \begin{pmatrix}
    0 & 0 & -\frac{\omega}{\varphi} & 0 & 0 \\
    \varphi & 0 & 0 & 0 & 0 \\
    0 & 0 & 0 & 0 & \omega^2 \\
    0 & \omega^4 \sqrt{\varphi} & 0 & -\frac{\omega^3}{\sqrt{\varphi}} & \frac{\omega^3}{\sqrt{\varphi}} \\
    0 & 0 & 0 & -\frac{\omega^3}{\sqrt{\varphi}} & \frac{\omega^3}{\sqrt{\varphi}} \\
    0 & \omega^4 \sqrt{\varphi} & 0 & 0 & \omega^2 \\
    \varphi & \omega^4 \sqrt{\varphi} & 0 & 0 & 0 \\
    \varphi^{3/2} & -\omega^4 & -\omega & -\omega^3 & \omega^3+\omega^2 \sqrt{\varphi} \\
  \end{pmatrix}, &
  \tilde{A}^{(9)} &= \begin{pmatrix}
    0 & 0 & 0 & -\frac{\omega^3}{\varphi^{3/2}} & \frac{\omega^3}{\varphi^{3/2}} \\
    0 & 0 & 0 & 0 & \omega^2 \\
    0 & \omega^4 & 0 & 0 & 0 \\
    \varphi & 0 & -\omega\sqrt{\varphi} & 0 & 0 \\
    \varphi & 0 & 0 & 0 & 0 \\
    0 & \omega^4 & -\omega\sqrt{\varphi} & 0 & 0 \\
    0 & 0 & -\omega\sqrt{\varphi} & 0 & \omega^2 \\
    \varphi^{3/2} & \omega^4 \sqrt{\varphi} & \omega & -\frac{\omega^3}{\sqrt{\varphi}} & \frac{\omega^3+\omega^2 \varphi}{\sqrt{\varphi}} \\
  \end{pmatrix}, \notag \\[2ex]
  %
  \tilde{A}^{(10)} &= \begin{pmatrix}
    1 & 0 & 0 & 0 & 0 \\
    0 & -\omega^3 & 0 & 0 & 0 \\
    0 & 0 & \omega^2 & 0 & 0 \\
    0 & 0 & 0 & -\omega & \omega^4+\omega \\
    0 & 0 & 0 & 0 & \omega^4 \\
    0 & 0 & \omega^2 & -\omega & \omega \\
    0 & -\omega^3 & 0 & -\omega & \omega \\
    \varphi & -\omega^3\sqrt{\varphi} & \omega^2 \sqrt{\varphi} & \frac{\omega}{\sqrt{\varphi}} & \frac{\omega^4 \varphi-\omega}{\sqrt{\varphi}} \\
  \end{pmatrix}, &
  \tilde{A}^{(11)} &= \begin{pmatrix}
    0 & 0 & 0 & 0 & \frac{\omega^4}{\varphi} \\
    0 & 0 & \omega^2 & 0 & 0 \\
    0 & 0 & 0 & -\frac{\omega}{\sqrt{\varphi}} & \frac{\omega}{\sqrt{\varphi}} \\
    \varphi^{3/2} & -\omega^3 & 0 & 0 & 0 \\
    0 & -\omega^3 & 0 & 0 & 0 \\
    \varphi^{3/2} & 0 & 0 & -\frac{\omega}{\sqrt{\varphi}} & \frac{\omega}{\sqrt{\varphi}} \\
    \varphi^{3/2} & 0 & \omega^2 & 0 & 0 \\
    -\varphi & -\omega^3\sqrt{\varphi} & \omega^2 \sqrt{\varphi} & -\omega & \omega^4+\omega \\
  \end{pmatrix}, \notag \\[2ex]
  %
  \tilde{A}^{(12)} &= \begin{pmatrix}
    0 & -\frac{\omega^3}{\varphi} & 0 & 0 & 0 \\
    0 & 0 & 0 & -\frac{\omega}{\sqrt{\varphi}} & \frac{\omega}{\sqrt{\varphi}} \\
    \varphi & 0 & 0 & 0 & 0 \\
    0 & 0 & \omega^2 & 0 & \omega^4 \sqrt{\varphi} \\
    0 & 0 & \omega^2 & 0 & 0 \\
    \varphi & 0 & 0 & 0 & \omega^4 \sqrt{\varphi} \\
    0 & 0 & 0 & -\frac{\omega}{\sqrt{\varphi}} & \frac{\omega+\omega^4 \varphi}{\sqrt{\varphi}} \\
    \varphi^{3/2} & -\omega^3 & \omega^2 \sqrt{\varphi} & -\omega & \omega-\omega^4 \\
  \end{pmatrix}, &
  \tilde{A}^{(13)} &= \begin{pmatrix}
    0 & 0 & \frac{\omega^2}{\varphi} & 0 & 0 \\
    \varphi & 0 & 0 & 0 & 0 \\
    0 & 0 & 0 & 0 & \omega^4 \\
    0 & -\omega^3\sqrt{\varphi} & 0 & -\frac{\omega}{\sqrt{\varphi}} & \frac{\omega}{\sqrt{\varphi}} \\
    0 & 0 & 0 & -\frac{\omega}{\sqrt{\varphi}} & \frac{\omega}{\sqrt{\varphi}} \\
    0 & -\omega^3\sqrt{\varphi} & 0 & 0 & \omega^4 \\
    \varphi & -\omega^3\sqrt{\varphi} & 0 & 0 & 0 \\
    \varphi^{3/2} & \omega^3 & \omega^2 & -\omega & \omega+\omega^4 \sqrt{\varphi} \\
  \end{pmatrix}, \notag \\[2ex]
  %
  \tilde{A}^{(14)} &= \begin{pmatrix}
    0 & 0 & 0 & -\frac{\omega}{\varphi^{3/2}} & \frac{\omega}{\varphi^{3/2}} \\
    0 & 0 & 0 & 0 & \omega^4 \\
    0 & -\omega^3 & 0 & 0 & 0 \\
    \varphi & 0 & \omega^2 \sqrt{\varphi} & 0 & 0 \\
    \varphi & 0 & 0 & 0 & 0 \\
    0 & -\omega^3 & \omega^2 \sqrt{\varphi} & 0 & 0 \\
    0 & 0 & \omega^2 \sqrt{\varphi} & 0 & \omega^4 \\
    \varphi^{3/2} & -\omega^3\sqrt{\varphi} & -\omega^2 & -\frac{\omega}{\sqrt{\varphi}} & \frac{\omega+\omega^4 \varphi}{\sqrt{\varphi}} \\
  \end{pmatrix}, &
  \tilde{A}^{(15)} &= \begin{pmatrix}
    1 & 0 & 0 & 0 & 0 \\
    0 & \omega^2 & 0 & 0 & 0 \\
    0 & 0 & -\omega^3 & 0 & 0 \\
    0 & 0 & 0 & \omega^4 & -\omega^4-\omega \\
    0 & 0 & 0 & 0 & -\omega \\
    0 & 0 & -\omega^3 & \omega^4 & -\omega^4 \\
    0 & \omega^2 & 0 & \omega^4 & -\omega^4 \\
    \varphi & \omega^2 \sqrt{\varphi} & -\omega^3\sqrt{\varphi} & -\frac{\omega^4}{\sqrt{\varphi}} & \frac{\omega^4-\omega \varphi}{\sqrt{\varphi}} \\
  \end{pmatrix}, \notag \\[2ex]
  %
  \tilde{A}^{(16)} &= \begin{pmatrix}
    0 & 0 & 0 & 0 & -\frac{\omega}{\varphi} \\
    0 & 0 & -\omega^3 & 0 & 0 \\
    0 & 0 & 0 & \frac{\omega^4}{\sqrt{\varphi}} & -\frac{\omega^4}{\sqrt{\varphi}} \\
    \varphi^{3/2} & \omega^2 & 0 & 0 & 0 \\
    0 & \omega^2 & 0 & 0 & 0 \\
    \varphi^{3/2} & 0 & 0 & \frac{\omega^4}{\sqrt{\varphi}} & -\frac{\omega^4}{\sqrt{\varphi}} \\
    \varphi^{3/2} & 0 & -\omega^3 & 0 & 0 \\
    -\varphi & \omega^2 \sqrt{\varphi} & -\omega^3\sqrt{\varphi} & \omega^4 & -\omega^4-\omega \\
  \end{pmatrix}, &
  \tilde{A}^{(17)} &= \begin{pmatrix}
    0 & \frac{\omega^2}{\varphi} & 0 & 0 & 0 \\
    0 & 0 & 0 & \frac{\omega^4}{\sqrt{\varphi}} & -\frac{\omega^4}{\sqrt{\varphi}} \\
    \varphi & 0 & 0 & 0 & 0 \\
    0 & 0 & -\omega^3 & 0 & -\omega\sqrt{\varphi} \\
    0 & 0 & -\omega^3 & 0 & 0 \\
    \varphi & 0 & 0 & 0 & -\omega\sqrt{\varphi} \\
    0 & 0 & 0 & \frac{\omega^4}{\sqrt{\varphi}} & \frac{-\omega^4-\omega \varphi}{\sqrt{\varphi}} \\
    \varphi^{3/2} & \omega^2 & -\omega^3\sqrt{\varphi} & \omega^4 & \omega-\omega^4 \\
  \end{pmatrix}, \notag \\[2ex]
  %
  \tilde{A}^{(18)} &= \begin{pmatrix}
    0 & 0 & -\frac{\omega^3}{\varphi} & 0 & 0 \\
    \varphi & 0 & 0 & 0 & 0 \\
    0 & 0 & 0 & 0 & -\omega \\
    0 & \omega^2 \sqrt{\varphi} & 0 & \frac{\omega^4}{\sqrt{\varphi}} & -\frac{\omega^4}{\sqrt{\varphi}} \\
    0 & 0 & 0 & \frac{\omega^4}{\sqrt{\varphi}} & -\frac{\omega^4}{\sqrt{\varphi}} \\
    0 & \omega^2 \sqrt{\varphi} & 0 & 0 & -\omega \\
    \varphi & \omega^2 \sqrt{\varphi} & 0 & 0 & 0 \\
    \varphi^{3/2} & -\omega^2 & -\omega^3 & \omega^4 & -\omega^4-\omega \sqrt{\varphi} \\
  \end{pmatrix}, &
  \tilde{A}^{(19)} &= \begin{pmatrix}
    0 & 0 & 0 & \frac{\omega^4}{\varphi^{3/2}} & -\frac{\omega^4}{\varphi^{3/2}} \\
    0 & 0 & 0 & 0 & -\omega \\
    0 & \omega^2 & 0 & 0 & 0 \\
    \varphi & 0 & -\omega^3\sqrt{\varphi} & 0 & 0 \\
    \varphi & 0 & 0 & 0 & 0 \\
    0 & \omega^2 & -\omega^3\sqrt{\varphi} & 0 & 0 \\
    0 & 0 & -\omega^3\sqrt{\varphi} & 0 & -\omega \\
    \varphi^{3/2} & \omega^2 \sqrt{\varphi} & \omega^3 & \frac{\omega^4}{\sqrt{\varphi}} & \frac{-\omega^4-\omega \varphi}{\sqrt{\varphi}} \\
  \end{pmatrix}, \notag \\[2ex]
  %
  \tilde{A}^{(20)} &= \begin{pmatrix}
    1 & 0 & 0 & 0 & 0 \\
    0 & -\omega & 0 & 0 & 0 \\
    0 & 0 & \omega^4 & 0 & 0 \\
    0 & 0 & 0 & \omega^2 & -\omega^3-\omega^2 \\
    0 & 0 & 0 & 0 & -\omega^3 \\
    0 & 0 & \omega^4 & \omega^2 & -\omega^2 \\
    0 & -\omega & 0 & \omega^2 & -\omega^2 \\
    \varphi & -\omega\sqrt{\varphi} & \omega^4 \sqrt{\varphi} & -\frac{\omega^2}{\sqrt{\varphi}} & \frac{\omega^2-\omega^3 \varphi}{\sqrt{\varphi}} \\
  \end{pmatrix}, &
  \tilde{A}^{(21)} &= \begin{pmatrix}
    0 & 0 & 0 & 0 & -\frac{\omega^3}{\varphi} \\
    0 & 0 & \omega^4 & 0 & 0 \\
    0 & 0 & 0 & \frac{\omega^2}{\sqrt{\varphi}} & -\frac{\omega^2}{\sqrt{\varphi}} \\
    \varphi^{3/2} & -\omega & 0 & 0 & 0 \\
    0 & -\omega & 0 & 0 & 0 \\
    \varphi^{3/2} & 0 & 0 & \frac{\omega^2}{\sqrt{\varphi}} & -\frac{\omega^2}{\sqrt{\varphi}} \\
    \varphi^{3/2} & 0 & \omega^4 & 0 & 0 \\
    -\varphi & -\omega\sqrt{\varphi} & \omega^4 \sqrt{\varphi} & \omega^2 & -\omega^3-\omega^2 \\
  \end{pmatrix}, \notag \\[2ex]
  %
  \tilde{A}^{(22)} &= \begin{pmatrix}
    0 & -\frac{\omega}{\varphi} & 0 & 0 & 0 \\
    0 & 0 & 0 & \frac{\omega^2}{\sqrt{\varphi}} & -\frac{\omega^2}{\sqrt{\varphi}} \\
    \varphi & 0 & 0 & 0 & 0 \\
    0 & 0 & \omega^4 & 0 & -\omega^3\sqrt{\varphi} \\
    0 & 0 & \omega^4 & 0 & 0 \\
    \varphi & 0 & 0 & 0 & -\omega^3\sqrt{\varphi} \\
    0 & 0 & 0 & \frac{\omega^2}{\sqrt{\varphi}} & \frac{-\omega^3\varphi-\omega^2}{\sqrt{\varphi}} \\
    \varphi^{3/2} & -\omega & \omega^4 \sqrt{\varphi} & \omega^2 & \omega^3-\omega^2 \\
  \end{pmatrix}, &
  \tilde{A}^{(23)} &= \begin{pmatrix}
    0 & 0 & \frac{\omega^4}{\varphi} & 0 & 0 \\
    \varphi & 0 & 0 & 0 & 0 \\
    0 & 0 & 0 & 0 & -\omega^3 \\
    0 & -\omega\sqrt{\varphi} & 0 & \frac{\omega^2}{\sqrt{\varphi}} & -\frac{\omega^2}{\sqrt{\varphi}} \\
    0 & 0 & 0 & \frac{\omega^2}{\sqrt{\varphi}} & -\frac{\omega^2}{\sqrt{\varphi}} \\
    0 & -\omega\sqrt{\varphi} & 0 & 0 & -\omega^3 \\
    \varphi & -\omega\sqrt{\varphi} & 0 & 0 & 0 \\
    \varphi^{3/2} & \omega & \omega^4 & \omega^2 & -\omega^3\sqrt{\varphi} -\omega^2 \\
  \end{pmatrix}, \notag \\[2ex]
  %
  \tilde{A}^{(24)} &= \begin{pmatrix}
    0 & 0 & 0 & \frac{\omega^2}{\varphi^{3/2}} & -\frac{\omega^2}{\varphi^{3/2}} \\
    0 & 0 & 0 & 0 & -\omega^3 \\
    0 & -\omega & 0 & 0 & 0 \\
    \varphi & 0 & \omega^4 \sqrt{\varphi} & 0 & 0 \\
    \varphi & 0 & 0 & 0 & 0 \\
    0 & -\omega & \omega^4 \sqrt{\varphi} & 0 & 0 \\
    0 & 0 & \omega^4 \sqrt{\varphi} & 0 & -\omega^3 \\
    \varphi^{3/2} & -\omega\sqrt{\varphi} & -\omega^4 & \frac{\omega^2}{\sqrt{\varphi}} & \frac{-\omega^3\varphi-\omega^2}{\sqrt{\varphi}} \\
  \end{pmatrix}.
  \label{eq:2+1d-fib-solution-2}
\end{align}
\endgroup


\backmatter

\printbibliography

\chapter{博士期间论文发表情况}

\begin{enumerate}[leftmargin=1.2em]
  \item \underline{Xiangdong Zeng} and Ling-Yan Hung.
    Bulk Operator Reconstruction in Topological Tensor Network and Generalized Free Fields.
    \emph{Entropy}, 2023, \textbf{25}(11): 1543.
    DOI: \href{https://doi.org/10.3390/e25111543}{10.3390/e25111543}.
  \item \underline{Xiangdong Zeng}, Ruoshui Wang, Ce Shen, and Ling-Yan Hung.
    Virasoro Generators in the Fibonacci Model Tensor Network: Tackling Finite-size Effects.
    \emph{Physical Review B}, 2023, \textbf{107}(24): 245146.
    DOI: \href{https://doi.org/10.1103/PhysRevB.107.245146}{10.1103/PhysRevB.107.245146}.
  \item Ruoshui Wang, \underline{Xiangdong Zeng}, Ce Shen, and Ling-Yan Hung.
    Virasoro and Kac--Moody Algebra in Generic Tensor Network Representations of Two-dimensional Critical Lattice Partition Functions.
    \emph{Physical Review B}, 2022, \textbf{106}(11): 115116.
    DOI: \href{https://doi.org/10.1103/PhysRevB.106.115116}{10.1103/PhysRevB.106.115116}.
  \item Lin Chen, Haochen Zhang, Kaixin Ji, Ce Shen, Ruoshui Wang, \underline{Xiangdong Zeng}, and Ling-Yan Hung.
    Exact Holographic Tensor Networks --- Constructing CFT$_D$ from TQFT$_{D+1}$.
    \emph{arXiv preprint}, 2022. arXiv: \href{https://arxiv.org/abs/2210.12127}{2210.12127}.
\end{enumerate}

\begin{acknowledgements}

博士的几年注定是一段孤独而漫长的旅程。纵使有老师、同学、朋友、亲人的陪伴,学术高峰和人生低谷也终归需要独自挑战和面对。然而,我仍然要感谢所有伴我走过这段旅程的人,因为“无穷的远方,无数的人们,都和我有关”。

感谢我的导师孔令欣教授。尽管由于疫情等种种原因,我们之间的讨论大都是远程进行,但她对物理的热情和对学术的严谨,却能穿透屏幕深深地感染着我。孔老师是良师,也是益友。在学术上,她凭借独特的眼光带领着我探索各种方向;而在生活中,她以身作则教会了我为人处事的正确态度。虽然我以后可能不再从事理论物理的研究,但孔老师的教诲将始终成为激励我前行的动力。

感谢万义顿、李晓鹏、戚扬、周洋、凌意、Bartek Czech 等各位评阅老师和答辩委员,本文的最终完成离不开他们的指导。

感谢闫晓辉、周雯婷、高太梅、唐诗蕊、肖文、张奕林、李炜等各位老师和辅导员在我博士期间的关怀与帮助,也感谢校工朋友们的工作与付出。

感谢王若水和沈策与我的合作,他们在科研上的宝贵经验让我受益匪浅。感谢杨智、陈超逸、娄家奇、程龙、Gabriel Wong、陈霖、季恺昕、刘希融、劳炳新、涂静欣、吴奇峰、杨宸等课题组的其他成员,和他们的交流也让我获得了很多支持与启发。

感谢王宏宇、李英诚、王思源、陈艳艳、于鑫阳、郝云超、林键、罗雨晨、陈东、夏威、朱家纬、张雯芊、邹杰、叶梦、常远和闵玥洋等办公室里的各位同学,以及邱型泽、南珏、王海、吴亚东、李子亮等各位博后。尽管方向不尽相同,但跨领域的讨论往往能激起更多灵感的火花。科研之外的日常生活更是离不开大家,朝夕相处、互相扶持,愿诸位都能有更美好的明天。

感谢室友李子晗对我阴间作息的包容,也感谢他与我一起熬过封校期间的艰苦时光。感谢郭家祺和费昳帆,他们对论文内容与格式提出了诸多修改建议,也为模板的开发做出了贡献。感谢 $\ket*{\text{Physics}}\otimes\ket*{\text{见证}}$ 群里的同学们,能与他们探讨物理、吐槽生活、针砭时弊,是我的幸运。

我还要特别感谢 3type(三言)、atelierAnchor(锚坞),以及许许多多热爱文字排印与字体设计的小伙伴们,他们为我推开了一扇通往全新世界的大门,让我的内心燃起了新的火苗。

最后,我要感谢父母和家人,是你们给了我生命,并用爱陪伴我的成长。

\end{acknowledgements}


\end{document}
